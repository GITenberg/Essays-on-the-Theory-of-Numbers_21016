%%%%%%%%%%%%%%%%%%%%%%%%%%%%%%%%%%%%%%%%%%%%%%%%%%%%%%%%%%%%%%%%%%%%%%%%%%%
%% Project Gutenberg's Essays on the Theory of Numbers by Rchd. Dedekind %%
%%                                                                       %%
%% This eBook is for the use of anyone anywhere at no cost and with      %%
%% almost no restrictions whatsoever.  You may copy it, give it away or  %%
%% re-use it under the terms of the Project Gutenberg License included   %%
%% with this eBook or online at www.gutenberg.net                        %%
%%                                                                       %%
%%                                                                       %%
%% Packages and substitutions:                                           %%
%%                                                                       %%
%% book   :   Document class.                                            %%
%% amsmath:   Basic AMS math package.                                    %%
%% amssymb:   Basic AMS symbols                                          %%
%% makeidx:   Index generation                                           %%
%% babel:     Hyphenation                                                %%
%% inputenc:  Encoding                                                   %%
%% teubner:   polyphonic Greek support                                   %%
%% graphicx:  for reflectbox                                             %%
%% mathtools: for shortintertext                                         %%
%%                                                                       %%
%% Producer's Comments:                                                  %%
%%                                                                       %%
%% The sequence latex - makeindex - latex is sufficient to get           %%
%% all the internal references right.                                    %%
%%                                                                       %%
%% Both pdflatex (to generate pdf) and latex (to generate dvi, and       %%
%% from this with appropriate tools other formats) work. The book        %%
%% contains no illustrations.                                            %%
%%                                                                       %%
%% Things to Check:                                                      %%
%%                                                                       %%
%% Lprep/gutcheck: OK                                                    %%
%% Lacheck: OK (completely clean)                                        %%
%% PDF pages, excl. Gutenberg boilerplate:  69                           %%
%%   (4 unnumbered, 1-58, 7 un.)                                         %%
%% PDF pages, incl. Gutenberg boilerplate:  78                           %%
%% ToC page numbers: OK                                                  %%
%% Images: None                                                          %%
%% Fonts: OK                                                             %%
%%                                                                       %%
%% Compile History:                                                      %%
%%                                                                       %%
%% Mar 07: Laverock. Compiled with latex:                                %%
%%         [pdfeTeX, Version 3.141592-1.30.6-2.2 (MiKTeX 2.5)]           %%
%%         latex numbers.tex                                             %%
%%         makeindex numbers                                             %%
%%         latex numbers.tex                                             %%
%%                                                                       %%
%% Apr 07: jt. Compiled:                                                 %%
%%             latex numbers  (there is no index)                        %%
%%             latex numbers                                             %%
%%             pdflatex numbers                                          %%
%%%%%%%%%%%%%%%%%%%%%%%%%%%%%%%%%%%%%%%%%%%%%%%%%%%%%%%%%%%%%%%%%%%%%%%%%%%

\documentclass[oneside]{book}
\usepackage[reqno]{amsmath}		% basic math support
\usepackage{amssymb}			% math symbols
\usepackage[greek,english]{babel}	% hyphenation
\usepackage[latin1]{inputenc}		% encoding system
\usepackage[or]{teubner}		% polyphonic Greek support
\usepackage{graphicx} 			% support \reflectbox
\usepackage{mathtools}			% support \shortintertext
\newcommand\partof{\mathrel{\raisebox{0.45ex}{$\mathfrak{3}$}}}   % Part-of symbol
\newcommand\wholeof{\mathrel{\reflectbox{$\partof$}}}             % Whole-of symbol (once only)

%simulated paragraph command (indented)
\newcommand\mypara{\medskip\textbf}

%simulated chapter heading & toc command
\newcommand\mychap[2]{
\medskip\begin{center}{\Large #2}\end{center}
\addtocontents{toc}{\medskip\protect \centering #1\@. #2.\medskip}
}

%simulated section heading & toc command
\newcommand\mysect[3]{
\medskip\begin{center}{\large #2.}\end{center}
\begin{center}\uppercase{#3.}\end{center}
\addcontentsline{toc}{section}{\protect\phantom{#1}#2\@. #3}
}

\begin{document}
%\setstocksize{794pt}{614pt}
%\settrimmedsize{\stockheight}{\stockwidth}{*}
%\settrims{0pt}{0pt}
%\setulmarginsandblock{110pt}{90pt}{*}
%\setlrmarginsandblock{90pt}{90pt}{*}
%\checkandfixthelayout

\thispagestyle{empty}
\small
\begin{verbatim}
Project Gutenberg's Essays on the Theory of Numbers, by Richard Dedekind

This eBook is for the use of anyone anywhere at no cost and with
almost no restrictions whatsoever.  You may copy it, give it away or
re-use it under the terms of the Project Gutenberg License included
with this eBook or online at www.gutenberg.net


Title: Essays on the Theory of Numbers

Author: Richard Dedekind

Translator: Wooster Woodruff Beman

Release Date: April 8, 2007 [EBook #21016]

Language: English

Character set encoding: TeX

*** START OF THE PROJECT GUTENBERG EBOOK THEORY OF NUMBERS ***


Produced by Jonathan Ingram, Keith Edkins and the 
Online Distributed Proofreading Team at http://www.pgdp.net

\end{verbatim}
\normalsize

\frontmatter
\pagestyle {empty}

\bigskip\bigskip

\noindent \textsc{Transcriber's Note:} \emph{The symbol $\partof$ is used as an approximation
to the author's Part-of symbol, not to be confused with the digit $3$.
Internal page references have been been adjusted to fit the pagination of this edition.
A few typographical errors have been corrected - these are noted at the very end of the text.}

\newpage

% [File: 001.png]
\begin{center}
{\large IN THE SAME SERIES.}

\medskip
\rule[.5ex]{2cm}{.2pt}
\end{center}

\smallskip
\begin{small}
\begin{itemize}
\item[]\hspace{-2em}ON CONTINUITY AND IRRATIONAL NUMBERS, and
ON THE NATURE AND MEANING OF NUMBERS\@.
By \textsc{R.~Dedekind}. From the German by \textit{W.~W. Beman}.
Pages, 115. Cloth, 75 cents net (3s.\ 6d.\ net).

\item[]\hspace{-2em}GEOMETRIC EXERCISES IN PAPER-FOLDING\@. By \textsc{T.
Sundara Row}. Edited and revised by \textit{W.~W. Beman} and
\textit{D.~E. Smith}. With many half-tone engravings from photographs
of actual exercises, and a package of papers for
folding. Pages, circa 200. Cloth, \$1.00. net (4s.\ 6d.\ net).
(In Preparation.)

\item[]\hspace{-2em}ON THE STUDY AND DIFFICULTIES OF MATHEMATICS\@.
By \textsc{Augustus De~Morgan}. Reprint edition
with portrait and bibliographies. Pp., 288. Cloth, \$1.25
net (4s.\ 6d.\ net).

\item[]\hspace{-2em}LECTURES ON ELEMENTARY MATHEMATICS\@. By
\textsc{Joseph Louis Lagrange}. From the French by \textit{Thomas J.
McCormack}. With portrait and biography. Pages, 172.
Cloth, \$1.00 net (4s.\ 6d.\ net).

\item[]\hspace{-2em}ELEMENTARY ILLUSTRATIONS OF THE DIFFERENTIAL
AND INTEGRAL CALCULUS\@. By \textsc{Augustus De
Morgan}. Reprint edition. With a bibliography of text-books
of the Calculus. Pp., 144. Price, \$1.00 net (4s.\ 6d.\ net).

\item[]\hspace{-2em}MATHEMATICAL ESSAYS AND RECREATIONS\@. By
\textsc{Prof.\ Hermann Schubert}, of Hamburg, Germany. From
the German by \textit{T.~J. McCormack}, Essays on Number,
The Magic Square, The Fourth Dimension, The Squaring
of the Circle. Pages, 149. Price, Cloth, 75c.\ net (3s.\ net).

\item[]\hspace{-2em}A BRIEF HISTORY OF ELEMENTARY MATHEMATICS\@.
By \textsc{Dr.\ Karl Fink}, of T�bingen. From the German by \textit{W.~W.
Beman} and \textit{D.~E. Smith}, Pp.~333. Cloth, \$1.50 net
(5s.\ 6d.\ net).

\end{itemize}
\end{small}

\begin{center}
\smallskip
\rule[.5ex]{2cm}{.2pt}

\medskip
THE OPEN COURT PUBLISHING COMPANY

{\scriptsize 324 DEARBORN ST., CHICAGO.}

{\small LONDON: Kegan Paul, Trench, Tr�bner \& Co.}
\end{center}
% [File: 002.png]
%title page

\newpage
\begin{center}
{\huge ESSAYS}

\vspace{0.1in}

{\footnotesize ON THE}

\vspace{0.1in}

{\huge THEORY OF NUMBERS}

\vspace{0.5in}

I\@. CONTINUITY AND IRRATIONAL NUMBERS

\smallskip
II\@. THE NATURE AND MEANING OF NUMBERS

\vspace{0.5in}

{\footnotesize BY }

\smallskip
{\large RICHARD DEDEKIND}

\vspace{0.5in}

{\footnotesize AUTHORISED TRANSLATION BY}

\smallskip
WOOSTER WOODRUFF BEMAN

\smallskip
{\footnotesize PROFESSOR OF MATHEMATICS IN THE UNIVERSITY OF MICHIGAN}

\vfill
\rule[.5ex]{3cm}{1pt}
\vfill

{\large CHICAGO}

THE OPEN COURT PUBLISHING COMPANY
\smallskip

{\footnotesize LONDON AGENTS}

{\footnotesize \textsc{Kegan Paul, Trench, Tr�bner \& Co., Ltd.}}

1901
\end{center}

% [File: 003.png]
%copyright page
\newpage

\begin{center}
TRANSLATION COPYRIGHTED
\smallskip

BY
\smallskip

\textsc{The Open Court Publishing Co.}

1901.
\end{center}

% [File: 004.png]
%[**F1: ToC commented out, but proofed text left for comparison.]
%
%\tableofcontents
%
%CONTENTS.
%
%I. CONTINUITY AND IRRATIONAL NUMBERS.
%
%Page
%Preface     1
%
%I. Properties of Rational Numbers     3
%
%II. Comparison of the Rational Numbers with the Points
%of a Straight Line     6
%
%III. Continuity of the Straight Line     8
%
%IV. Creation of Irrational Numbers     12
%
%V. Continuity of the Domain of Real Numbers     19
%
%VI. Operations with Real Numbers     21
%
%VII. Infinitesimal Analysis     24
%
%
%II. THE NATURE AND MEANING OF NUMBERS
%
%Prefaces     31
%
%I. Systems of Elements     44
%
%II. Transformation of a System     50
%
%III. Similarity of a Transformation. Similar Systems     53
%
%IV. Transformation of a System in Itself     56
%
%V. The Finite and Infinite     63
%
%VI. Simply Infinite Systems. Series of Natural Numbers     67
%
%VII. Greater and Less Numbers     70
%
%VIII. Finite and Infinite Parts of the Number-Series     81
%
%IX. Definition of a Transformation of the Number-Series
%by Induction     83
%
%X. The Class of Simply Infinite Systems     92
%
%XI. Addition of Numbers     96
%
%XII. Multiplication of Numbers     101
%
%XIII. Involution of Numbers     104
%
%XIV. Number of the Elements of a Finite System     105
%
\thispagestyle{empty}
% [File: 005.png]

\newpage
\mainmatter
\pagestyle{plain}
%\part*{CONTINUITY AND IRRATIONAL NUMBERS}

% [File: 006.png]
%[Blank Page]
% [File: 007.png]

\mychap{I}{CONTINUITY AND IRRATIONAL NUMBERS}
\addcontentsline{toc}{section}{Preface}

My attention was first directed toward the considerations
which form the subject of this pamphlet
in the autumn of 1858. As professor in the
Polytechnic School in Z�rich I found myself for the
first time obliged to lecture upon the elements of the
differential calculus and felt more keenly than ever
before the lack of a really scientific foundation for
arithmetic. In discussing the notion of the approach
of a variable magnitude to a fixed limiting value, and
especially in proving the theorem that every magnitude
which grows continually, but not beyond all limits,
must certainly approach a limiting value, I had recourse
to geometric evidences. Even now such resort
to geometric intuition in a first presentation of the
differential calculus, I regard as exceedingly useful,
from the didactic standpoint, and indeed indispensable,
if one does not wish to lose too much time. But
that this form of introduction into the differential calculus
can make no claim to being scientific, no one
will deny. For myself this feeling of dissatisfaction
was so overpowering that I made the fixed resolve to
keep meditating on the question till I should find a
% [File: 008.png]
purely arithmetic and perfectly rigorous foundation
for the principles of infinitesimal analysis. The statement
is so frequently made that the differential calculus
deals with continuous magnitude, and yet an
explanation of this continuity is nowhere given; even
the most rigorous expositions of the differential calculus
do not base their proofs upon continuity but,
with more or less consciousness of the fact, they
either appeal to geometric notions or those suggested
by geometry, or depend upon theorems which are
never established in a purely arithmetic manner.
Among these, for example, belongs the above-mentioned
theorem, and a more careful investigation convinced
me that this theorem, or any one equivalent to
it, can be regarded in some way as a sufficient basis
for infinitesimal analysis. It then only remained to
discover its true origin in the elements of arithmetic
and thus at the same time to secure a real definition
of the essence of continuity. I succeeded Nov.~24,~%
1858, and a few days afterward I communicated the
results of my meditations to my dear friend Dur�ge
with whom I had a long and lively discussion. Later
I explained these views of a scientific basis of arithmetic
to a few of my pupils, and here in Braunschweig
read a paper upon the subject before the scientific
club of professors, but I could not make up
my mind to its publication, because, in the first place,
the presentation did not seem altogether simple, and
further, the theory itself had little promise. Nevertheless
% [File: 009.png---\Firefly Diva\\sahobart\glimpseofchaos\hutcheson\-----
I had already half determined to select this
theme as subject for this occasion, when a few days
ago, March~14, by the kindness of the author, the
paper \textit{Die Elemente der Funktionenlehre} by E.~Heine
(\textit{Crelle's Journal}, Vol.~74) came into my hands and
confirmed me in my decision. In the main I fully
agree with the substance of this memoir, and indeed
I could hardly do otherwise, but I will frankly
acknowledge that my own presentation seems to me
to be simpler in form and to bring out the vital point
more clearly. While writing this preface (March~20,~%
1872), I am just in receipt of the interesting paper
\textit{Ueber die Ausdehnung eines Satzes aus der Theorie der
trigonometrischen Reihen}, by G.~Cantor (\textit{Math. Annalen},
Vol.~5), for which I owe the ingenious author my
hearty thanks. As I find on a hasty perusal, the axiom
given in Section II. of that paper, aside from the
form of presentation, agrees with what I designate
in Section III. as the essence of continuity. But what
advantage will be gained by even a purely abstract
definition of real numbers of a higher type, I am as
yet unable to see, conceiving as I do of the domain
of real numbers as complete in itself.

%I.

\mysect{XVII}{I}{Properties of Rational Numbers}

%\hspace{\parindent}
The development of the arithmetic of rational
numbers is here presupposed, but still I think it
worth while to call attention to certain important
% [File: 010.png---\Firefly Diva\\sahobart\glimpseofchaos\hutcheson\-----
matters without discussion, so as to show at the outset
the standpoint assumed in what follows. I regard
the whole of arithmetic as a necessary, or at least natural,
consequence of the simplest arithmetic act, that
of counting, and counting itself as nothing else than
the successive creation of the infinite series of positive
integers in which each individual is defined by the
one immediately preceding; the simplest act is the
passing from an already-formed individual to the consecutive
new one to be formed. The chain of these
numbers forms in itself an exceedingly useful instrument
for the human mind; it presents an inexhaustible
wealth of remarkable laws obtained by the introduction
of the four fundamental operations of arithmetic.
Addition is the combination of any arbitrary repetitions
of the above-mentioned simplest act into a single
act; from it in a similar way arises multiplication.
While the performance of these two operations is always
possible, that of the inverse operations, subtraction
and division, proves to be limited. Whatever the
immediate occasion may have been, whatever comparisons
or analogies with experience, or intuition,
may have led thereto; it is certainly true that just
this limitation in performing the indirect operations
has in each case been the real motive for a new creative
act; thus negative and fractional numbers have
been created by the human mind; and in the system
of all rational numbers there has been gained an instrument
of infinitely greater perfection. This system,
% [File: 011.png---\Firefly Diva\\sahobart\glimpseofchaos\hutcheson\-----
which I shall denote by $R$, possesses first of all a completeness
and self-containedness which I have designated
in another place\footnote{\textit{Vorlesungen �ber Zahlentheorie}, by P.~G.~Lejeune Dirichlet. 2d~ed. �159.} as characteristic of a \textit{body of
numbers} [Zahlk�rper] and which consists in this that
the four fundamental operations are always performable
with any two individuals in $R$, i.~e., the result is
always an individual of $R$, the single case of division
by the number zero being excepted.

For our immediate purpose, however, another
property of the system $R$ is still more important; it
may be expressed by saying that the system $R$ forms
a well-arranged domain of one dimension extending
to infinity on two opposite sides. What is meant by
this is sufficiently indicated by my use of expressions
borrowed from geometric ideas; but just for this reason
it will be necessary to bring out clearly the corresponding
purely arithmetic properties in order to
avoid even the appearance as if arithmetic were in
need of ideas foreign to it.

To express that the symbols $a$ and $b$ represent one
and the same rational number we put $a = b$ as well as
$b = a$. The fact that two rational numbers $a$, $b$ are
different appears in this that the difference $a - b$ has
either a positive or negative value. In the former
case $a$ is said to be \textit{greater} than $b$, $b$ \textit{less} than $a$; this
is also indicated by the symbols $a > b$, $b < a$.\footnote{Hence in what follows the so-called ``algebraic'' greater and less are
understood unless the word ``absolute'' is added.} As in
the latter case $b - a$ has a positive value it follows
% [File: 012.png---\Firefly Diva\\sahobart\glimpseofchaos\hutcheson\-----
that $b > a$, $a < b$. In regard to these two ways in
which two numbers may differ the following laws will
hold:

\textsc{i.} If $a > b$, and $b > c$, then $a > c$. Whenever $a$,
$c$ are two different (or unequal) numbers, and $b$ is
greater than the one and less than the other, we shall,
without hesitation because of the suggestion of geometric
ideas, express this briefly by saying: $b$ lies between
the two numbers $a$, $c$.

\textsc{ii.} If $a$, $c$ are two different numbers, there are infinitely
many different numbers lying between $a$, $c$.

\textsc{iii.} If $a$ is any definite number, then all numbers
of the system $R$ fall into two classes, $A_1$ and $A_2$, each
of which contains infinitely many individuals; the first
class $A_1$ comprises all numbers $a_1$ that are $< a$, the
second class $A_2$ comprises all numbers $a_2$ that are
$> a$; the number $a$ itself may be assigned at pleasure
to the first or second class, being respectively the
greatest number of the first class or the least of the
second. In every case the separation of the system
$R$ into the two classes $A_1$, $A_2$ is such that every number
of the first class $A_1$ is less than every number of
the second class $A_2$.

%II.

\mysect{XVI}{II}{Comparison of the Rational Numbers with the Points of a Straight Line}

The above-mentioned properties of rational numbers
recall the corresponding relations of position of
% [File: 013.png]
the points of a straight line $L$. If the two opposite
directions existing upon it are distinguished by
``right'' and ``left,'' and $p$, $q$ are two different points,
then either $p$ lies to the right of $q$, and at the same
time $q$ to the left of $p$, or conversely $q$ lies to the right
of $p$ and at the same time $p$ to the left of $q$. A third
case is impossible, if $p$, $q$ are actually different points.
In regard to this difference in position the following
laws hold:

\textsc{i.} If $p$ lies to the right of $q$, and $q$ to the right of
$r$, then $p$ lies to the right of $r$; and we say that $q$ lies
between the points $p$ and $r$.

\textsc{ii.} If $p$, $r$ are two different points, then there always
exist infinitely many points that lie between $p$
and $r$.

\textsc{iii.} If $p$ is a definite point in $L$, then all points in
$L$ fall into two classes, $P_1$, $P_2$, each of which contains
infinitely many individuals; the first class $P_1$ contains
all the points $p_1$, that lie to the left of $p$, and the second
class $P_2$ contains all the points $p_2$ that lie to the
right of $p$; the point $p$ itself may be assigned at pleasure
to the first or second class. In every case the
separation of the straight line $L$ into the two classes
or portions $P_1$, $P_2$, is of such a character that every
point of the first class $P_1$ lies to the left of every point
of the second class $P_2$.

This analogy between rational numbers and the
points of a straight line, as is well known, becomes a
real correspondence when we select upon the straight
% [File: 014.png]
line a definite origin or zero-point $o$ and a definite unit
of length for the measurement of segments. With
the aid of the latter to every rational number $a$ a corresponding
length can be constructed and if we lay
this off upon the straight line to the right or left of $o$
according as $a$ is positive or negative, we obtain a
definite end-point $p$, which may be regarded as the
point corresponding to the number $a$; to the rational
number zero corresponds the point $o$. In this way to
every rational number $a$, i.~e., to every individual in
$R$, corresponds one and only one point $p$, i.~e., an individual
in $L$. To the two numbers $a$, $b$ respectively
correspond the two points, $p$, $q$, and if $a>b$, then $p$
lies to the right of $q$. To the laws \textsc{i}, \textsc{ii}, \textsc{iii} of the previous
Section correspond completely the laws \textsc{i}, \textsc{ii}, \textsc{iii}
of the present.

%III.

\mysect{XV}{III}{Continuity of the Straight Line}
\label{EISIII}

Of the greatest importance, however, is the fact
that in the straight line $L$ there are infinitely many
points which correspond to no rational number. If
the point $p$ corresponds to the rational number $a$,
then, as is well known, the length $o\,p$ is commensurable
with the invariable unit of measure used in the
construction, i.~e., there exists a third length, a so-called
common measure, of which these two lengths
are integral multiples. But the ancient Greeks already
% [File: 015.png]
knew and had demonstrated that there are lengths incommensurable
with a given unit of length, e.~g., the
diagonal of the square whose side is the unit of length.
If we lay off such a length from the point $o$ upon the
line we obtain an end-point which corresponds to no
rational number. Since further it can be easily shown
that there are infinitely many lengths which are incommensurable
with the unit of length, we may affirm:
The straight line $L$ is infinitely richer in point-individuals
than the domain $R$ of rational numbers in
number-individuals.

If now, as is our desire, we try to follow up arithmetically
all phenomena in the straight line, the domain
of rational numbers is insufficient and it becomes
absolutely necessary that the instrument $R$ constructed
by the creation of the rational numbers be essentially
improved by the creation of new numbers such that
the domain of numbers shall gain the same completeness,
or as we may say at once, the same \textit{continuity},
as the straight line.

The previous considerations are so familiar and
well known to all that many will regard their repetition
quite superfluous. Still I regarded this recapitulation
as necessary to prepare properly for the main
question. For, the way in which the irrational numbers
are usually introduced is based directly upon the
conception of extensive magnitudes---which itself is
nowhere carefully defined---and explains number as
the result of measuring such a magnitude by another
% [File: 016.png]
of the same kind.\footnote{The apparent advantage of the generality of this definition of number
disappears as soon as we consider complex numbers. According to my view,
on the other hand, the notion of the ratio between two numbers of the same
kind can be clearly developed only after the introduction of irrational numbers.} Instead of this I demand that
arithmetic shall be developed out of itself.

That such comparisons with non-arithmetic notions
have furnished the immediate occasion for the extension
of the number-concept may, in a general way,
be granted (though this was certainly not the case in
the introduction of complex numbers); but this surely
is no sufficient ground for introducing these foreign
notions into arithmetic, the science of numbers. Just
as negative and fractional rational numbers are formed
by a new creation, and as the laws of operating with
these numbers must and can be reduced to the laws
of operating with positive integers, so we must endeavor
completely to define irrational numbers by
means of the rational numbers alone. The question
only remains how to do this.

The above comparison of the domain $R$ of rational
numbers with a straight line has led to the recognition
of the existence of gaps, of a certain incompleteness
or discontinuity of the former, while we ascribe to the
straight line completeness, absence of gaps, or continuity.
In what then does this continuity consist?
Everything must depend on the answer to this question,
and only through it shall we obtain a scientific
basis for the investigation of \textit{all} continuous domains.
By vague remarks upon the unbroken connection in
% [File: 017.png]
the smallest parts obviously nothing is gained; the
problem is to indicate a precise characteristic of continuity
that can serve as the basis for valid deductions.
For a long time I pondered over this in vain, but
finally I found what I was seeking. This discovery
will, perhaps, be differently estimated by different
people; the majority may find its substance very commonplace.
It consists of the following. In the preceding
section attention was called to the fact that
every point $p$ of the straight line produces a separation
of the same into two portions such that every
point of one portion lies to the left of every point of
the other. I find the essence of continuity in the converse,
i.~e., in the following principle:

``If all points of the straight line fall into two
classes such that every point of the first class lies to
the left of every point of the second class, then there
exists one and only one point which produces this division
of all points into two classes, this severing of
the straight line into two portions.''

As already said I think I shall not err in assuming
that every one will at once grant the truth of this
statement; the majority of my readers will be very
much disappointed in learning that by this commonplace
remark the secret of continuity is to be revealed.
To this I may say that I am glad if every one finds
the above principle so obvious and so in harmony
with his own ideas of a line; for I am utterly unable
to adduce any proof of its correctness, nor has any
% [File: 018.png]
one the power. The assumption of this property of
the line is nothing else than an axiom by which we
attribute to the line its continuity, by which we find
continuity in the line. If space has at all a real existence
it is \textit{not} necessary for it to be continuous;
many of its properties would remain the same even
were it discontinuous. And if we knew for certain
that space was discontinuous there would be nothing
to prevent us, in case we so desired, from filling up
its gaps, in thought, and thus making it continuous;
this filling up would consist in a creation of new point-individuals
and would have to be effected in accordance
with the above principle.

%IV.

\mysect{XII}{IV}{Creation of Irrational Numbers}
\label{EISIV}

From the last remarks it is sufficiently obvious
how the discontinuous domain $R$ of rational numbers
may be rendered complete so as to form a continuous
domain. In Section~I it was pointed out that every
rational number $a$ effects a separation of the system $R$
into two classes such that every number $a_1$ of the first
class $A_1$ is less than every number $a_2$ of the second
class $A_2$; the number $a$ is either the greatest number
of the class $A_1$ or the least number of the class $A_2$. If
now any separation of the system $R$ into two classes
$A_1$, $A_2$ is given which possesses only \textit{this} characteristic
property that every number $a_1$ in $A_1$ is less than
every number $a_2$ in $A_2$, then for brevity we shall call
% [File: 019.png]
such a separation a \textit{cut} [Schnitt] and designate it by
$(A_1, A_2)$.  We can then say that every rational number
$a$ produces one cut or, strictly speaking, two cuts,
which, however, we shall not look upon as essentially
different; this cut possesses, \textit{besides}, the property that
either among the numbers of the first class there exists
a greatest or among the numbers of the second
class a least number. And conversely, if a cut possesses
this property, then it is produced by this greatest
or least rational number.

But it is easy to show that there exist infinitely
many cuts not produced by rational numbers. The
following example suggests itself most readily.

Let $D$ be a positive integer but not the square of
an integer, then there exists a positive integer $\lambda$ such
that

\[
\lambda^2 < D <(\lambda + 1)^2.
\]

If we assign to the second class $A_2$, every positive
rational number $a_2$ whose square is $> D$, to the first
class $A_1$ all other rational numbers $a_1$, this separation
forms a cut $(A_1, A_2)$, i.~e., every number $a_1$ is less
than every number $a_2$. For if $a_1 = 0$, or is negative,
then on that ground $a_1$ is less than any number $a_2$,
because, by definition, this last is positive; if $a_1$ is
positive, then is its square $\leqq D$, and hence $a_1$ is less
than any positive number $a_2$ whose square is $> D$.

But this cut is produced by no rational number.
To demonstrate this it must be shown first of all that
there exists no rational number whose square $= D$.
% [File: 020.png]
Although this is known from the first elements of the
theory of numbers, still the following indirect proof
may find place here. If there exist a rational number
whose square $= D$, then there exist two positive integers
$t$, $u$, that satisfy the equation
\[
t^2 - Du^2 = 0,
\]
and we may assume that $u$ is the \textit{least} positive integer
possessing the property that its square, by multiplication
by $D$, may be converted into the square of an
integer $t$. Since evidently
\[
\lambda u < t < (\lambda + 1) u,
\]
the number $u' = t - \lambda u$ is a positive integer certainly
\textit{less} than $u$. If further we put
\[
t' = Du - \lambda t,
\]
$t'$ is likewise a positive integer, and we have
\[
t'^2 - Du'^2 = (\lambda^2 - D)(t^2 - Du^2) = 0,
\]
which is contrary to the assumption respecting $u$.

Hence the square of every rational number $x$ is
either $< D$ or $> D$. From this it easily follows that
there is neither in the class $A_1$ a greatest, nor in the
class $A_2$ a least number. For if we put
\begin{align*}
y &= \frac{x(x^2 + 3D)}{3x^2 + D},
\shortintertext {we have}
y - x &= \frac{2x(D - x^2)}{3x^2 + D}
\shortintertext {and}
y^2 - D &= \frac{(x^2 - D)^3}{(3x^2 + D)^2}.
\end{align*}

% [File: 021.png]

If in this we assume $x$ to be a positive number
from the class $A_1$, then $x^2<D$, and hence $y>x$ and
$y^2<D$. Therefore $y$ likewise belongs to the class $A_1$.
But if we assume $x$ to be a number from the class $A_2$,
then $x^2>D$, and hence $y<x$, $y>0$, and $y^2>D$.
Therefore $y$ likewise belongs to the class $A_2$. This
cut is therefore produced by no rational number.

In this property that not all cuts are produced by
rational numbers consists the incompleteness or discontinuity
of the domain $R$ of all rational numbers.

Whenever, then, we have to do with a cut $(A_1, A_2)$
produced by no rational number, we create a new, an
\textit{irrational} number $\alpha$, which we regard as completely
defined by this cut $(A_1, A_2)$; we shall say that the
number $\alpha$ corresponds to this cut, or that it produces
this cut. From now on, therefore, to every definite
cut there corresponds a definite rational or irrational
number, and we regard two numbers as \textit{different} or
\textit{unequal} always and only when they correspond to essentially
different cuts.

In order to obtain a basis for the orderly arrangement
of all \textit{real}, i.~e., of all rational and irrational
numbers we must investigate the relation between
any two cuts $(A_1, A_2)$ and $(B_1, B_2)$ produced by any
two numbers $\alpha$ and $\beta$. Obviously a cut $(A_1, A_2)$ is
given completely when one of the two classes, e.~g.,
the first $A_1$ is known, because the second $A_2$ consists
of all rational numbers not contained in $A_1$, and the
characteristic property of such a first class lies in this
% [File: 022.png]
that if the number $a_1$ is contained in it, it also contains
all numbers less than $a_1$. If now we compare
two such first classes $A_1$, $B_1$ with each other, it may
happen

1. That they are perfectly identical, i.~e., that every
number contained in $A_1$ is also contained in $B_1$, and
that every number contained in $B_1$ is also contained
in $A_1$. In this case $A_2$ is necessarily identical with
$B_2$, and the two cuts are perfectly identical, which we
denote in symbols by $\alpha=\beta$ or $\beta=\alpha$.

But if the two classes $A_1$, $B_1$ are not identical,
then there exists in the one, e.~g., in $A_1$, a number
$a'_1=b'_2$ not contained in the other $B_1$ and consequently
found in $B_2$; hence all numbers $b_1$ contained
in $B_1$ are certainly less than this number $a'_1=b'_2$ and
therefore all numbers $b_1$ are contained in $A_1$.

2. If now this number $a'_1$ is the only one in $A_1$ that
is not contained in $B_1$, then is every other number $a_1$
contained in $A_1$ also contained in $B_1$ and is consequently
$<a'_1$, i.~e., $a'_1$ is the greatest among all the
numbers $a_1$, hence the cut $(A_1, A_2)$ is produced by
the rational number $a=a'_1=b'_2$. Concerning the
other cut $(B_1, B_2)$ we know already that all numbers
$b_1$ in $B_1$ are also contained in $A_1$ and are less than
the number $a'_1=b'_2$ which is contained in $B_2$; every
other number $b_2$ contained in $B_2$ must, however, be
greater than $b'_2$, for otherwise it would be less than
$a'_1$, therefore contained in $A_1$ and hence in $B_1$; hence
$b'_2$ is the least among all numbers contained in $B_2$,
% [File: 023.png]
and consequently the cut $(B_1, B_2)$ is produced by the
same rational number $\beta = b'_2 = a'_1 = \alpha$. The two cuts
are then only unessentially different.

3. If, however, there exist in $A_1$ at least two different
numbers $a'_1 = b'_2$ and $a''_1 = b''_2$, which are not contained
in $B_1$, then there exist infinitely many of them,
because all the infinitely many numbers lying between
$a'_1$ and $a''_1$ are obviously contained in $A_1$ (Section I,
\textsc{ii}) but not in $B_1$. In this case we say that the numbers
$\alpha$ and $\beta$ corresponding to these two essentially
different cuts $(A_1, A_2)$ and $(B_1, B_2)$ are \textit{different}, and
further that $\alpha$ is \textit{greater} than $\beta$, that $\beta$ is \textit{less} than $\alpha$,
which we express in symbols by $\alpha > \beta$ as well as $\beta < \alpha$.
It is to be noticed that this definition coincides completely
with the one given earlier, when $\alpha$, $\beta$ are rational.

The remaining possible cases are these:

4. If there exists in $B_1$ one and only one number
$b'_1 = a'_2$, that is not contained in $A_1$ then the two cuts
$(A_1, A_2)$ and $(B_1, B_2)$ are only unessentially different
and they are produced by one and the same rational
number $\alpha = a'_2 = b'_1 = \beta$.

5. But if there are in $B_1$ at least two numbers
which are not contained in $A_1$, then $\beta > \alpha$, $\alpha < \beta$.

As this exhausts the possible cases, it follows that
of two different numbers one is necessarily the greater,
the other the less, which gives two possibilities. A
third case is impossible. This was indeed involved
in the use of the \textit{comparative} (greater, less) to designate
% [File: 024.png]
the relation between $\alpha$, $\beta$; but this use has only
now been justified. In just such investigations one
needs to exercise the greatest care so that even with
the best intention to be honest he shall not, through
a hasty choice of expressions borrowed from other notions
already developed, allow himself to be led into
the use of inadmissible transfers from one domain to
the other.

If now we consider again somewhat carefully the
case $\alpha>\beta$ it is obvious that the less number $\beta$, if
rational, certainly belongs to the class $A_1$; for since
there is in $A_1$ a number $a'_1=b'_2$ which belongs to the
class $B_2$, it follows that the number $\beta$, whether the
greatest number in $B_1$ or the least in $B_2$ is certainly
$\leqq a'_1$ and hence contained in $A_1$. Likewise it is obvious
from $\alpha>\beta$ that the greater number $\alpha$, if rational,
certainly belongs to the class $B_2$, because $\alpha \geqq a'_1$. Combining
these two considerations we get the following
result: If a cut is produced by the number $\alpha$ then any
rational number belongs to the class $A_1$ or to the class
$A_2$ according as it is less or greater than $\alpha$; if the
number $\alpha$ is itself rational it may belong to either
class.

From this we obtain finally the following: If $\alpha>\beta$,
i.~e., if there are infinitely many numbers in $A_1$ not
contained in $B_1$ then there are infinitely many such
numbers that at the same time are different from $\alpha$ and
from $\beta$; every such rational number $c$ is $<\alpha$, because
% [File: 025.png]
it is contained in $A_1$ and at the same time it is $>\beta$
because contained in $B_2$.

%V.

\mysect{XIII}{V}{Continuity of the Domain of Real Numbers}

In consequence of the distinctions just established
the system $\mathfrak{R}$ of all real numbers forms a well-arranged
domain of one dimension; this is to mean merely that
the following laws prevail:

\textsc{i.} If $\alpha>\beta$, and $\beta>\gamma$, then is also $\alpha>\gamma$. We
shall say that the number $\beta$ lies between $\alpha$ and $\gamma$.

\textsc{ii.} If $\alpha$, $\gamma$ are any two different numbers, then
there exist infinitely many different numbers $\beta$ lying
between $\alpha$, $\gamma$.

\textsc{iii.} If $\alpha$ is any definite number then all numbers
of the system $\mathfrak{R}$ fall into two classes $\mathfrak{A}_1$ and $\mathfrak{A}_2$ each
of which contains infinitely many individuals; the
first class $\mathfrak{A}_1$ comprises all the numbers $\alpha_1$ that are
less than $\alpha$, the second $\mathfrak{A}_2$ comprises all the numbers
$\alpha_2$ that are greater than $\alpha$; the number $\alpha$ itself may be
assigned at pleasure to the first class or to the second,
and it is respectively the greatest of the first or the
least of the second class. In each case the separation
of the system $\mathfrak{R}$ into the two classes $\mathfrak{A}_1$, $\mathfrak{A}_2$ is such
that every number of first class $\mathfrak{A}_1$ is smaller than
every number of the second class $\mathfrak{A}_2$ and we say that
this separation is produced by the number $\alpha$.

For brevity and in order not to weary the reader I
suppress the proofs of these theorems which follow
% [File: 026.png]
immediately from the definitions of the previous section.

Beside these properties, however, the domain $\mathfrak{R}$
possesses also \textit{continuity}; i.~e., the following theorem
is true:

\textsc{iv.} \label{EISViv}If the system $\mathfrak{R}$ of all real numbers breaks up
into two classes $\mathfrak{A}_1$, $\mathfrak{A}_2$ such that every number $\alpha_1$ of
the class $\mathfrak{A}_1$ is less than every number $\alpha_2$ of the class
$\mathfrak{A}_2$ then there exists one and only one number $\alpha$ by
which this separation is produced.

\textit{Proof.} By the separation or the cut of $\mathfrak{R}$ into $\mathfrak{A}_1$
and $\mathfrak{A}_2$ we obtain at the same time a cut $(A_1, A_2)$
of the system $R$ of all rational numbers which is defined
by this that $A_1$ contains all rational numbers of
the class $\mathfrak{A}_1$ and $A_2$ all other rational numbers, i.~e.,
all rational numbers of the class $\mathfrak{A}_2$. Let $\alpha$ be the
perfectly definite number which produces this cut
$(A_1, A_2)$. If $\beta$ is any number different from $\alpha$, there
are always infinitely many rational numbers $c$ lying
between $\alpha$ and $\beta$. If $\beta<\alpha$, then $c<\alpha$; hence $c$ belongs
to the class $A_1$ and consequently also to the
class $\mathfrak{A}_1$, and since at the same time $\beta<c$ then $\beta$ also
belongs to the same class $\mathfrak{A}_1$, because every number
in $\mathfrak{A}_2$ is greater than every number $c$ in $\mathfrak{A}_1$. But if
$\beta>\alpha$, then is $c>\alpha$; hence $c$ belongs to the class $A_2$
and consequently also to the class $\mathfrak{A}_2$, and since at
the same time $\beta>c$, then $\beta$ also belongs to the same
class $\mathfrak{A}_2$, because every number in $\mathfrak{A}_1$ is less than
every number $c$ in $\mathfrak{A}_2$. Hence every number $\beta$ different
% [File: 027.png]
from $\alpha$ belongs to the class $\mathfrak{A}_1$ or to the class $\mathfrak{A}_2$
according as $\beta<\alpha$ or $\beta>\alpha$; consequently $\alpha$ itself is
either the greatest number in $\mathfrak{A}_1$ or the least number
in $\mathfrak{A}_2$, i.~e., $\alpha$ is one and obviously the only number
by which the separation of $R$ into the classes $\mathfrak{A}_1$, $\mathfrak{A}_2$
is produced. Which was to be proved.

%VI.

\mysect{XII}{VI}{Operations with Real Numbers}
\label{EISVI}

To reduce any operation with two real numbers
$\alpha$, $\beta$ to operations with rational numbers, it is only
necessary from the cuts $(A_1, A_2)$, $(B_1, B_2)$ produced
by the numbers $\alpha$ and $\beta$ in the system $R$ to define the
cut $(C_1, C_2)$ which is to correspond to the result of
the operation, $\gamma$. I confine myself here to the discussion
of the simplest case, that of addition.

If $c$ is any rational number, we put it into the class
$C_1$, provided there are two numbers one $a_1$ in $A_1$ and
one $b_1$ in $B_1$ such that their sum $a_1+b_1 \geqq c$; all other
rational numbers shall be put into the class $C_2$. This
separation of all rational numbers into the two classes
$C_1$, $C_2$ evidently forms a cut, since every number $c_1$ in
$C_1$ is less than every number $c_2$ in $C_2$. If both $\alpha$ and
$\beta$ are rational, then every number $c_1$ contained in $C_1$ is
$\leqq\alpha+\beta$, because $a_1\leqq\alpha$, $b_1\leqq\beta$, and therefore $a_1+b_1
\leqq\alpha+\beta$; further, if there were contained in $C_2$ a number
$c_2<\alpha+\beta$, hence $\alpha+\beta=c_2+p$, where $p$ is a positive
rational number, then we should have
\[
c_2=(\alpha- \frac{1}{2} p)+(\beta- \frac{1}{2} p),
\]
% [File: 028.png]
which contradicts the definition of the number $c_2$, because
$\alpha- \frac{1}{2} p$ is a number in $A_1$, and $\beta- \frac{1}{2} p$ a number
in $B_1$; consequently every number $c_2$ contained in $C_2$
is $\geqq\alpha+\beta$. Therefore in this case the cut $(C_1, C_2)$ is
produced by the sum $\alpha+\beta$. Thus we shall not violate
the definition which holds in the arithmetic of rational
numbers if in all cases we understand by the sum
$\alpha+\beta$ of any two real numbers $\alpha$, $\beta$ that number $\gamma$ by
which the cut $(C_1, C_2)$ is produced. Further, if only
one of the two numbers $\alpha$, $\beta$ is rational, e.~g., $\alpha$, it is
easy to see that it makes no difference with the sum
$\gamma=\alpha+\beta$ whether the number $\alpha$ is put into the class
$A_1$ or into the class $A_2$.

Just as addition is defined, so can the other operations
of the so-called elementary arithmetic be defined,
viz., the formation of differences, products,
quotients, powers, roots, logarithms, and in this way
we arrive at real proofs of theorems (as, e.~g., $\sqrt{2}\cdot\sqrt{3}
=\sqrt{6}$), which to the best of my knowledge have never
been established before. The excessive length that is
to be feared in the definitions of the more complicated
operations is partly inherent in the nature of the subject
but can for the most part be avoided. Very useful in
this connection is the notion of an \textit{interval}, i.~e., a
system $A$ of rational numbers possessing the following
characteristic property: if $a$ and $a'$ are numbers
of the system $A$, then are all rational numbers lying
between $a$ and $a'$ contained in $A$. The system $R$ of
all rational numbers, and also the two classes of any
% [File: 029.png]
cut are intervals. If there exist a rational number $a_1$
which is less and a rational number $a_2$ which is greater
than every number of the interval $A$, then $A$ is called
a finite interval; there then exist infinitely many numbers
in the same condition as $a_1$ and infinitely many in
the same condition as $a_2$; the whole domain $R$ breaks
up into three parts $A_1$, $A$, $A_2$ and there enter two perfectly
definite rational or irrational numbers $\alpha_1$, $\alpha_2$
which may be called respectively the lower and upper
(or the less and greater) \textit{limits} of the interval; the
lower limit $\alpha_1$ is determined by the cut for which the
system $A_1$ forms the first class and the upper $\alpha_2$ by the
cut for which the system $A_2$ forms the second class.
Of every rational or irrational number $\alpha$ lying between
$\alpha_1$ and $\alpha_2$ it may be said that it lies \textit{within} the interval
$A$. If all numbers of an interval $A$ are also numbers
of an interval $B$, then $A$ is called a portion of $B$.

Still lengthier considerations seem to loom up
when we attempt to adapt the numerous theorems of
the arithmetic of rational numbers (as, e.~g., the theorem
$(a+b)c=ac+bc$) to any real numbers. This,
however, is not the case. It is easy to see that it
all reduces to showing that the arithmetic operations
possess a certain continuity. What I mean by this
statement may be expressed in the form of a general
theorem:

``If the number $\lambda$ is the result of an operation performed
on the numbers $\alpha$, $\beta$, $\gamma$, $\ldots$ and $\lambda$ lies within
the interval $L$, then intervals $A$, $B$, $C$, $\ldots$ can be
% [File: 030.png]
taken within which lie the numbers $\alpha$, $\beta$, $\gamma$, $\ldots$ such
that the result of the same operation in which the
numbers $\alpha$, $\beta$, $\gamma$, $\ldots$ are replaced by arbitrary numbers
of the intervals $A$, $B$, $C$, $\ldots$ is always a number
lying within the interval $L$.'' The forbidding clumsiness,
however, which marks the statement of such a
theorem convinces us that something must be brought
in as an aid to expression; this is, in fact, attained in
the most satisfactory way by introducing the ideas of
\textit{variable magnitudes}, \textit{functions}, \textit{limiting values}, and it
would be best to base the definitions of even the simplest
arithmetic operations upon these ideas, a matter
which, however, cannot be carried further here.

%Observed to be close to bottom of page so force skip
\newpage
%VII.

\mysect{XI}{VII}{Infinitesimal Analysis}

Here at the close we ought to explain the connection
between the preceding investigations and certain
fundamental theorems of infinitesimal analysis.

We say that a variable magnitude $x$ which passes
through successive definite numerical values approaches
a fixed limiting value $\alpha$ when in the course
of the process $x$ lies finally between two numbers between
which $\alpha$ itself lies, or, what amounts to the
same, when the difference $x-\alpha$ taken absolutely becomes
finally less than any given value different from
zero.

One of the most important theorems may be stated
in the following manner: ``If a magnitude $x$ grows
% [File: 031.png]
continually but not beyond all limits it approaches a
limiting value.''

I prove it in the following way. By hypothesis
there exists one and hence there exist infinitely many
numbers $\alpha_2$ such that $x$ remains continually $<\alpha_2$; I
designate by $\mathfrak{A}_2$ the system of all these numbers $\alpha_2$,
by $\mathfrak{A}_1$ the system of all other numbers $\alpha_1$; each of the
latter possesses the property that in the course of the
process $x$ becomes finally $\geqq \alpha_1$, hence every number $\alpha_1$
is less than every number $\alpha_2$ and consequently there
exists a number $\alpha$ which is either the greatest in $\mathfrak{A}_1$
or the least in $\mathfrak{A}_2$ (V, \textsc{iv}). The former cannot be the
case since $x$ never ceases to grow, hence $\alpha$ is the least
number in $\mathfrak{A}_2$. Whatever number $\alpha_1$ be taken we shall
have finally $\alpha_1<x<\alpha$, i.~e., $x$ approaches the limiting
value $\alpha$.

This theorem is equivalent to the principle of continuity,
i.~e., it loses its validity as soon as we assume
a single real number not to be contained in the domain
$\mathfrak{R}$; or otherwise expressed: if this theorem is
correct, then is also theorem \textsc{iv}. in V. correct.

Another theorem of infinitesimal analysis, likewise
equivalent to this, which is still oftener employed,
may be stated as follows: ``If in the variation of a
magnitude $x$ we can for every given positive magnitude
$\delta$ assign a corresponding position from and after
which $x$ changes by less than $\delta$ then $x$ approaches a
limiting value.''

This converse of the easily demonstrated theorem
% [File: 032.png]
that every variable magnitude which approaches a
limiting value finally changes by less than any given
positive magnitude can be derived as well from the
preceding theorem as directly from the principle of
continuity.  I take the latter course.  Let $\delta$ be any
positive magnitude (i.~e., $\delta>0$), then by hypothesis
a time will come after which $x$ will change by less
than $\delta$, i.~e., if at this time $x$ has the value $a$, then
afterwards we shall continually have $x>a-\delta$ and
$x<a+\delta$.  I now for a moment lay aside the original
hypothesis and make use only of the theorem just
demonstrated that all later values of the variable $x$ lie
between two assignable finite values. Upon this I base
a double separation of all real numbers.  To the system
$\mathfrak{A}_2$ I assign a number $\alpha_2$ (e.g., $a+\delta$) when in the
course of the process $x$ becomes finally $\leqq \alpha_2$; to the
system $\mathfrak{A}_1$ I assign every number not contained in $\mathfrak{A}_2$;
if $\alpha_1$ is such a number, then, however far the process
may have advanced, it will still happen infinitely many
times that $x>\alpha_2$.  Since every number $\alpha_1$ is less than
every number $\alpha_2$ there exists a perfectly definite number
$\alpha$ which produces this cut $(\mathfrak{A}_1, \mathfrak{A}_2)$ of the system
$\mathfrak{R}$ and which I will call the upper limit of the variable
$x$ which always remains finite.  Likewise as a result
of the behavior of the variable $x$ a second cut $(\mathfrak{B}_1,
\mathfrak{B}_2)$ of the system $\mathfrak{R}$ is produced; a number $\beta_2$ (e.~g.,
$a-\delta$) is assigned to $\mathfrak{B}_2$ when in the course of the process
$x$ becomes finally $\geqq\beta$; every other number $\beta_2$,
to be assigned to $\mathfrak{B}_2$, has the property that $x$ is never
% [File: 033.png]
finally $\geqq \beta_2$; therefore infinitely many times $x$ becomes
$<\beta_2$; the number $\beta$ by which this cut is produced I
call the lower limiting value of the variable $x$. The
two numbers $\alpha$, $\beta$ are obviously characterised by the
following property: if $\epsilon$ is an arbitrarily small positive
magnitude then we have always finally $x < \alpha + \epsilon$ and
$x > \beta - \epsilon$, but never finally $x < \alpha - \epsilon$ and never finally
$x > \beta + \epsilon$. Now two cases are possible. If $\alpha$ and $\beta$
are different from each other, then necessarily $\alpha > \beta$,
since continually $\alpha_2 \geqq \beta_2$; the variable $x$ oscillates,
and, however far the process advances, always undergoes
changes whose amount surpasses the value
$(\alpha - \beta) -2\epsilon$ where $\epsilon$ is an arbitrarily small positive
magnitude. The original hypothesis to which I now
return contradicts this consequence; there remains
only the second case $\alpha = \beta$ since it has already
been shown that, however small be the positive magnitude
$\epsilon$, we always have finally $x < \alpha + \epsilon$ and $x > \beta - \epsilon$,
$x$ approaches the limiting value $\alpha$, which was to be
proved.

These examples may suffice to bring out the connection
between the principle of continuity and infinitesimal
analysis.

% [File: 034.png]
%\part*{THE NATURE AND MEANING OF NUMBERS}

% [File: 035.png]
%[Blank Page]
% [File: 036.png]
\newpage
\mychap{II}{THE NATURE AND MEANING OF NUMBERS}
\addcontentsline{toc}{section}{Prefaces}
\begin{center}
{\large PREFACE TO THE FIRST EDITION.}
\end{center}

In science nothing capable of proof ought to be accepted
without proof. Though this demand seems
so reasonable yet I cannot regard it as having been
met even in the most recent methods of laying the
foundations of the simplest science; viz., that part of
logic which deals with the theory of numbers.\footnote{Of the works which have come under my observation I mention the valuable
\textit{Lehrbuch der Arithmetik und Algebra} of E.~Schr�der (Leipzig, 1873),
which contains a bibliography of the subject, and in addition the memoirs of
Kronecker and von Helmholtz upon the Number-Concept and upon Counting
and Measuring (in the collection of philosophical essays published in honor
of E.~Zeller, Leipzig, 1887). The appearance of these memoirs has induced
me to publish my own views, in many respects similar but in foundation
essentially different, which I formulated many years ago in absolute independence
of the works of others.} In
speaking of arithmetic (algebra, analysis) as a part
of logic I mean to imply that I consider the number-concept
entirely independent of the notions or intuitions
of space and time, that I consider it an immediate
result from the laws of thought. My answer to
the problems propounded in the title of this paper is,
then, briefly this: numbers are free creations of the
human mind; they serve as a means of apprehending
more easily and more sharply the difference of things.
It is only through the purely logical process of building
up the science of numbers and by thus acquiring
% [File: 037.png]
the continuous number-domain that we are prepared
accurately to investigate our notions of space and
time by bringing them into relation with this number-domain
created in our mind.\footnote{See Section III. of my memoir, \textit{Continuity and Irrational Numbers}
(Braunschweig, 1872), translated at pages \pageref{EISIII} et seq.\ of the present volume.} If we scrutinise closely
what is done in counting an aggregate or number
of things, we are led to consider the ability of the
mind to relate things to things, to let a thing correspond
to a thing, or to represent a thing by a thing,
an ability without which no thinking is possible.
Upon this unique and therefore absolutely indispensable
foundation, as I have already affirmed in an announcement
of this paper,\footnote{Dirichlet's \textit{Vorlesungen �ber Zahlentheorie}, third edition, 1879, � 163, note
on page 470.} must, in my judgment,
the whole science of numbers be established. The
design of such a presentation I had formed before the
publication of my paper on \textit{Continuity}, but only after
its appearance and with many interruptions occasioned
by increased official duties and other necessary
labors, was I able in the years 1872 to 1878 to commit
to paper a first rough draft which several mathematicians
examined and partially discussed with me. It
bears the same title and contains, though not arranged
in the best order, all the essential fundamental ideas
of my present paper, in which they are more carefully
elaborated. As such main points I mention here the
sharp distinction between finite and infinite (64), the
notion of the number [\textit{Anzahl}] of things (161), the
% [File: 038.png]
proof that the form of argument known as complete
induction (or the inference from $n$ to $n + 1$) is really
conclusive (59), (60), (80), and that therefore the
definition by induction (or recursion) is determinate
and consistent (126).

This memoir can be understood by any one possessing
what is usually called good common sense;
no technical philosophic, or mathematical, knowledge
is in the least degree required. But I feel conscious
that many a reader will scarcely recognise in the
shadowy forms which I bring before him his numbers
which all his life long have accompanied him as faithful
and familiar friends; he will be frightened by the
long series of simple inferences corresponding to our
step-by-step understanding, by the matter-of-fact dissection
of the chains of reasoning on which the laws
of numbers depend, and will become impatient at
being compelled to follow out proofs for truths which
to his supposed inner consciousness seem at once evident
and certain. On the contrary in just this possibility
of reducing such truths to others more simple,
no matter how long and apparently artificial the series
of inferences, I recognise a convincing proof that their
possession or belief in them is never given by inner
consciousness but is always gained only by a more or
less complete repetition of the individual inferences.
I like to compare this action of thought, so difficult
to trace on account of the rapidity of its performance,
with the action which an accomplished reader performs
% [File: 039.png]
in reading; this reading always remains a more
or less complete repetition of the individual steps
which the beginner has to take in his wearisome
spelling-out; a very small part of the same, and therefore
a very small effort or exertion of the mind, is sufficient
for the practised reader to recognise the correct,
true word, only with very great probability, to be
sure; for, as is well known, it occasionally happens
that even the most practised proof-reader allows a
typographical error to escape him, i.~e., reads falsely,
a thing which would be impossible if the chain of
thoughts associated with spelling were fully repeated.
So from the time of birth, continually and in increasing
measure we are led to relate things to things and
thus to use that faculty of the mind on which the
creation of numbers depends; by this practice continually
occurring, though without definite purpose,
in our earliest years and by the attending formation
of judgments and chains of reasoning we acquire a
store of real arithmetic truths to which our first teachers
later refer as to something simple, self-evident,
given in the inner consciousness; and so it happens
that many very complicated notions (as for example
that of the number [\textit{Anzahl}] of things) are erroneously
regarded as simple. In this sense which I wish
to express by the word formed after a well-known
saying \textit{\textgreek{\as e\ig{} \oR{} \asa njrwpos \as rijmht\ia zai}},
%$\alpha\epsilon\iota\;\omicron\;\alpha\nu\theta\rho\omega\pi\omicron\varsigma\;\alpha\rho\iota\theta\mu\eta\tau\iota\zeta\epsilon\iota$,
I hope that the following
pages, as an attempt to establish the science of
numbers upon a uniform foundation will find a generous
% [File: 040.png]
welcome and that other mathematicians will be
led to reduce the long series of inferences to more
moderate and attractive proportions.

In accordance with the purpose of this memoir I
restrict myself to the consideration of the series of
so-called natural numbers. In what way the gradual
extension of the number-concept, the creation of
zero, negative, fractional, irrational and complex
numbers are to be accomplished by reduction to the
earlier notions and that without any introduction of
foreign conceptions (such as that of measurable magnitudes,
which according to my view can attain perfect
clearness only through the science of numbers),
this I have shown at least for irrational numbers
in my former memoir on \textit{Continuity} (1872); in a way
wholly similar, as I have already shown in Section~III.
of that memoir,\footnote{Pages \pageref{EISIII} et seq.\ of the present volume.}
may the other extensions be treated,
and I propose sometime to present this whole subject
in systematic form. From just this point of view it
appears as something self-evident and not new that
every theorem of algebra and higher analysis, no matter
how remote, can be expressed as a theorem about
natural numbers,---a declaration I have heard repeatedly
from the lips of Dirichlet. But I see nothing
meritorious--and this was just as far from Dirichlet's
thought---in actually performing this wearisome circumlocution
and insisting on the use and recognition
of no other than rational numbers. On the contrary,
% [File: 041.png]
the greatest and most fruitful advances in mathematics
and other sciences have invariably been made by the
creation and introduction of new concepts, rendered
necessary by the frequent recurrence of complex phenomena
which could be controlled by the old notions
only with difficulty. On this subject I gave a lecture
before the philosophic faculty in the summer of 1854
on the occasion of my admission as privat-docent in
G�ttingen. The scope of this lecture met with the
approval of Gauss; but this is not the place to go
into further detail.

Instead of this I will use the opportunity to make
some remarks relating to my earlier work, mentioned
above, on \textit{Continuity and Irrational Numbers}. The
theory of irrational numbers there presented, wrought
out in the fall of 1853, is based on the phenomenon
(Section IV.)\footnote{Pages \pageref{EISIV} et seq.\ of the present volume.} occurring in the domain of rational
numbers which I designate by the term cut [\textit{Schnitt}]
and which I was the first to investigate carefully; it
culminates in the proof of the continuity of the new
domain of real numbers (Section V., \textsc{iv}.).\footnote{Page \pageref{EISViv} of the present volume.} It appears
to me to be somewhat simpler, I might say easier,
than the two theories, different from it and from each
other, which have been proposed by Weierstrass and
G.~Cantor, and which likewise are perfectly rigorous.
It has since been adopted without essential modification
by U.~Dini in his \textit{Fondamenti per la teorica delle
% [File: 042.png]
funzioni di variabili reali} (Pisa, 1878); but the fact that
in the course of this exposition my name happens to
be mentioned, not in the description of the purely
arithmetic phenomenon of the cut but when the author
discusses the existence of a measurable quantity
corresponding to the cut, might easily lead to the supposition
that my theory rests upon the consideration
of such quantities. Nothing could be further from
the truth; rather have I in Section III.\footnote{Pages \pageref{EISIII} et seq.\ of the present volume.} of my paper
advanced several reasons why I wholly reject the introduction
of measurable quantities; indeed, at the
end of the paper I have pointed out with respect to
their existence that for a great part of the science of
space the continuity of its configurations is not even
a necessary condition, quite aside from the fact that
in works on geometry arithmetic is only casually mentioned
by name but is never clearly defined and therefore
cannot be employed in demonstrations. To explain
this matter more clearly I note the following
example: If we select three non-collinear points $A$,
$B$, $C$ at pleasure, with the single limitation that the
ratios of the distances $AB$, $AC$, $BC$ are algebraic
numbers,\footnote{Dirichlet's \textit{Vorlesungen �ber Zahlentheorie}, �~159 of the second edition,
�~160 of the third.} and regard as existing in space only those
points $M$, for which the ratios of $AM$, $BM$, $CM$ to $AB$
are likewise algebraic numbers, then is the space made
up of the points $M$, as is easy to see, everywhere discontinuous;
% [File: 043.png]
but in spite of this discontinuity, and despite
the existence of gaps in this space, all constructions
that occur in Euclid's \textit{Elements}, can, so far as I
can see, be just as accurately effected as in perfectly
continuous space; the discontinuity of this space
would not be noticed in Euclid's science, would not
be felt at all. If any one should say that we cannot
conceive of space as anything else than continuous, I
should venture to doubt it and to call attention to the
fact that a far advanced, refined scientific training is
demanded in order to perceive clearly the essence of
continuity and to comprehend that besides rational
quantitative relations, also irrational, and besides algebraic,
also transcendental quantitative relations are
conceivable. All the more beautiful it appears to me
that without any notion of measurable quantities and
simply by a finite system of simple thought-steps man
can advance to the creation of the pure continuous
number-domain; and only by this means in my view
is it possible for him to render the notion of continuous
space clear and definite.

The same theory of irrational numbers founded
upon the phenomenon of the cut is set forth in the
\textit{Introduction � la th�orie des fonctions d'une variable} by
J.~Tannery (Paris, 1886). If I rightly understand a
passage in the preface to this work, the author has
thought out his theory independently, that is, at a
time when not only my paper, but Dini's \textit{Fondamenti}
mentioned in the same preface, was unknown to him.
% [File: 044.png]
This agreement seems to me a gratifying proof that
my conception conforms to the nature of the case, a
fact recognised by other mathematicians, e.~g., by
Pasch in his \textit{Einleitung in die Differential- und Integralrechnung}
(Leipzig, 1883). But I cannot quite agree
with Tannery when he calls this theory the development
of an idea due to J.~Bertrand and contained in
his \textit{Trait� d'arithm�tique}, consisting in this that an irrational
number is defined by the specification of all
rational numbers that are less and all those that are
greater than the number to be defined. As regards
this statement which is repeated by Stolz---apparently
without careful investigation---in the preface to the
second part of his \textit{Vorlesungen �ber allgemeine Arithmetik}
(Leipzig, 1886), I venture to remark the following:
That an irrational number is to be considered
as fully defined by the specification just described,
this conviction certainly long before the time of Bertrand
was the common property of all mathematicians
who concerned themselves with the notion of the
irrational. Just this manner of determining it is in
the mind of every computer who calculates the irrational
root of an equation by approximation, and if,
as Bertrand does exclusively in his book, (the eighth
edition, of the year 1885, lies before me,) one regards
the irrational number as the ratio of two measurable
quantities, then is this manner of determining it
already set forth in the clearest possible way in the
celebrated definition which Euclid gives of the equality
% [File: 045.png]
of two ratios (\textit{Elements}, V., 5). This same most
ancient conviction has been the source of my theory
as well as that of Bertrand and many other more or
less complete attempts to lay the foundations for the
introduction of irrational numbers into arithmetic.
But though one is so far in perfect agreement with
Tannery, yet in an actual examination he cannot fail
to observe that Bertrand's presentation, in which the
phenomenon of the cut in its logical purity is not
even mentioned, has no similarity whatever to mine,
inasmuch as it resorts at once to the existence of a
measurable quantity, a notion which for reasons mentioned
above I wholly reject. Aside from this fact
this method of presentation seems also in the succeeding
definitions and proofs, which are based on the
postulate of this existence, to present gaps so essential
that I still regard the statement made in my paper
(Section VI.),\footnote
  {Pages \pageref{EISVI} et seq.\ of this volume.}
that the theorem $\sqrt{2}\cdot\sqrt{3}=\sqrt{6}$ has nowhere
yet been strictly demonstrated, as justified with
respect to this work also, so excellent in many other
regards and with which I was unacquainted at that
time.

\begin{flushright}
\textsc{R. Dedekind.}\mbox\qquad
\end{flushright}

\textsc{Harzburg}, October 5, 1887.

% [File: 046.png]
\newpage
\begin{center}
{\large PREFACE TO THE SECOND EDITION.}
\end{center}

The present memoir soon after its appearance met
with both favorable and unfavorable criticisms;
indeed serious faults were charged against it. I have
been unable to convince myself of the justice of these
charges, and I now issue a new edition of the memoir,
which for some time has been out of print, without
change, adding only the following notes to the first
preface.

The property which I have employed as the definition
of the infinite system had been pointed out before
the appearance of my paper by G.~Cantor (\textit{Ein
Beitrag zur Mannigfaltigkeitslehre}, \textit{Crelle's Journal}, Vol.~%
84, 1878), as also by Bolzano (\textit{Paradoxien des Unendlichen},
�~20, 1851). But neither of these authors made
the attempt to use this property for the definition of
the infinite and upon this foundation to establish with
rigorous logic the science of numbers, and just in this
consists the content of my wearisome labor which in
all its essentials I had completed several years before
the appearance of Cantor's memoir and at a time
when the work of Bolzano was unknown to me even
by name. For the benefit of those who are interested
in and understand the difficulties of such an investigation,
% [File: 047.png]
I add the following remark. We can lay down
an entirely different definition of the finite and infinite,
which appears still simpler since the notion of similarity
of transformation is not even assumed, viz.:

``A system $S$ is said to be finite when it may be so
transformed in itself (36) that no proper part (6) of $S$
is transformed in itself; in the contrary case $S$ is
called an infinite system.''

Now let us attempt to erect our edifice upon this
new foundation! We shall soon meet with serious
difficulties, and I believe myself warranted in saying
that the proof of the perfect agreement of this definition
with the former can be obtained only (and then
easily) when we are permitted to assume the series of
natural numbers as already developed and to make
use of the final considerations in (131); and yet nothing
is said of all these things in either the one definition
or the other! From this we can see how very
great is the number of steps in thought needed for
such a remodeling of a definition.

About a year after the publication of my memoir
I became acquainted with G.~Frege's \textit{Grundlagen der
Arithmetik}, which had already appeared in the year
1884. However different the view of the essence of
number adopted in that work is from my own, yet it
contains, particularly from �~79 on, points of very
close contact with my paper, especially with my definition
(44). The agreement, to be sure, is not easy
to discover on account of the different form of expression;
% [File: 048.png]
but the positiveness with which the author
speaks of the logical inference from $n$ to $n+1$ (page
\pageref{Ind132}, below) shows plainly that here he stands upon the
same ground with me. In the meantime E.~Schr�der's
\textit{Vorlesungen �ber die Algebra der Logik} has been almost
completed (1890--1891). Upon the importance of this
extremely suggestive work, to which I pay my highest
tribute, it is impossible here to enter further; I will
simply confess that in spite of the remark made on
p.~253 of Part~I., I have retained my somewhat clumsy
symbols (8) and (17); they make no claim to be
adopted generally but are intended simply to serve
the purpose of this arithmetic paper to which in my
view they are better adapted than sum and product
symbols.

\begin{flushright}
\textsc{R. Dedekind.}\mbox\qquad
\end{flushright}

\textsc{Harzburg}, August 24, 1893.

% [File: 049.png]
\newpage
\begin{center}
\large THE NATURE AND MEANING OF NUMBERS.
\end{center}

%I.

\mysect{XVII}{I}{Systems of Elements}

\mypara{1.} In what follows I understand by \textit{thing} every
object of our thought. In order to be able easily to
speak of things, we designate them by symbols, e.~g.,
by letters, and we venture to speak briefly of the
thing $a$ or of $a$ simply, when we mean the thing denoted
by $a$ and not at all the letter $a$ itself. A thing
is completely determined by all that can be affirmed
or thought concerning it. A thing $a$ is the same as $b$
(identical with $b$), and $b$ the same as $a$, when all that
can be thought concerning $a$ can also be thought concerning
$b$, and when all that is true of $b$ can also be
thought of $a$. That $a$ and $b$ are only symbols or names
for one and the same thing is indicated by the notation
$a=b$, and also by $b=a$. If further $b=c$, that
is, if $c$ as well as $a$ is a symbol for the thing denoted
by $b$, then is also $a=c$. If the above coincidence of
the thing denoted by $a$ with the thing denoted by $b$
does not exist, then are the things $a$, $b$ said to be different,
$a$ is another thing than $b$, $b$ another thing than
% [File: 050.png]
$a$; there is some property belonging to the one that
does not belong to the other.

\mypara{2.} It very frequently happens that different things,
$a$, $b$, $c$, \ldots\ for some reason can be considered from
a common point of view, can be associated in the
mind, and we say that they form a \textit{system} $S$; we call
the things $a$, $b$, $c$, \ldots\ \textit{elements} of the system $S$, they
are \textit{contained} in $S$; conversely, $S$ \textit{consists} of these
elements. Such a system $S$ (an aggregate, a manifold,
a totality) as an object of our thought is likewise
a thing (1); it is completely determined when
with respect to every thing it is determined whether
it is an element of $S$ or not.\footnote
  {In what manner this determination is brought about, and whether we
know a way of deciding upon it, is a matter of indifference for all that follows;
the general laws to be developed in no way depend upon it; they hold under
all circumstances. I mention this expressly because Kronecker not long ago
(\textit{Crelle's Journal}, Vol.~99, pp.~334--336) has endeavored to impose certain limitations
upon the free formation of concepts in mathematics which I do not
believe to be justified; but there seems to be no call to enter upon this matter
with more detail until the distinguished mathematician shall have published
his reasons for the necessity or merely the expediency of these
  limitations.}
The system $S$ is hence
the same as the system $T$, in symbols $S=T$, when
every element of $S$ is also element of $T$, and every
element of $T$ is also element of $S$. For uniformity of
expression it is advantageous to include also the special
case where a system $S$ consists of a \textit{single} (one
and only one) element $a$, i.~e., the thing $a$ is element
of $S$, but every thing different from $a$ is not an element
of $S$. On the other hand, we intend here for
certain reasons wholly to exclude the empty system
which contains no element at all, although for other
% [File: 051.png]
investigations it may be appropriate to imagine such
a system.

\mypara{3.} Definition.  A system $A$ is said to be \textit{part} of a
system $S$ when every element of $A$ is also element of
$S$. Since this relation between a system $A$ and a system
$S$ will occur continually in what follows, we shall
express it briefly by the symbol $A \partof S$. The inverse
symbol $S \wholeof A$, by which the same fact might be expressed,
for simplicity and clearness I shall wholly
avoid, but for lack of a better word I shall sometimes
say that $S$ is \textit{whole} of $A$, by which I mean to express
that among the elements of $S$ are found all the elements
of $A$. Since further every element $s$ of a system
$S$ by (2) can be itself regarded as a system, we can
hereafter employ the notation $s \partof S$.

\mypara{4.} Theorem.  $A \partof A$, by reason of (3).

\mypara{5.} Theorem.  If $A \partof B$ and $B \partof A$, then $A = B$.

The proof follows from (3), (2).

\mypara{6.} Definition. A system $A$ is said to be a \textit{proper}
[\textit{echter}] part of $S$, when $A$ is part of $S$, but different
from $S$.  According to (5) then $S$ is not a part of $A$,
i.~e., there is in $S$ an element which is not an element
of $A$.

\mypara{7.} Theorem.  If $A \partof B$ and $B \partof C$, which may be
denoted briefly by $A \partof B \partof C$, then is $A \partof C$, and $A$ is
certainly a proper part of $C$, if $A$ is a proper part of
$B$ or if $B$ is a proper part of $C$.

The proof follows from (3), (6).

\mypara{8.} Definition.  By the system \textit{compounded} out of
% [File: 052.png]
any systems $A$, $B$, $C$,~\ldots\ to be denoted by $\mathfrak{M}(A, B,
C,\,\ldots)$ we mean that system whose elements are determined
by the following prescription: a thing is
considered as element of $\mathfrak{M}(A, B, C,\,\ldots)$ when and
only when it is element of some one of the systems
$A$, $B$, $C$,~\ldots, i.~e., when it is element of $A$, or $B$, or
$C$,~\ldots. We include also the case where only a single
system $A$ exists; then obviously $\mathfrak{M}(A)=A$. We
observe further that the system $\mathfrak{M}(A, B, C,\,\ldots)$
compounded out of $A$, $B$, $C$,~\ldots\ is carefully to be distinguished
from the system whose elements are the
systems $A$, $B$, $C$,~\ldots\ themselves.

\mypara{9.} Theorem. The systems $A$, $B$, $C$, \ldots\ are parts
of $\mathfrak{M}(A, B, C,\,\ldots)$.

The proof follows from (8), (3).

\mypara{10.} Theorem.  If $A$, $B$, $C$,~\ldots\ are parts of a system
$S$, then is $\mathfrak{M}(A, B, C,\,\ldots)\\ \partof S$.

The proof follows from (8), (3).

\mypara{11.} Theorem.  If $P$ is part of one of the systems
$A$, $B$, $C$,~\ldots\ then is $P \partof \mathfrak{M}(A, B, C,\,\ldots)$.

The proof follows from (9), (7).

\mypara{12.} Theorem.  If each of the systems $P$, $Q$,~\ldots\
is part of one of the systems $A$, $B$, $C$,~\ldots\ then is
$\mathfrak{M}(P, Q,\,\ldots) \partof \mathfrak{M}(A, B, C,\,\ldots)$.

The proof follows from (11), (10).

\mypara{13.} Theorem. If $A$ is compounded out of any of
the systems $P$, $Q$,~\ldots\ then is $A \partof \mathfrak{M}(P, Q,\,\ldots)$.

Proof. For every element of $A$ is by (8) element
of one of the systems $P$, $Q$,~\ldots, consequently by (8)
% [File: 053.png]
also element of $\mathfrak{M}(P, Q,\,\ldots)$, whence the theorem
follows by (3).

\mypara{14.} Theorem. If each of the systems $A$, $B$, $C$,~\ldots\
is compounded out of any of the systems $P$, $Q$,~\ldots\
then is
\[
\mathfrak{M}(A,B,C,\,\ldots) \partof \mathfrak{M}(P,Q,\,\ldots)
\]

The proof follows from (13), (10).

\mypara{15.} Theorem.  If each of the systems $P$, $Q$,~\ldots\
is part of one of the systems $A$, $B$, $C$,~\ldots, and if
each of the latter is compounded out of any of the
former, then is
\[
\mathfrak{M}(P, Q,\, \ldots) = \mathfrak{M}(A, B, C,\, \ldots).
\]

The proof follows from (12), (14), (5).

\mypara{16.} Theorem.  If
\begin{gather*}
A = \mathfrak{M}(P, Q) \mathrm{\ and\ } B = \mathfrak{M}(Q, R)  \\
\mathrm{then\ is\ } \mathfrak{M}(A, R) = \mathfrak{M}(P, B).
\end{gather*}

Proof.  For by the preceding theorem (15)
\[
\mathfrak{M}(A, R) \mathrm{\ as\ well\ as\ } \mathfrak{M}(P, B) = \mathfrak{M}(P, Q, R).
\]

\mypara{17.} Definition.  A thing $g$ is said to be \textit{common}
element of the systems $A$, $B$, $C$,~\ldots, if it is contained
in each of these systems (that is in $A$ \textit{and} in $B$ \textit{and}
in $C$~\ldots). Likewise a system $T$ is said to be a \textit{common
part} of $A$, $B$, $C$,~\ldots\ when $T$ is part of each of
these systems; and by the \textit{community} [\textit{Gemeinheit}] of
the systems $A$, $B$, $C$,~\ldots\ we understand the perfectly
determinate system $\mathfrak{G}(A, B, C,\,\ldots)$ which consists
of all the common elements $g$ of $A$, $B$, $C$,~\ldots\ and
% [File: 054.png]
hence is likewise a common part of those systems.
We again include the case where only a single system
$A$ occurs; then $\mathfrak{G}(A)$ (is to be put) $=A$. But the
case may also occur that the systems $A$, $B$, $C$,~\ldots\
possess no common element at all, therefore no common
part, no community; they are then called systems
\textit{without} common part, and the symbol $\mathfrak{G}(A, B,
C,\,\ldots)$ is meaningless (compare the end of (2)).
We shall however almost always in theorems concerning
communities leave it to the reader to add in
thought the condition of their existence and to discover
the proper interpretation of these theorems for
the case of non-existence.

\mypara{18.} Theorem. Every common part of $A$, $B$, $C$,~\ldots\
is part of $\mathfrak{G}(A, B, C,\,\ldots)$.

The proof follows from (17).

\mypara{19.} Theorem. Every part of $\mathfrak{G}(A, B, C,\,\ldots)$ is
common part of $A$, $B$, $C$,~\ldots.

The proof follows from (17), (7).

\mypara{20.} Theorem. If each of the systems $A$, $B$, $C$,~\ldots\
is whole (3) of one of the systems $P$, $Q$,~\ldots\ then is
\[
\mathfrak{G}(P, Q,\,\ldots) \partof \mathfrak{G}(A, B, C,\,\ldots)
\]

Proof. For every element of $\mathfrak{G}(P, Q,\,\ldots)$ is
common element of $P$, $Q$,~\ldots, therefore also common
element of $A$, $B$, $C$,~\ldots, which was to be proved.
% [File: 055.png]

%II.

\mysect{XVI}{II}{Transformation of a System}

\mypara{21.} Definition.\footnote
  {See Dirichlet's \textit{Vorlesungen �ber Zahlentheorie}, 3d edition, 1879, �~163.}
By a \textit{transformation} [\textit{Abbildung}]
$\phi$ of a system $S$ we understand a law according to
which to every determinate element $s$ of $S$ there \textit{belongs}
a determinate thing which is called the \textit{transform}
of $s$ and denoted by $\phi(s)$; we say also that $\phi(s)$
\textit{corresponds} to the element $s$, that $\phi(s)$ \textit{results} or is
\textit{produced} from $s$ by the transformation $\phi$, that $s$ is
\textit{transformed} into $\phi(s)$ by the transformation $\phi$. If now
$T$ is any part of $S$, then in the transformation $\phi$ of $S$
is likewise contained a determinate transformation of
$T$, which for the sake of simplicity may be denoted by
the same symbol $\phi$ and consists in this that to every
element $t$ of the system $T$ there corresponds the same
transform $\phi(t)$, which $t$ possesses as element of $S$; at
the same time the system consisting of all transforms
$\phi(t)$ shall be called the transform of $T$ and be denoted
by $\phi(T)$, by which also the significance of $\phi(S)$ is
defined. As an example of a transformation of a system
we may regard the mere assignment of determinate
symbols or names to its elements. The simplest
transformation of a system is that by which each
of its elements is transformed into itself; it will be
called the \textit{identical} transformation of the system. For
convenience, in the following theorems (22), (23),
(24), which deal with an arbitrary transformation $\phi$ of
% [File: 056.png]
an arbitrary system $S$, we shall denote the transforms
of elements $s$ and parts $T$ respectively by $s'$ and $T'$;
in addition we agree that small and capital italics
without accent shall always signify elements and parts
of this system $S$.

\mypara{22.} Theorem.\footnote{See theorem~27.} If $A \partof B$ then $A' \partof B'$.

Proof. For every element of $A'$ is the transform
of an element contained in $A$, and therefore also in $B$,
and is therefore element of $B'$, which was to be proved.

\mypara{23.} Theorem. The transform of $\mathfrak{M}(A, B, C,\,\ldots)$
is $\mathfrak{M}(A', B', C',\,\ldots)$.

Proof. If we denote the system $\mathfrak{M}(A, B, C,\, \ldots)$
which by (10) is likewise part of $S$ by $M$, then is every
element of its transform $M'$ the transform $m'$ of an
element $m$ of $M$; since therefore by (8) $m$ is also element
of one of the systems $A$, $B$, $C$,~\ldots\ and consequently
$m'$ element of one of the systems $A'$, $B'$, $C'$,~\ldots,
and hence by (8) also element of $\mathfrak{M}(A', B', C',\, \ldots)$,
we have by (3)
\[
M' \partof \mathfrak{M}(A', B', C',\,\ldots).
\]
On the other hand, since $A$, $B$, $C$,~\ldots\ are by (9) parts
of $M$, and hence $A'$, $B'$, $C'$,~\ldots\ by (22) parts of $M'$,
we have by (10)
\[
\mathfrak{M}(A', B', C',\,\ldots) \partof M'.
\]
By combination with the above we have by (5) the
theorem to be proved
\[
M' = \mathfrak{M}(A', B', C',\, \ldots).
\]

% [File: 057.png]

\mypara{24.} Theorem.\footnote{See theorem~29.} The transform of every common
part of $A$, $B$, $C$,~\ldots, and therefore that of the community
$\mathfrak{G}(A, B, C,\,\ldots)$ is part of $\mathfrak{G}(A', B', C',\,\ldots)$.

Proof. For by (22) it is common part of $A'$, $B'$,
$C'$,~\ldots, whence the theorem follows by (18).

\mypara{25.} Definition and theorem. If $\phi$ is a transformation
of a system $S$, and $\psi$ a transformation of the
transform $S'=\phi(S)$, there always results a transformation
$\theta$ of S, \textit{compounded}\footnote
  {A confusion of this compounding of transformations with that of systems
  of elements is hardly to be feared.}
out of $\phi$ and $\psi$, which consists
of this that to every element $s$ of $S$ there corresponds
the transform
\[
  \theta(s)=\psi(s')=\psi\bigl(\phi(s)\bigr),
\]
where again we have put $\phi(s)=s'$. This transformation
$\theta$ can be denoted briefly by the symbol $\psi\centerdot\phi$ or
$\psi\phi$, the transform $\theta(s)$ by $\psi\phi(s)$ where the order of
the symbols $\phi$, $\psi$ is to be considered, since in general
the symbol $\phi\psi$ has no interpretation and actually has
meaning only when $\psi(s') \partof s$. If now $\chi$ signifies a
transformation of the system $\psi(s')=\psi\phi(s)$ and $\eta$ the
transformation $\chi\psi$ of the system $S'$ compounded out
of $\psi$ and $\chi$, then is $\chi\theta(s)=\chi\psi(s')=\eta(s')=\eta\phi(s)$;
therefore the compound transformations $\chi\theta$ and $\eta\phi$
coincide for every element $s$ of $S$, i.~e., $\chi\theta = \eta \phi$. In
accordance with the meaning of $\theta$ and $\eta$ this theorem
can finally be expressed in the form
\[
  \chi\centerdot\psi\phi = \chi\psi\centerdot\phi,
\]
% [File: 058.png]
and this transformation compounded out of $\phi$, $\psi$, $\chi$
can be denoted briefly by $\chi\psi\phi$.

%III.

\mysect{XV}{III}{Similarity of a Transformation. Similar Systems}

\mypara{26.} Definition.  A transformation $\phi$ of a system $S$
is said to be \textit{similar} [\textit{�hnlich}] or \textit{distinct}, when to different
elements $a$, $b$ of the system $S$ there always correspond
different transforms $a' = \phi(a)$, $b' = \phi(b)$.
Since in this case conversely from $s' = t'$ we always
have $s = t$, then is every element of the system $S' =
\phi(S)$ the transform $s'$ of a single, perfectly determinate
element $s$ of the system $S$, and we can therefore
set over against the transformation $\phi$ of $S$ an \textit{inverse}
transformation of the system $S'$, to be denoted by $\overline{\phi}$,
which consists in this that to every element $s'$ of $S'$
there corresponds the transform $\overline{\phi}(s') = s$, and obviously
this transformation is also similar. It is clear that
$\overline{\phi}(S') = S$, that further $\phi$ is the inverse transformation
belonging to $\overline{\phi}$ and that the transformation $\overline{\phi}\phi$ compounded
out of $\phi$ and $\overline{\phi}$ by (25) is the identical transformation
of $S$ (21). At once we have the following
additions to II., retaining the notation there given.

\mypara{27.} Theorem.\footnote{See theorem~22.} If $A' \partof B'$, then $A \partof B$.

Proof. For if $a$ is an element of $A$ then is $a'$ an
element of $A'$, therefore also of $B'$, hence $=b'$, where
$b$ is an element of $B$; but since from $a' = b'$ we always
% [File: 059.png]
have $a = b$, then is every element of $A$ also element of
$B$, which was to be proved.

\mypara{28.} Theorem.  If $A' = B'$, then $A = B$.

The proof follows from (27), (4), (5).

\mypara{29.} Theorem.\footnote{See theorem~24.} If $G = \mathfrak{G}(A, B, C,\, \ldots)$, then
\[
G' = \mathfrak{G}(A', B', C',\, \ldots).
\]

Proof. Every element of $\mathfrak{G}(A', B', C',\, \ldots)$ is
certainly contained in $S'$, and is therefore the transform
$g'$ of an element $g$ contained in $S$; but since $g'$
is common element of $A'$, $B'$, $C'$,~\ldots\ then by (27) must
$g$ be common element of $A$, $B$, $C$,~\ldots\ therefore also
element of $G$; hence every element of $\mathfrak{G}(A', B',
C',\, \ldots)$ is transform of an element $g$ of $G$, therefore
element of $G'$, i.~e., $\mathfrak{G}(A', B', C',\,\ldots) \partof G'$, and accordingly
our theorem follows from (24), (5).

\mypara{30.} Theorem. The identical transformation of a
system is always a similar transformation.

\mypara{31.} Theorem. If $\phi$ is a similar transformation of
$S$ and $\psi$ a similar transformation of $\phi(S)$, then is the
transformation $\psi\phi$ of $S$, compounded of $\phi$ and $\psi$, a similar
transformation, and the associated inverse transformation
$\overline{\psi}\,\overline{\phi} = \overline{\phi}\,\overline{\psi}$.

Proof. For to different elements $a$, $b$ of $S$ correspond
different transforms $a' = \phi(a)$, $b'=\phi(b)$, and
to these again different transforms $\psi(a')=\psi\phi(a)$,
$\psi(b') =\psi\phi(b)$ and therefore $\psi\phi$ is a similar transformation.
Besides every element $\psi\phi(s) = \psi(s')$ of the
system $\psi\phi(S)$ is transformed by $\overline{\psi}$ into $s' = \phi(s)$ and
% [File: 060.png]
this by $\overline{\phi}$ into $s$, therefore $\psi\phi(s)$ is transformed by
$\overline{\phi}\,\overline{\psi}$ into $s$, which was to be proved.

\mypara{32.} Definition. The systems $R$, $S$ are said to be
\textit{similar} when there exists such a similar transformation
$\phi$ of $S$ that $\phi(S)=R$, and therefore $\overline{\phi}(R) = S$.
Obviously by (30) every system is similar to itself.

\mypara{33.} Theorem. If $R$, $S$ are similar systems, then
every system $Q$ similar to $R$ is also similar to $S$.

Proof. For if $\phi$, $\psi$ are similar transformations of
$S$, $R$ such that $\phi(S)=R$, $\psi(R)=Q$, then by (31) $\psi\phi$
is a similar transformation of $S$ such that $\psi\phi(S) = Q$,
which was to be proved.

\mypara{34.} Definition. We can therefore separate all systems
into \textit{classes} by putting into a determinate class
all systems $Q$, $R$, $S$,~\ldots, and only those, that are
similar to a determinate system $R$, the \textit{representative}
of the class; according to (33) the class is not changed
by taking as representative any other system belonging
to it.

\mypara{35.} Theorem. If $R$, $S$ are similar systems, then
is every part of $S$ also similar to a part of $R$, every
proper part of $S$ also similar to a proper part of $R$.

Proof. For if $\phi$ is a similar transformation of $S$,
$\phi(S)=R$, and $T \partof S$, then by (22) is the system similar
to $T$ $\phi(T) \partof R$; if further $T$ is proper part of $S$,
and $s$ an element of $S$ not contained in $T$, then by (27)
the element $\phi(s)$ contained in $R$ cannot be contained
in $\phi(T)$; hence $\phi(T)$ is proper part of $R$, which was
to be proved.

% [File: 061.png]

%IV.

\mysect{XII}{IV}{Transformation of a System in Itself}

\mypara{36.} Definition. If $\phi$ is a similar or dissimilar transformation
of a system $S$, and $\phi(S)$ part of a system
$Z$, then $\phi$ is said to be a transformation of $S$ \textit{in} $Z$, and
we say $S$ is transformed by $\phi$ in $Z$. Hence we call
$\phi$ a transformation of the system $S$ \textit{in itself}, when
$\phi(S) \partof S$, and we propose in this paragraph to investigate
the general laws of such a transformation $\phi$.  In
doing this we shall use the same notations as in II.
and again put $\phi(s)=s'$, $\phi(T)=T'$. These transforms
$s'$, $T'$ are by (22), (7) themselves again elements
or parts of $S$, like all things designated by italic
letters.

\mypara{37.} Definition.  $K$ is called a \textit{chain} [\textit{Kette}], when
$K' \partof K$. We remark expressly that this name does
not in itself belong to the part $K$ of the system $S$, but
is given only with respect to the particular transformation
$\phi$; with reference to another transformation
of the system $S$ in itself $K$ can very well not be a
chain.

\mypara{38.} Theorem. $S$ is a chain.

\mypara{39.} Theorem.  The transform $K'$ of a chain $K$ is
a chain.

Proof.  For from $K' \partof K$ it follows by (22) that
$(K')' \partof K'$, which was to be proved.

\mypara{40.} Theorem.  If $A$ is part of a chain $K$, then is
also $A' \partof K$.

% [File: 062.png]

Proof. For from $A \partof K$ it follows by (22) that
$A' \partof K'$, and since by (37) $K' \partof K$, therefore by (7)
$A' \partof K$, which was to be proved.

\mypara{41.} Theorem.  If the transform $A'$ is part of a
chain $L$, then is there a chain $K$, which satisfies the
conditions $A \partof K$, $K' \partof L$; and $\mathfrak{M}(A,L)$ is just such a
chain $K$.

Proof. If we actually put $K=\mathfrak{M}(A, L)$, then by
(9) the one condition $A \partof K$ is fulfilled. Since further
by (23) $K' = \mathfrak{M}(A',L')$ and by hypothesis $A' \partof L$,
$L' \partof L$, then by (10) is the other condition $K' \partof L$ also
fulfilled and hence it follows because by (9) $L \partof K$,
that also $K' \partof K$, i.~e., $K$ is a chain, which was to be
proved.

\mypara{42.} Theorem. A system $M$ compounded simply
out of chains $A$, $B$, $C$,~\ldots\ is a chain.

Proof. \label{chain}Since by (23) $M'= \mathfrak{M}(A', B', C',\, \ldots)$ and
by hypothesis $A' \partof A$, %[**Typo corrected, see note at end.]
$B' \partof B$, $C' \partof C$,~\ldots\ therefore by
(12) $M' \partof M$, which was to be proved.

\mypara{43.} Theorem.  The community $G$ of chains $A$,  %[**Comma added, assumed just poor image.]
$B$, $C$,~\ldots\ is a chain.

Proof. Since by (17) $G$ is common part of $A$, $B$,
$C$,~\ldots, therefore by (22) $G'$ common part of $A'$, $B'$,
$C'$,~\ldots, and by hypothesis $A' \partof A$, $B' \partof B$, $C' \partof C$,~\ldots,
then by (7) $G'$ is also common part of $A$, $B$, $C$,~\ldots\
and hence by (18) also part of $G$, which was to be
proved.

\mypara{44.} Definition.  If $A$ is any part of $S$, we will denote
by $A_0$ the community of all those chains (e.~g., $S$)
% [File: 063.png]
of which $A$ is part; this community $A_0$ exists (17) because
$A$ is itself common part of all these chains.
Since further by (43) $A_0$ is a chain, we will call $A_0$
the \textit{chain of the system} $A$, or briefly the chain of $A$.
This definition too is strictly related to the fundamental
determinate transformation $\phi$ of the system $S$ in
itself, and if later, for the sake of clearness, it is
necessary we shall at pleasure use the symbol $\phi_0(A)$
instead of $A_0$, and likewise designate the chain of $A$
corresponding to another transformation $\omega$ by $\omega_0(A)$.
For this very important notion the following theorems
hold true.

\mypara{45.} Theorem. $A \partof A_0$.

Proof. For $A$ is common part of all those chains
whose community is $A_0$, whence the theorem follows
by (18).

\mypara{46.} Theorem. $(A_0)' \partof A_0$.

Proof. For by (44) $A_0$ is a chain (37).

\mypara{47.} Theorem. If $A$ is part of a chain $K$, then is
also $A_0 \partof K$.

Proof. For $A_0$ is the community and hence also
a common part of all the chains $K$, of which $A$ is
part.

\mypara{48.} Remark. One can easily convince himself that
the notion of the chain $A_0$ defined in (44) is completely
characterised by the preceding theorems, (45),
(46), (47).

\mypara{49.} Theorem. $A' \partof (A_0)'$.

The proof follows from (45), (22).

% [File: 064.png]

\mypara{50.} Theorem. $A' \partof A_0$.

The proof follows from (49), (46), (7).

\mypara{51.} Theorem. If $A$ is a chain, then $A_0=A$.

Proof. Since $A$ is part of the chain $A$, then by
(47) $A_0 \partof A$, whence the theorem follows by (45), (5).

\mypara{52.} Theorem. If $B \partof A$, then $B \partof A_0$.

The proof follows from (45), (7).

\mypara{53.} Theorem. If $B \partof A_0$, then $B_0 \partof A_0$, and conversely.

Proof. Because $A_0$ is a chain, then by (47) from
$B \partof A_0$, we also get $B_0 \partof A_0$; conversely, if $B_0 \partof A_0$, then
by (7) we also get $B \partof A_0$, because by (45) $B \partof B_0$.

\mypara{54.} Theorem. If $B \partof A$, then is $B_0 \partof A_0$.

The proof follows from (52), (53).

\mypara{55.} Theorem. If $B \partof A_0$, then is also $B' \partof A_0$.

Proof. For by (53) $B_0 \partof A_0$, and since by (50) $B' \partof B_0$,
the theorem to be proved follows by (7). The same
result, as is easily seen, can be obtained from (22),
(46), (7), or also from (40).

\mypara{56.} Theorem. If $B \partof A_0$, then is $(B_0)' \partof (A_0)'$.

The proof follows from (53), (22).

\mypara{57.} Theorem and definition. $(A_0)'=(A')_0$, i.~e.,
the transform of the chain of $A$ is at the same time
the chain of the transform of $A$. Hence we can designate
this system in short by $A'_0$ and at pleasure call it
the \textit{chain-transform} or \textit{transform-chain} of $A$. With the
clearer notation given in (44) the theorem might be
expressed by $\phi\bigl(\phi_0(A)\bigr) = \phi_0\bigl(\phi(A)\bigr)$.

Proof. If for brevity we put $(A')_0=L$, $L$ is a
% [File: 065.png]
chain (44) and by (45) $A' \partof L$; hence by (41) there exists
a chain $K$ satisfying the conditions $A \partof K$, $K' \partof L$;
hence from (47) we have $A_0 \partof K$, therefore $(A_0)' \partof K'$,
and hence by (7) also $(A_0)' \partof L$, i.~e.,
\[
(A_0)' \partof (A')_0.
\]
Since further by (49) $A' \partof (A_0)'$, and by (44), (39)
$(A_0)'$ is a chain, then by (47) also
\[
(A')_0 \partof (A_0)',
\]
whence the theorem follows by combining with the
preceding result (5).

\mypara{58.} Theorem. $A_0=\mathfrak{M}(A, A'_0)$, i.~e., the chain of
$A$ is compounded out of $A$ and the transform-chain
of $A$.

Proof. If for brevity we again put
\[
L=A'_0 =(A_0)'=(A')_0 \text{\ and\ } K=\mathfrak{M}(A, L),
\]
then by (45) $A' \partof L$, and since $L$ is a chain, by (41)
the same thing is true of $K$; since further $A \partof K$ (9),
therefore by (47)
\[
A_0 \partof K.
\]
On the other hand, since by (45) $A \partof A_0$, and by (46)
also $L \partof A_0$, then by (10) also
\[
K \partof A_0,
\]
whence the theorem to be proved $A_0=K$ follows by
combining with the preceding result (5).

\mypara{59.} Theorem of complete induction. In order to
show that the chain $A_0$ is part of any system $\Sigma$---be
this latter part of $S$ or not---it is sufficient to show,

$\rho$.  that $A \partof \Sigma$, and

% [File: 066.png]

$\sigma$. that the transform of every common element of
$A_0$ and $\Sigma$ is likewise element of $\Sigma$.

Proof. For if $\rho$ is true, then by (45) the community
$G=\mathfrak{G}(A_0,\Sigma)$ certainly exists, and by (18)
$A \partof G$; since besides by (17)
\[
G \partof A_0,
\]
then is $G$ also part of our system $S$, which by $\phi$ is
transformed in itself and at once by (55) we have also
$G' \partof A_0$. If then $\sigma$ is likewise true, i.~e.\ if $G' \partof \Sigma$, then
must $G'$ as common part of the systems $A_0$, $\Sigma$ by (18)
be part of their community $G$, i.~e. $G$ is a chain (37),
and since, as above noted, $A \partof G$, then by (47) is also
\[
A_0 \partof G,
\]
and therefore by combination with the preceding result
$G=A_0$, hence by (17) also $A_0 \partof \Sigma$, which was to
be proved.

\mypara{60.} The preceding theorem, as will be shown later,
forms the scientific basis for the form of demonstration
known by the name of complete induction (the
inference from $n$ to $n+1$); it can also be stated in
the following manner: In order to show that all elements
of the chain $A_0$ possess a certain property $\mathfrak{E}$
(or that a theorem $\mathfrak{S}$ dealing with an undetermined
thing $n$ actually holds good for all elements $n$ of the
chain $A_0$) it is sufficient to show

$\rho$. that all elements $a$ of the system $A$ possess the
property $\mathfrak{E}$ (or that $\mathfrak{S}$ holds for all $a$'s) and

$\sigma$. that to the transform $n'$ of every such element
$n$ of $A_0$ possessing the property $\mathfrak{E}$, belongs the same
% [File: 067.png]
property $\mathfrak{E}$ (or that the theorem $\mathfrak{S}$, as soon as it holds
for an element $n$ of $A_0$, certainly must also hold for
its transform $n'$).

Indeed, if we denote by $\Sigma$ the system of all things
possessing the property $\mathfrak{E}$ (or for which the theorem
$\mathfrak{S}$ holds) the complete agreement of the present manner
of stating the theorem with that employed in (59)
is immediately obvious.

\mypara{61.} Theorem.  The chain of $\mathfrak{M}(A, B, C,\, \ldots)$ is
$\mathfrak{M}(A_0,B_0,C_0,\,\ldots)$.

Proof. If we designate by $M$ the former, by $K$
the latter system, then by (42) $K$ is a chain. Since
then by (45) each of the systems $A$, $B$, $C$,~\ldots\ is part
of one of the systems $A_0$, $B_0$, $C_0$,~\ldots, and therefore
by (12) $M \partof K$, then by (47) we also have
\[
M_0 \partof K.
\]
On the other hand, since by (9) each of the systems
$A$, $B$, $C$,~\ldots\ is part of $M$, and hence by (45), (7)
also part of the chain $M_0$, then by (47) must also each
of the systems $A_0$, $B_0$, $C_0$,~\ldots\ be part of $M_0$, therefore
by (10)
\[
K \partof M_0
\]
whence by combination with the preceding result follows
the theorem to be proved $M_0=K$ (5).

\mypara{62.} Theorem. The chain of $\mathfrak{G}(A, B, C,\,\ldots)$ is
part of $\mathfrak{G}(A_0, B_0, C_0,\,\ldots)$.

Proof. If we designate by $G$ the former, by $K$ the
latter system, then by (43) $K$ is a chain. Since then
each of the systems $A_0$, $B_0$, $C_0$,~\ldots\ by (45) is whole
% [File: 068.png]
of one of the systems $A$, $B$, $C$,~\ldots, and hence by (20)
$G \partof K$, therefore by (47) we obtain the theorem to be
proved $G_0 \partof K$.

\mypara{63.} Theorem. If $K' \partof L \partof K$, and therefore $K$ is a
chain, $L$ is also a chain.  If the same is proper part
of $K$, and $U$ the system of all those elements of $K$
which are not contained in $L$, and if further the chain
$U_0$ is proper part of $K$, and $V$ the system of all those
elements of $K$ which are not contained in $U_0$, then is
$K=\mathfrak{M}(U_0, V)$ and $L = \mathfrak{M}(U'_0, V)$. If finally $L=K'$
then $V \partof V'$.

The proof of this theorem of which (as of the two
preceding) we shall make no use may be left for the
reader.

%V.

\mysect{XIII}{V}{The Finite and Infinite}

\mypara{64.} Definition.\footnote
  {If one does not care to employ the notion of similar systems (32) he must
say: $S$ is said to be infinite, when there is a proper part of $S$ (6) in which $S$
can be distinctly (similarly) transformed (26), (36). In this form I submitted
the definition of the infinite which forms the core of my whole investigation
in September, 1882, to G.~Cantor and several years earlier to Schwarz and
Weber. All other attempts that have come to my knowledge to distinguish
the infinite from the finite seem to me to have met with so little success that
  I think I may be permitted to forego any criticism of them.}
A system $S$ is said to be \textit{infinite}
when it is similar to a proper part of itself (32); in
the contrary case $S$ is said to be a \textit{finite} system.

\mypara{65.} Theorem. Every system consisting of a single
element is finite.

Proof. For such a system possesses no proper
part (2), (6).

% [File: 069.png---\michael.j.white\\sahobart\glimpseofchaos\wwoods\-----

\mypara{66.} Theorem.  There exist infinite systems.

Proof.\footnote
  {A similar consideration is found in �~13 of the \textit{Paradoxien des Unendlichen}
  by Bolzano (Leipzig, 1851).}
My own realm of thoughts, i.~e., the totality
$S$ of all things, which can be objects of my
thought, is infinite. For if $s$ signifies an element of
$S$, then is the thought $s'$, that $s$ can be object of my
thought, itself an element of $S$. If we regard this as
transform $\phi(s)$ of the element $s$ then has the transformation
$\phi$ of $S$, thus determined, the property that the
transform $S'$ is part of $S$; and $S'$ is certainly proper
part of $S$, because there are elements in $S$ (e.~g., my
own ego) which are different from such thought $s'$ and
therefore are not contained in $S'$. Finally it is clear
that if $a$, $b$ are different elements of $S$, their transforms
$a'$, $b'$ are also different, that therefore the transformation
$\phi$ is a distinct (similar) transformation (26).
Hence $S$ is infinite, which was to be proved.

\mypara{67.} Theorem. If $R$, $S$ are similar systems, then is
$R$ finite or infinite according as $S$ is finite or infinite.

Proof. If $S$ is infinite, therefore similar to a proper
part $S'$ of itself, then if $R$ and $S$ are similar, $S'$ by
(33) must be similar to $R$ and by (35) likewise similar
to a proper part of $R$, which therefore by (33) is itself
similar to $R$; therefore $R$ is infinite, which was to be
proved.

\mypara{68.} Theorem.  Every system $S$, which possesses
an infinite part is likewise infinite; or, in other words,
every part of a finite system is finite.

% [File: 070.png---\michael.j.white\\sahobart\glimpseofchaos\wwoods\-----

Proof. If $T$ is infinite and there is hence such a
similar transformation $\psi$ of $T$, that $\psi(T)$ is a proper
part of $T$, then, if $T$ is part of $S$, we can extend this
transformation $\psi$ to a transformation $\phi$ of $S$ in which,
if $s$ denotes any element of $S$, we put $\phi(s)=\psi(s)$ or
$\phi(s)=s$ according as $s$ is element of $T$ or not. This
transformation $\phi$ is a similar one; for, if $a$, $b$ denote
different elements of $S$, then if both are contained in
$T$, the transform $\phi(a)=\psi(a)$ is different from the
transform $\phi(b)=\psi(b)$, because $\psi$ is a similar transformation;
if further $a$ is contained in $T$, but $b$ not, then
is $\phi(a)=\psi(a)$ different from $\phi(b)=b$, because $\psi(a)$
is contained in $T$; if finally neither $a$ nor $b$ is contained
in $T$ then also is $\phi(a)=a$ different from $\phi(b)=b$,
which was to be shown. Since further $\psi(T)$ is part
of $T$, because by (7) also part of $S$, it is clear that also
$\phi(S) \partof S$. Since finally $\psi(T)$ is proper part of $T$ there
exists in $T$ and therefore also in $S$, an element $t$, not
contained in $\psi(T)=\phi(T)$; since then the transform
$\phi(s)$ of every element $s$ not contained in $T$ is equal to
$s$, and hence is different from $t$, $t$ cannot be contained
in $\phi(S)$; hence $\phi(S)$ is proper part of $S$ and consequently
$S$ is infinite, which was to be proved.

\mypara{69.} Theorem. Every system which is similar to
a part of a finite system, is itself finite.

The proof follows from (67), (68).

\mypara{70.} Theorem.  If $a$ is an element of $S$, and if the
aggregate $T$ of all the elements of $S$ different from $a$ is
finite, then is also $S$ finite.
% [File: 071.png]

Proof. We have by (64) to show that if $\phi$ denotes
any similar transformation of $S$ in itself, the transform
$\phi(S)$ or $S'$ is never a proper part of $S$ but always
$=S$. Obviously $S=\mathfrak{M}(a,T)$ and hence by
(23), if the transforms are again denoted by accents,
$S'=\mathfrak{M}(a',T')$, and, on account of the similarity of
the transformation $\phi$, $a'$ is not contained in $T'$ (26).
Since further by hypothesis $S' \partof S$, then must $a'$ and likewise
every element of $T'$ either $=a$, or be element of
$T$. If then---a case which we will treat first---$a$ is not
contained in $T'$, then must $T' \partof T$ and hence $T'=T$,
because $\phi$ is a similar transformation and because $T$ is
a finite system; and since $a'$, as remarked, is not contained
in $T'$, i.~e., not in $T$, then must $a'=a$, and hence
in this case we actually have $S'=S$ as was stated. In
the opposite case when $a$ is contained in $T'$ and hence
is the transform $b'$ of an element $b$ contained in $T$, we
will denote by $U$ the aggregate of all those elements $u$
of $T$, which are different from $b$; then $T=\mathfrak{M}(b,U)$
and by (15) $S=\mathfrak{M}(a,b,U)$, hence $S'=\mathfrak{M}(a',a,U')$.
We now determine a new transformation $\psi$ of $T$ in
which we put $\psi(b)=a'$, and generally $\psi(u)=u'$,
whence by (23) $\psi(T)=\mathfrak{M}(a',U')$. Obviously $\psi$ is
a similar transformation, because $\phi$ was such, and because
$a$ is not contained in $U$ and therefore also $a'$ not
in $U'$. Since further $a$ and every element $u$ is different
from $b$ then (on account of the similarity of $\phi$)
must also $a'$ and every element $u'$ be different from $a$
and consequently contained in $T$; hence $\psi(T) \partof T$
% [File: 072.png]
and since $T$ is finite, therefore must $\psi(T) =T$, and
$\mathfrak{M}(a',U')= T$. From this by (15) we obtain
\[
\mathfrak{M}(a', a, U') = \mathfrak{M}(a, T)
\]
i.~e., according to the preceding $S' = S$. Therefore
in this case also the proof demanded has been secured.

%VI.

\mysect{XII}{VI}{Simply Infinite Systems. Series of Natural Numbers}

\mypara{71.} Definition. A system $N$ is said to be \textit{simply
infinite} when there exists a similar transformation $\phi$ of
$N$ in itself such that $N$ appears as chain (44) of an
element not contained in $\phi(N)$. We call this element,
which we shall denote in what follows by the
symbol 1, the \textit{base-element} of $N$ and say the simply
infinite system $N$ is \textit{set in order} [\textit{geordnet}] by this
transformation $\phi$. If we retain the earlier convenient
symbols for transforms and chains (IV) then the essence
of a simply infinite system $N$ consists in the
existence of a transformation $\phi$ of $N$ and an element 1
which satisfy the following conditions $\alpha$, $\beta$, $\gamma$, $\delta$:
\begin{itemize}
\item[]  $\alpha$. $N' \partof N$.

\item[]  $\beta$. $N=1_0$.

\item[]  $\gamma$. The element 1 is not contained in $N'$.

\item[]  $\delta$. The transformation $\phi$ is similar.
\end{itemize}
Obviously it follows from $\alpha$, $\gamma$, $\delta$ that every simply infinite
system $N$ is actually an infinite system (64) because
it is similar to a proper part $N'$ of itself.

% [File: 073.png]

\mypara{72.} Theorem. In every infinite system $S$ a simply
infinite system $N$ is contained as a part.

Proof. By (64) there exists a similar transformation
$\phi$ of $S$ such that $\phi(S)$ or $S'$ is a proper part of
$S$; hence there exists an element 1 in $S$ which is not
contained in $S'$. The chain $N=1_0$, which corresponds
to this transformation $\phi$ of the system $S$ in itself (44),
is a simply infinite system set in order by $\phi$; for the
characteristic conditions $\alpha, \beta, \gamma, \delta$ in (71) are obviously
all fulfilled.

\mypara{73.} Definition. If in the consideration of a simply
infinite system $N$ set in order by a transformation $\phi$
we entirely neglect the special character of the elements;
simply retaining their distinguishability and
taking into account only the relations to one another
in which they are placed by the order-setting transformation
$\phi$, then are these elements called \textit{natural
numbers} or \textit{ordinal numbers} or simply \textit{numbers}, and the
base-element 1 is called the \textit{base-number} of the \textit{number-series}
$N$. With reference to this freeing the elements
from every other content (abstraction) we are justified
in calling numbers a free creation of the human mind.
The relations or laws which are derived entirely from
the conditions $\alpha, \beta, \gamma, \delta$ in (71) and therefore are always
the same in all ordered simply infinite systems,
whatever names may happen to be given to the individual
elements (compare 134), form the first object of
the \textit{science of numbers} or \textit{arithmetic}. From the general
notions and theorems of IV. about the transformation
% [File: 074.png]
of a system in itself we obtain immediately the following
fundamental laws where $a$, $b$,~\ldots\ $m$, $n$,~\ldots\ always
denote elements of $N$, therefore numbers, $A$, $B$, $C$,~\ldots\
parts of $N$, $a'$, $b'$,~\ldots\ $m'$, $n'$,~\ldots\ $A'$, $B'$, $C'$~\ldots\ the
corresponding transforms, which are produced by the
order-setting transformation $\phi$ and are always elements
or parts of $N$; the transform $n'$ of a number $n$
is also called the number \textit{following} $n$.

\mypara{74.} Theorem.  Every number $n$ by (45) is contained
in its chain $n_0$ and by (53) the condition $n \partof m_0$
is equivalent to $n_0 \partof m_0$.

\mypara{75.} Theorem. By (57) $n'_0=(n_0)'=(n')_0$.

\mypara{76.} Theorem. By (46) $n'_0 \partof n_0$.

\mypara{77.} Theorem. By (58) $n_0 = \mathfrak{M}(n,n'_0)$.

\mypara{78.} Theorem. $N = \mathfrak{M}(1,N')$, hence every number
different from the base-number 1 is element of $N'$,
i.~e., transform of a number.

The proof follows from (77) and (71).

\mypara{79.} Theorem. $N$ is the only number-chain containing
the base-number 1.

Proof. For if 1 is element of a number-chain $K$,
then by (47) the associated chain $N \partof K$, hence $N=K$,
because it is self-evident that $K \partof N$.

\mypara{80.} Theorem of complete induction (inference
from $n$ to $n'$). In order to show that a theorem holds
for all numbers $n$ of a chain $m_0$, it is sufficient to show,

$\rho$. that it holds for $n=m$, and

$\sigma$. that from the validity of the theorem for a number
% [File: 075.png]
$n$ of the chain $m_0$ its validity for the following
number $n'$ always follows.

This results immediately from the more general
theorem (59) or (60). The most frequently occurring
case is where $m = 1$ and therefore $m_0$ is the complete
number-series $N$.

%VII.

\mysect{XI}{VII}{Greater and Less Numbers}

\mypara{81.} Theorem. Every number $n$ is different from
the following number $n'$.

Proof by complete induction (80):

$\rho$. The theorem is true for the number $n = 1$, because
it is not contained in $N'$ (71), while the following
number $1'$ as transform of the number 1 contained
in $N$ is element of $N'$.

$\sigma$. If the theorem is true for a number $n$ and we
put the following number $n' = p$, then is $n$ different
from $p$, whence by (26) on account of the similarity
(71) of the order-setting transformation $\phi$ it follows
that $n'$, and therefore $p$, is different from $p'$. Hence
the theorem holds also for the number $p$ following $n$,
which was to be proved.

\mypara{82.} Theorem. In the transform-chain $n'_0$  of a number
$n$ by (74), (75) is contained its transform $n'$, but
not the number $n$ itself.

Proof by complete induction (80):

$\rho$. The theorem is true for $n = 1$, because $1'_0 = N'$,
% [File: 076.png]
and because by (71) the base-number 1 is not contained
in $N'$.

$\sigma$. If the theorem is true for a number $n$, and we
again put $n' =p$, then is $n$ not contained in $p_0$, therefore
is it different from every number $q$ contained in
$p_0$, whence by reason of the similarity of $\phi$ it follows
that $n'$, and therefore $p$, is different from every number
$q'$ contained in $p'_0$, and is hence not contained in
$p'_0$. Therefore the theorem holds also for the number
$p$ following $n$, which was to be proved.

\mypara{83.} Theorem. The transform-chain $n'_0$ is proper
part of the chain $n_0$.

The proof follows from (76), (74), (82).

\mypara{84.} Theorem. From $m_0= n_0$ it follows that $m=n$.

Proof.  Since by (74) $m$ is contained in $m_0$, and
\[
m_0= n_0 = \mathfrak{M}(n, n'_0)
\]
by (77), then if the theorem were false and hence $m$
different from $n$, $m$ would be contained in the chain
$n'_0$, hence by (74) also $m_0 \partof n'_0$, i.~e., $n_0 \partof n'_0$; but this
contradicts theorem (83). Hence our theorem is established.

\mypara{85.} Theorem. If the number $n$ is not contained
in the number-chain $K$, then is $K \partof n'_0$.

Proof by complete induction (80):

$\rho$. By (78) the theorem is true for $n=1$.

$\sigma$. If the theorem is true for a number $n$, then is
it also true for the following number, $p= n'$; for if $p$
is not contained in the number-chain $K$, then by (40)
$n$ also cannot be contained in $K$ and hence by our
% [File: 077.png]
hypothesis $K \partof n'_0$; now since by (77) $n'_0 = p_0 =
\mathfrak{M}(p,p'_0)$, hence $K \partof \mathfrak{M}(p,p'_0)$ and $p$ is not contained
in $K$, then must $K \partof p'_0$, which was to be proved.

\mypara{86.} Theorem. If the number $n$ is not contained
in the number-chain $K$, but its transform $n'$ is, then
$K=n'_0$.

Proof. Since $n$ is not contained in $K$, then by
(85) $K \partof n'_0$, and since $n' \partof K$, then by (47) is also
$n'_0 \partof K$, and hence $K=n'_0$, which was to be proved.

\mypara{87.} Theorem. In every number-chain $K$ there exists
one, and by (84) only one, number $k$, whose chain
$k_0 = K$.

Proof. If the base-number 1 is contained in $K$,
then by (79) $K=N=1_0$. In the opposite case let $Z$
be the system of all numbers not contained in $K$;
since the base-number 1 is contained in $Z$, but $Z$ is
only a proper part of the number-series $N$, then by
(79) $Z$ cannot be a chain, i.~e., $Z'$ cannot be part of
$Z$; hence there exists in $Z$ a number $n$, whose transform
$n'$ is not contained in $Z$, and is therefore certainly
contained in $K$; since further $n$ is contained in $Z$, and
therefore not in $K$, then by (86) $K=n'_0$, and hence
$k = n'$, which was to be proved.

\mypara{88.} Theorem. If $m$, $n$ are different numbers then
by (83), (84) one and only one of the chains $m_0$, $n_0$ is
proper part of the other and either $n_0 \partof m'_0$ or $m_0 \partof n'_0$.

Proof. If $n$ is contained in $m_0$, and hence by (74)
also $n_0 \partof m_0$, then $m$ can not be contained in the chain $n_0$
(because otherwise by (74) we should have $m_0 \partof n_0$,
% [File: 078.png]
therefore $m_0 = n_0$, and hence by (84) also $m = n$) and
thence it follows by (85) that $n_0 \partof m'_0$. In the contrary
case, when $n$ is not contained in the chain $m_0$, we must
have by (85) $m_0 \partof n'_0$, which was to be proved.

\mypara{89.} Definition. The number $m$ is said to be \textit{less}
than the number $n$ and at the same time $n$ \textit{greater} than
$m$, in symbols
\[
m < n, \quad n >m,
\]
when the condition
\[
n_0 \partof m'_0
\]
is fulfilled, which by (74) may also be expressed
\[
n \partof m'_0.
\]

\mypara{90.} Theorem. If $m$, $n$ are any numbers, then always
one and only one of the following cases $\lambda$, $\mu$, $\nu$
occurs:
\begin{eqnarray*}
\lambda.& m = n,\ n = m,\ \mathrm{i.~e.}, &m_0 = n_0\\
\mu.    & m < n,\ n > m,\ \mathrm{i.~e.}, &n_0 \partof m'_0\\
\nu.    & m > n,\ n < m,\ \mathrm{i.~e.}, &m_0 \partof n'_0.
\end{eqnarray*}

Proof. For if $\lambda$ occurs (84) then can neither $\mu$
nor $\nu$ occur because by (83) we never have $n_0 \partof n'_0$. But
if $\lambda$ does not occur then by (88) one and only one of
the cases $\mu$, $\nu$ occurs, which was to be proved.

\mypara{91.} Theorem. $n<n'$.

Proof. For the condition for the case $\nu$ in (90) is
fulfilled by $m = n'$.

\mypara{92.} Definition.  To express that $m$ is either $=n$
or $< n$, hence not $> n$ (90) we use the symbols
\[
m\leqq n \mathrm{\ or\ also\ } n\geqq m
\]
% [File: 079.png]
and we say $m$ is \textit{at most equal} to $n$, and $n$ is \textit{at least
equal} to $m$.

\mypara{93.} Theorem.  Each of the conditions
\[
m \leqq n,\quad m<n',\quad n_0 \partof m_0
\]
is equivalent to each of the others.

Proof. For if $m \leqq n$, then from $\lambda, \mu$ in (90) we
always have $n_0 \partof m_0$, because by (76) $m'_0 \partof m$. Conversely,
if $n_0 \partof m_0$, and therefore by (74) also $n \partof m_0$, it follows
from $m_0 = \mathfrak{M}(m, m'_0)$ that either $n = m$, or $n \partof m'_0$,
i.~e., $n>m$. Hence the condition $m \leqq n$ is equivalent
to $n_0 \partof m_0$. Besides it follows from (22), (27), (75)
that this condition $n_0 \partof m_0$ is again equivalent to $n'_0 \partof m'_0$,
i.~e., by $\mu$ in (90) to $m<n'$, which was to be proved.

\mypara{94.} Theorem. Each of the conditions
\[
m' \leqq n,\quad m'<n',\quad m<n
\]
is equivalent to each of the others.

The proof follows immediately from (93), if we
replace in it $m$ by $m'$, and from $\mu$ in (90).

\mypara{95.} Theorem. If $l<m$ and $m\leqq n$ or if $l \leqq m$, and
$m<n$, then is $l<n$.  But if $l \leqq m$ and $m \leqq n$, then is
$l \leqq n$.

Proof. For from the corresponding conditions
(89), (93) $m_0 \partof l'_0$ and $n_0 \partof m_0$, we have by (7) $n_0 \partof l'_0$ and
the same thing comes also from the conditions $m_0 \partof l_0$
and $n_0 \partof m'_0$, because in consequence of the former we
have also $m'_0 \partof l'_0$. Finally from $m_0 \partof l_0$ and $n_0 \partof m_0$ we
have also $n_0 \partof l_0$ which was to be proved.

\mypara{96.} Theorem. In every part $T$ of $N$ there exists
one and only one \textit{least} number $k$, i.~e., a number $k$
% [File: 080.png]
which is less than every other number contained in
$T$. If $T$ consists of a single number, then is it also
the least number in $T$.

Proof. Since $T_0$ is a chain (44), then by (87) there
exists one number $k$ whose chain $k_0=T_0$. Since from
this it follows by (45), (77) that $T\partof\mathfrak{M}(k, k'_0)$, then
first must $k$ itself be contained in $T$ (because otherwise
$T\partof k'_0$, hence by (47) also $T_0\partof k'_0$, i.~e., $k\partof k'_0$,
which by (83) is impossible), and besides every number
of the system $T$, different from $k$, must be contained
in $k'_0$, i.~e., be $>k$ (89), whence at once from
(90) it follows that there exists in $T$ one and only one
least number, which was to be proved.

\mypara{97.} Theorem. The least number of the chain $n_0$ is
$n$, and the base-number 1 is the least of all numbers.

Proof. For by (74), (93) the condition $m\partof n_0$ is
equivalent to $m\geqq n$. Or our theorem also follows immediately
from the proof of the preceding theorem,
because if in that we assume $T=n_0$, evidently $k=n$
(51).

\mypara{98.} Definition. If $n$ is any number, then will we
denote by $Z_n$ the system of all numbers that are \emph{not
greater} than $n$, and hence \emph{not} contained in $n'_0$. The
condition
\[
m\partof Z_n
\]
by (92), (93) is obviously equivalent to each of the
following conditions:
\[
m\leqq n,\quad m<n',\quad n_0\partof m_0.
\]

\mypara{99.} Theorem. $1\partof Z_n$ and $n\partof Z_n$.

% [File: 081.png]

The proof follows from (98) or from (71) and (82).

\mypara{100.} Theorem. Each of the conditions equivalent
by (98)
\[
m\partof Z_n,\ m\leqq n,\ m<n',\ n_0\partof m_0
\]
is also equivalent to the condition
\[
Z_m\partof Z_n.
\]

Proof. For if $m\partof Z_n$, and hence $m\leqq n$, and if $l\partof Z_m$,
and hence $l\leqq m$, then by (95) also $l\leqq n$, i.~e., $l\partof Z_n$; if
therefore $m\partof Z_n$, then is every element $l$ of the system
$Z_m$ also element of $Z_n$, i.~e., $Z_m\partof Z_n$. Conversely, if
$Z_m\partof Z_n$, then by (7) must also $m\partof Z_n$, because by (99)
$m\partof Z_m$, which was to be proved.

\mypara{101.} Theorem. The conditions for the cases $\lambda$, $\mu$,
$\nu$ in (90) may also be put in the following form:
\begin{eqnarray*}
\lambda. & \quad m=n,\quad n=m,\quad Z_m=Z_n\\
\mu. & \quad m<n,\quad n>m,\quad Z_{m'}\partof Z_n\\
\nu. & \quad m>n,\quad n<m,\quad Z_{n'}\partof Z_m
\end{eqnarray*}

The proof follows immediately from (90) if we observe
that by (100) the conditions $n_0\partof m_0$ and $Z_m\partof Z_n$ are
equivalent.

\mypara{102.} Theorem. $Z_1=1$.

Proof. For by (99) the base-number 1 is contained
in $Z_1$, while by (78) every number different
from $1$ is contained in $1'_0$, hence by (98) not in $Z_1$,
which was to be proved.

\mypara{103.} Theorem. By (98) $N=\mathfrak{M}(Z_n, n'_0)$.

\mypara{104.} Theorem. $n=\mathfrak{G}(Z_n, n_0)$, i.~e., $n$ is the only
common element of the system $Z_n$ and $n_0$.

Proof. From (99) and (74) it follows that $n$ is
% [File: 082.png]
contained in $Z_n$ and $n_0$; but every element of the chain
$n_0$, different from $n$ by (77) is contained in $n'_0$, and hence
by (98) not in $Z_n$, which was to be proved.

\mypara{105.} Theorem. By (91), (98) the number $n'$ is not
contained in $Z_n$.

\mypara{106.} Theorem. If $m<n$, then is $Z_m$ proper part
of $Z_n$ and conversely.

Proof. If $m<n$, then by (100) $Z_m \partof Z_n$, and since
the number $n$, by (99) contained in $Z_n$, can by (98)
not be contained in $Z_m$ because $n>m$, therefore $Z_m$ is
proper part of $Z_n$. Conversely if $Z_m$ is proper part of
$Z_n$ then by (100) $m\leqq n$, and since $m$ cannot be $=n$,
because otherwise $Z_m=Z_n$, we must have $m<n$, which
was to be proved.

\mypara{107.} Theorem. $Z_n$ is proper part of $Z_{n'}$.

The proof follows from (106), because by (91)
$n<n'$.

\mypara{108.} Theorem. $Z_{n'}= \mathfrak{M}(Z_n,n')$.

Proof. For every number contained in $Z_{n'}$, by (98)
is $\leqq n'$, hence either $=n'$ or $<n'$, and therefore by (98)
element of $Z_n$. Therefore certainly $Z_{n'} \partof \mathfrak{M}(Z_n,n')$.
Since conversely by (107) $Z_n \partof Z_{n'}$ and by (99) $n' \partof Z_{n'}$,
then by (10) we have
\[
\mathfrak{M}(Z_n,n') \partof Z_{n'},
\]
whence our theorem follows by (5).

\mypara{109.} Theorem. The transform $Z'_n$ of the system
$Z_n$ is proper part of the system $Z_{n'}$.

Proof. For every number contained in $Z'_n$ is the
transform $m'$ of a number $m$ contained in $Z_n$, and since
% [File: 083.png]
$m \leqq n$, and hence by (94) $m' \leqq n'$, we have by (98)
$Z'_n \partof Z_{n'}$. Since further the number 1 by (99) is contained
in $Z_{n'}$, but by (71) is not contained in the transform
$Z'_n$, then is $Z'_n$ proper part of $Z_{n'}$, which was to
be proved.

\mypara{110.} Theorem. $Z_{n'} = \mathfrak{M}(1, Z'_n)$.

Proof. Every number of the system $Z_{n'}$ different
from 1 by (78) is the transform $m'$ of a number $m$ and
this must be $\leqq n$, and hence by (98) contained in $Z_n$
(because otherwise $m>n$, hence by (94) also $m'>n'$
and consequently by (98) $m'$ would not be contained
in $Z_{n'}$); but from $m \partof Z_n$ we have $m' \partof Z'_n$, and hence
certainly
\[
Z_{n'} \partof \mathfrak{M}(1, Z'_n).
\]
Since conversely by (99) $1 \partof Z_n$, and by (109) $Z'_n \partof Z_{n'}$,
then by (10) we have $\mathfrak{M}(1, Z'_n) \partof Z_{n'}$ and hence our
theorem follows by (5).

\mypara{111.} Definition.  If in a system $E$ of numbers
there exists an element $g$, which is greater than every
other number contained in $E$, then $g$ is said to be the
\textit{greatest} number of the system $E$, and by (90) there can
evidently be only one such greatest number in $E$. If
a system consists of a single number, then is this number
itself the greatest number of the system.

\mypara{112.} Theorem.  By (98) $n$ is the greatest number
of the system $Z_n$.

\mypara{113.} Theorem. If there exists in $E$ a greatest
number $g$, then is $E \partof Z_g$.

Proof.  For every number contained in $E$ is $\leqq g$,
% [File: 084.png]
and hence by (98) contained in $Z_g$, which was to be
proved.

\mypara{114.} Theorem.  If $E$ is part of a system $Z_n$, or
what amounts to the same thing, there exists a number
$n$ such that all numbers contained in $E$ are $\leqq n$,
then $E$ possesses a greatest number $g$.

Proof. The system of all numbers $p$ satisfying
the condition $E \partof Z_p$---and by our hypothesis such
numbers exist---is a chain (37), because by (107),
(7) it follows also that $E \partof Z_{p'}$, and hence by (87) $=g_0$,
where $g$ signifies the least of these numbers (96), (97).
Hence also $E \partof Z_g$, therefore by (98) every number contained
in $E$ is $\leqq g$, and we have only to show that the
number $g$ is itself contained in $E$. This is immediately
obvious if $g=1$, for then by (102) $Z_g$, and consequently
also $E$ consists of the single number 1. But if $g$ is
different from 1 and consequently by (78) the transform
$f'$ of a number $f$, then by (108) is $E \partof \mathfrak{M}(Z_f,g)$;
if therefore $g$ were not contained in $E$, then would
$E \partof Z_f$, and there would consequently be among the
numbers $p$ a number $f$ by (91) $<g$, which is contrary
to what precedes; hence $g$ is contained in $E$, which
was to be proved.

\mypara{115.} Definition.  If $l<m$ and $m<n$ we say the
number $m$ \textit{lies between} $l$ and $n$ (also between $n$ and $l$).

\mypara{116.} Theorem.  There exists no number lying between
$n$ and $n'$.

Proof.  For as soon as $m < n'$, and hence by (93)
% [File: 085.png]
$m\leqq n$, then by (90) we cannot have $n<m$, which was
to be proved.

\mypara{117.} Theorem. If $t$ is a number in $T$, but not the
least (96), then there exists in $T$ one and only one
\textit{next less} number $s$, i.~e., a number $s$ such that $s<t$,
and that there exists in $T$ no number lying between $s$
and $t$. Similarly, if $t$ is not the greatest number in $T$
(111) there always exists in $T$ one and only one \textit{next
greater} number $u$, i.~e., a number $u$ such that $t<u$,
and that there exists in $T$ no number lying between $t$
and $u$. At the same time in $T$ $t$ is next greater than $s$
and next less than $u$.

Proof. If $t$ is not the least number in $T$, then let
$E$ be the system of all those numbers of $T$ that are
$<t$; then by (98) $E \partof Z_t$, and hence by (114) there
exists in $E$ a greatest number $s$ obviously possessing
the properties stated in the theorem, and also it is the
only such number. If further $t$ is not the greatest
number in $T$, then by (96) there certainly exists among
all the numbers of $T$, that are $>t$, a least number $u$,
which and which alone possesses the properties stated
in the theorem. In like manner the correctness of the
last part of the theorem is obvious.

\mypara{118.} Theorem. In $N$ the number $n'$ is next greater
than $n$, and $n$ next less than $n'$.

The proof follows from (116), (117).
% [File: 086.png]

%VIII.

\mysect{X}{VIII}{Finite and Infinite Parts of the Number-Series}

\mypara{119.} Theorem. Every system $Z_n$ in (98) is finite.

Proof by complete induction (80).

$\rho$. By (65), (102) the theorem is true for $n=1$.

$\sigma$. If $Z_n$ is finite, then from (108) and (70) it follows
that $Z_{n'}$ is also finite, which was to be proved.

\mypara{120.} Theorem. If $m$, $n$ are different numbers, then
are $Z_m$, $Z_n$ dissimilar systems.

Proof. By reason of the symmetry we may by
(90) assume that $m<n$; then by (106) $Z_m$ is proper
part of $Z_n$, and since by (119) $Z_n$ is finite, then by (64)
$Z_m$ and $Z_n$ cannot be similar, which was to be proved.

\mypara{121.} Theorem. Every part $E$ of the number-series
$N$, which possesses a greatest number (111), is
finite.

The proof follows from (113), (119), (68).

\mypara{122.} Theorem. Every part $U$ of the number-series
$N$, which possesses no greatest number, is simply infinite
(71).

Proof. If $u$ is any number in $U$, there exists in $U$
by (117) one and only one next greater number than
$u$, which we will denote by $\psi(u)$ and regard as transform
of $u$. The thus perfectly determined transformation
$\psi$ of the system $U$ has obviously the property
%\[
%  \alpha.\quad \psi(U) \partof U,
%\]
\\ \null\hfill
$\alpha.\quad \psi(U) \partof U$,
\hfill\null\\
i.~e., $U$ transformed in itself by $\psi$. If further $u$, $v$
% [File: 087.png]
are different numbers in $U$, then by symmetry we may
by (90) assume that $u<v$; thus by (117) it follows
from the definition of $\psi$ that $\psi(u) \leqq v$ and $v < \psi(v)$,
and hence by (95) $\psi(u) < \psi(v)$; therefore by (90) the
transforms $\psi(u)$, $\psi(v)$ are different, i.~e.,
%\begin{quote}
%[**F2: Odd bits. {quote} and/or {center} puts whitespace above and below the lines which isn't in the image.]
\\ \null\hfill
$\delta$. the transformation $\psi$ is similar.
\hfill\null\\
%\end{quote}
Further, if $u_1$ denotes the least number (96) of the
system $U$, then every number $u$ contained in $U$ is
$\geqq u_1$, and since generally $u < \psi(u)$, then by (95) $u_1<\psi(u)$,
and therefore by (90) $u_1$ is different from $\psi(u)$,
i.~e.,
\\ \null\hfill
%\begin{quote}
$\gamma$. the element $u_1$ of $U$ is not contained in $\psi(u)$.
%\end{quote}
\hfill\null\\
Therefore $\psi(U)$ is proper part of $U$ and hence by (64)
$U$ is an infinite system. If then in agreement with
(44) we denote by $\psi_0(V)$, when $V$ is any part of $U$,
the chain of $V$ corresponding to the transformation $\psi$,
we wish to show finally that
\\ \null\hfill
%\begin{quote}
$\beta$. $U=\psi_0(u_1)$.
%\end{quote}
\hfill\null\\
In fact, since every such chain $\psi_0(V)$ by reason of its
definition (44) is a part of the system $U$ transformed
in itself by $\psi$, then evidently is $\psi_0(u_1) \partof U$; conversely
it is first of all obvious from (45) that the element $u_1$
contained in $U$ is certainly contained in $\psi_0(u_1)$; but
if we assume that there exist elements of $U$, that
are not contained in $\psi_0(u_1)$, then must there be among
them by (96) a least number $w$, and since by what
precedes this is different from the least number $u_1$ of
the system $U$, then by (117) must there exist in $U$
also a number $v$ which is next less than $w$, whence it
% [File: 088.png]
follows at once that $w = \phi(v)$; since therefore $v<w$,
then must $v$ by reason of the definition of $w$ certainly
be contained in $\psi_0(u_1)$; but from this by (55) it follows
that also  $\psi(v)$, and hence $w$ must be contained
in $\psi_0(u_1)$, and since this is contrary to the definition of
$w$, our foregoing hypothesis is inadmissible; therefore
$U \partof \psi_0(u_1)$ and hence also $U=\psi_0(u_1)$ as stated. From
$\alpha$, $\beta$, $\gamma$, $\delta$ it then follows by (71) that $U$ is a simply infinite
system set in order by $\psi$, which was to be proved.

\mypara{123.} Theorem.  In consequence of (121), (122)
any part $T$ of the number-series $N$ is finite or simply
infinite, according as a greatest number exists or does
not exist in $T$.


%Observed to be close to bottom of page so force skip
\newpage
%IX.

\mysect{VII}{IX}{Definition of a Transformation of the Number-Series by Induction}

\mypara{124.} In what follows we denote numbers by small
Italics and retain throughout all symbols of the previous
sections VI.\ to VIII., while $\Omega$ designates an
arbitrary system whose elements are not necessarily
contained in $N$.

\mypara{125.} Theorem. If there is given an arbitrary (similar
or dissimilar) transformation $\theta$ of a system $\Omega$ in
itself, and besides a determinate element $\omega$ in $\Omega$, then
to every number $n$ corresponds one transformation
$\psi_n$ and one only of the associated number-system $Z_n$
explained in (98), which satisfies the conditions:\footnote
  {For clearness here and in the following theorem (126) I have especially
mentioned condition I., although properly it is a consequence of II.\ and III.}

% [File: 089.png]
\par\begin{tabular}[b]{@{}r@{}l@{}}
  &I\@. $\psi_n{Z_n} \partof \Omega$ \\
 I&I\@. $\psi_n(1)=\omega$ \\
II&I\@. $\psi_n(t')=\theta\psi_n(t)$,
\end{tabular}
if $t<n$, where the symbol
$\theta\psi_n$ has the meaning given in (25).

Proof by complete induction (80).

$\rho$. \label{psi1}The theorem is true for $n = 1$. In this case indeed
by (102) the system $Z_{n}$ consists of the single
number 1, and the transformation $\psi_1$ %[**Corrected from comma, noted at end.]
is therefore completely
defined by II alone so that I is fulfilled while
III drops out entirely.

$\sigma$. If the theorem is true for a number $n$ then we
show that it is also true for the following number
$p=n'$, and we begin by proving that there can be only
a single corresponding transformation $\psi_p$ of the system
$Z_p$. In fact, if a transformation $\psi_p$ satisfies the
conditions
\par\begin{tabular}[b]{@{}r@{}l@{}}
  &I$'$. $\psi_p(Z_p) \partof \Omega$ \\
 I&I$'$. $\psi_p(1)=\omega$\\
II&I$'$. $\psi_p(m')=\theta\psi_p(m)$,
\end{tabular}
when $m<p$, then there is
also contained in it by (21), because $Z_n \partof Z_p$ (107) a
transformation of $Z_{n}$ which obviously satisfies the
same conditions I, II, III as $\psi_n$, and therefore coincides
throughout with $\psi_n$; for all numbers contained
in $Z_{n}$, and hence (98) for all numbers $m$ which are
$<p$, i.~e., $\leqq n$, must therefore
\[
\psi_p(m)=\psi_n(m) \tag{$m$}
\]
whence there follows, as a special case,
\[
\psi_p(n)=\psi_n(n); \tag{$n$}
\]
since further by (105), (108) $p$ is the only number of
% [File: 090.png]
the system $Z_p$ not contained in $Z_n$, and since by III$'$
and $(n)$ we must also have
\[
\psi_p(p) = \theta\psi_n(n) \tag{$p$}
\]
there follows the correctness of our foregoing statement
that there can be only one transformation $\psi_p$ of
the system $Z_p$ satisfying the conditions I$'$, II$'$, III$'$,
because by the conditions $(m)$ and $(p)$ just derived
$\psi_p$ is completely reduced to $\psi_n$. We have next to show
conversely that this transformation $\psi_p$ of the system
$Z_p$ completely determined by $(m)$ and $(p)$ actually
satisfies the conditions I$'$, II$'$, III$'$. Obviously I$'$ follows
from $(m)$ and $(p)$ with reference to I, and because
$\theta(\Omega) \partof \Omega$. Similarly II$'$ follows from $(m)$ and II, since
by (99) the number 1 is contained in $Z_n$. The correctness
of III$'$ follows first for those numbers $m$ which
are $<n$ from $(m)$ and III, and for the single number
$m=n$ yet remaining it results from $(p)$ and $(n)$. Thus
it is completely established that from the validity of
our theorem for the number $n$ always follows its validity
for the following number $p$, which was to be proved.

\mypara{126.} Theorem of the definition by induction. If
there is given an arbitrary (similar or dissimilar) transformation
$\theta$ of a system $\Omega$ in itself, and besides a determinate
element $\omega$ in $\Omega$, then there exists one and
only one transformation $\psi$ of the number-series $N$,
which satisfies the conditions
\par\begin{tabular}[b]{@{}r@{}l@{}}
 &I\@. $\psi(N) \partof \Omega$\\
I&I\@. $\psi(1)=\omega$\\
% [File: 091.png]
II&I\@. $\psi(n') = \theta\psi(n)$,
\end{tabular}
where $n$ represents every number.

Proof. Since, if there actually exists such a transformation
$\psi$, there is contained in it by (21) a transformation
$\psi_n$, of the system $Z_n$, which satisfies the conditions
I, II, III stated in (125), then because there
exists one and only one such transformation $\psi_n$ must
necessarily
\[
\psi(n) =\psi_n(n). \tag{$n$}
\]
Since thus $\psi$ is completely determined it follows also
that there can exist only one such transformation $\psi$
(see the closing remark in (130)). That conversely
the transformation $\psi$ determined by $(n)$ also satisfies
our conditions I, II, III, follows easily from $(n)$ with
reference to the properties I, II and $(p)$ shown in (125),
which was to be proved.

\mypara{127.} Theorem. Under the hypotheses made in the
foregoing theorem,
\[
\psi(T') = \theta\psi(T),
\]
where $T$ denotes any part of the number-series $N$.

Proof. For if $t$ denotes every number of the system
$T$, then $\psi(T')$ consists of all elements $\psi(t')$, and
$\theta\psi(T)$ of all elements $\theta\psi(t)$; hence our theorem follows
because by III in (126) $\psi(t') = \theta\psi(t)$.

\mypara{128.} Theorem. If we maintain the same hypotheses
and denote by $\theta_{0}$ the chains (44) which correspond
to the transformation $\theta$ of the system $\Omega$ in itself,
then is
\[
\psi(N) = \theta_0(\omega).
\]

% [File: 092.png]

Proof.  We show first by complete induction (80)
that
\[
\psi(N) \partof \theta_0(\omega),
\]
i.~e., that every transform $\psi(n)$ is also element of
$\theta_0(\omega)$. In fact,

$\rho$. this theorem is true for $n = 1$, because by (126,
II) $\psi(1)=\omega$, and because by (45) $\omega \partof \theta_0(\omega)$.

$\sigma$. If the theorem is true for a number $n$, and hence
$\psi(n) \partof \theta_0(\omega)$, then by (55) also $\theta\bigl(\psi(n)\bigr) \partof \theta_0(\omega)$, i.~e., by
(126, III) $\psi(n') \partof \theta_0(\omega)$, hence the theorem is true for
the following number $n'$, which was to be proved.

In order further to show that every element $\nu$ of
the chain $\theta_0(\omega)$ is contained in $\psi(N)$, therefore that
\[
\theta_0(\omega) \partof \psi(N)
\]
we likewise apply complete induction, i.~e., theorem
(59) transferred to $\Omega$ and the transformation $\theta$. In
fact,

$\rho$. the element $\omega=\psi(1)$, and hence is contained in
$\psi(N)$.

$\sigma$. If $\nu$ is a common element of the chain $\theta_0(\omega)$
and the system $\psi(N)$, then $\nu=\psi(n)$, where $n$ denotes
a number, and by (126, III) we get $\theta(\nu) = \theta\psi(n) = \psi(n')$,
and therefore $\theta(\nu)$ is contained in $\psi(N)$, which
was to be proved.

From the theorems just established, $\psi(N) \partof \theta_0(\omega)$
and $\theta_0(\omega) \partof \psi(N)$, we get by (5) $\psi(N) = \theta_0(\omega)$, which
was to be proved.

\mypara{129.} Theorem. Under the same hypotheses we
have generally:
% [File: 093.png]
\[
\psi(n_0) = \theta_0\bigl(\psi(n)\bigr).
\]

Proof by complete induction (80). For

$\rho$. By (128) the theorem holds for $n = 1$, since
$1_0=N$ and $\psi(1)=\omega$.

$\sigma$. If the theorem is true for a number $n$, then
\[
  \theta\bigl(\psi(n_0)\bigr)
= \theta\bigl(\theta_0\bigl(\psi(n)\bigr)\bigr);
\]
since by (127), (75)
\[
\theta\bigl(\psi(n_0)\bigr) = \psi(n'_0),
\]
and by (57), (126, III)
\[
  \theta\bigl(\theta_0\bigl(\psi(n)\bigr)\bigr)
= \theta_0\bigl(\theta\bigl(\psi(n)\bigr)\bigr)
= \theta_0\bigl(\psi(n')\bigr),
\]
\rlap{we get}
\hfill$
\psi(n'_0) = \theta_0\bigl(\psi(n')\bigr)$,
\hfill\null\\
i.~e., the theorem is true for the number $n'$ following
$n$, which was to be proved.

\mypara{130.} Remark. Before we pass to the most important
applications of the theorem of definition by induction
proved in (126), (sections X--XIV), it is worth
while to call attention to a circumstance by which it
is essentially distinguished from the theorem of demonstration
by induction proved in (80) or rather in
(59), (60), however close may seem the relation between
the former and the latter. For while the theorem
(59) is true quite generally for every chain $A_0$ where
$A$ is any part of a system $S$ transformed in itself by
any transformation $\phi$ (IV), the case is quite different
with the theorem (126), which declares only the existence
of a consistent (or one-to-one) transformation $\psi$
of the simply infinite system $1_0$. If in the latter theorem
(still maintaining the hypotheses regarding $\Omega$
and $\theta$) we replace the number-series $1_0$ by an arbitrary
% [File: 094.png]
chain $A_0$ out of such a system $S$, and define a transformation
$\psi$ of $A_0$ in $\Omega$ in a manner analogous to that
in (126, II, III) by assuming that

$\rho$. to every element $a$ of $A$ there is to correspond a
determinate element $\psi(a)$ selected from $\Omega$, and

$\sigma$. for every element $n$ contained in $A_0$ and its
transform $n'=\phi(n)$, the condition $\psi(n')=\theta \psi(n)$ is to
hold, then would the case very frequently occur that
such a transformation $\psi$ does not exist, since these conditions
$\rho$, $\sigma$ may prove incompatible, even though the
freedom of choice contained in $\rho$ be restricted at the
outset to conform to the condition $\sigma$. An example will
be sufficient to convince one of this. If the system $S$
consisting of the different elements $a$ and $b$ is so transformed
in itself by $\phi$ that $a'=b$, $b'=a$, then obviously
$a_0=b_0=S$; suppose further the system $\Omega$ consisting of
the different elements $\alpha$, $\beta$, and $\gamma$ be so transformed in
itself by $\theta$ that $\theta(\alpha)=\beta$, $\theta(\beta)=\gamma$, $\theta(\gamma)=\alpha$; if we
now demand a transformation $\psi$ of $a_0$ in $\Omega$ such that
$\psi(a)=\alpha$, and that besides for every element $n$ contained
in $a_0$ always $\psi(n')=\theta\psi(n)$, we meet a contradiction;
since for $n=a$, we get $\psi(b)=\theta(\alpha)=\beta$, and
hence for $n=b$, we must have $\psi(a)=\theta(\beta)=\gamma$, while
we had assumed $\psi(a)=\alpha$.

But if there exists a transformation $\psi$ of $A_0$ in $\Omega$,
which satisfies the foregoing conditions $\rho$, $\sigma$ without
contradiction, then from (60) it follows easily that it
is completely determined; for if the transformation $\chi$
satisfies the same conditions, then we have, generally,
% [File: 095.png]
$\chi(n) = \psi(n)$, since by $\rho$ this theorem is true for all elements
$n=a$ contained in $A$, and since if it is true
for an element $n$ of $A_0$ it must by $\sigma$ be true also for its
transform $n'$.

\mypara{131.} In order to bring out clearly the import of
our theorem (126), we will here insert a consideration
which is useful for other investigations also, e.~g., for
the so-called group-theory.

We consider a system $\Omega$, whose elements allow a
certain combination such that from an element $\nu$ by
the effect of an element $\omega$, there always results again a
determinate element of the same system $\Omega$, which may
be denoted by $\omega\centerdot \nu$  or $\omega\nu$, and in general is to be distinguished
from $\nu\omega$. We can also consider this in
such a way that to every determinate element $\omega$, there
corresponds a determinate transformation of the system
$\Omega$ in itself (to be denoted by $\dot{\omega}$), in so far as every
element $\nu$ furnishes the determinate transform $\dot{\omega}(\nu)=
\omega \nu$.  If to this system $\Omega$ and its element $\omega$ we apply
theorem (126), designating by $\dot{\omega}$ the transformation
there denoted by $\theta$, then there corresponds to every
number $n$ a determinate element $\psi(n)$ contained in $\Omega$,
which may now be denoted by the symbol $\omega^n$ and sometimes
called the $n$th power of $\omega$; this notion is completely
defined by the conditions imposed upon it
\begin{align*}
\text{II\@. }& \omega^1 = \omega\\
\text{III\@. }& \omega^{n'}\! = \omega\, \omega^n,
\end{align*}
and its existence is established by the proof of theorem
(126).

% [File: 096.png]

If the foregoing combination of the elements is
further so qualified that for arbitrary elements $\mu$, $\nu$,
$\omega$, we always have $\omega(\nu\,\mu)$ = $\omega\,\nu(\mu)$, then are true also
the theorems
\[
  \omega^{n'}\! = \omega^n\, \omega, \quad
  \omega^m\, \omega^n = \omega^n\, \omega^m,
\]
whose proofs can easily be effected by complete induction
and may be left to the reader.

The foregoing general consideration may be immediately
applied to the following example. If $S$ is
a system of arbitrary elements, and $\Omega$ the associated
system whose elements are all the transformations $\nu$ of
$S$ in itself (36), then by (25) can these elements be continually
compounded, since $\nu(S) \partof S$, and the transformation
$\omega\nu$ compounded out of such transformations $\nu$
and $\omega$ is itself again an element of $\Omega$. Then are also
all elements $\omega^n$ transformations of $S$ in itself, and we
say they arise by repetition of the transformation $\omega$.
We will now call attention to a simple connection existing
between this notion and the notion of the chain
$\omega_0(A)$ defined in (44), where $A$ again denotes any part
of $S$. If for brevity we denote by $A_n$ the transform
$\omega^n(A)$ produced by the transformation $\omega^n$, then from
III and (25) it follows that $\omega(A_n)=A_{n'}$. Hence it is
easily shown by complete induction (80) that all these
systems $A_n$ are parts of the chain $\omega_0(A)$; for

$\rho$. by (50) this statement is true for $n = 1$, and

$\sigma$. if it is true for a number $n$, then from (55) and
from $A_{n'}=\omega(A_n)$ it follows that it is also true for the
following number $n'$, which was to be proved. Since
% [File: 097.png]
further by (45) $A \partof \omega_0(A)$, then from (10) it results that
the system $K$ compounded out of $A$ and all transforms
$A_n$ is part of $\omega_0(A)$. Conversely, since by (23) $\omega(K)$
is compounded out of $\omega(A)=A_1$ and all systems
$\omega(A_n)=A_{n'}$, therefore by (78) out of all systems $A_n$,
which by (9) are parts of $K$, then by (10) is $\omega(K) \partof K$,
i.~e., $K$ is a chain (37), and since by (9) $A \partof K$, then
by (47) it follows also that that $\omega_0(A) \partof K$. Therefore
$\omega_0(A)=K$, i.~e., the following theorem holds: If $\omega$ is a
transformation of a system $S$ in itself, and $A$ any part
of $S$, then is the chain of $A$ corresponding to the transformation
$\omega$ compounded out of $A$ and all the transforms
$\omega^n(A)$ resulting from repetitions of $\omega$. We advise
the reader with this conception of a chain to return
to the earlier theorems (57), (58).


%X.

\mysect{VIII}{X}{The Class of Simply Infinite Systems}

\mypara{132.} Theorem. All simply infinite systems are
similar to the number-series $N$ and consequently by
(33) also to one another.

Proof. Let the simply infinite system $\Omega$ be set in
order (71) by the transformation $\theta$, and let $\omega$ be the
base-element of $\Omega$ thus resulting; if we again denote
by $\theta_0$ the chains corresponding to the transformation
$\theta$ (44), then by (71) is the following true:

$\alpha$. $\theta(\Omega) \partof \Omega$.

$\beta$. $\Omega = \theta_0(\omega)$.

% [File: 098.png]

$\gamma$. $\omega$ is not contained in $\theta(\Omega)$.

$\delta$. The transformation $\theta$ is similar.

\noindent If then $\psi$ denotes the transformation of the number-series
$N$ defined in (126), then from $\beta$ and (128) we
get first
\[
\psi(N)=\Omega,
\]
and hence we have only yet to show that $\psi$ is a similar
transformation, i.~e., (26) that to different numbers
$m$, $n$ correspond different transforms $\psi(m)$, $\psi(n)$.
On account of the symmetry we may by (90) assume
that $m>n$, hence $m \partof n'_0$, and the theorem to prove
comes to this that $\psi(n)$ is not contained in $\psi(n'_0)$, and
hence by (127) is not contained in $\theta\psi(n_0)$. This we
establish for every number $n$ by complete induction
(80). In fact,

$\rho$. \label{Ind132}this theorem is true by $\gamma$ for $n = 1$, since $\psi(1)=\omega$
and $\psi(1_0)= \psi(N)=\Omega$.

$\sigma$. If the theorem is true for a number $n$, then is it
also true for the following number $n'$; for if $\psi(n')$,
i.~e., $\theta \psi(n)$, were contained in $\theta\psi(n'_0)$, then by $\delta$ and
(27), $\psi(n)$ would also be contained in $\psi(n'_0)$ while
our hypothesis states just the opposite; which was to
be proved.

\mypara{133.} Theorem. Every system which is similar to
a simply infinite system and therefore by (132), (33)
to the number-series $N$ is simply infinite.

Proof. If $\Omega$ is a system similar to the number-series
$N$, then by (32) there exists a similar transformation
$\psi$ of $N$ such that
% [File: 099.png]
\begin{align*}
& \text{I\@. } \psi(N) = \Omega;
\shortintertext{then we put }
& \text{II\@. } \psi(1) = \omega.
\end{align*}
If we denote, as in (26), by $\overline{\psi}$ the inverse, likewise
similar transformation of $\Omega$, then to every element $\nu$
of $\Omega$ there corresponds a determinate number $\overline{\psi}(\nu)=n$,
viz., that number whose transform $\psi(n) = \nu$. Since
to this number $n$ there corresponds a determinate following
number $\phi(n)=n'$, and to this again a determinate
element $\psi(n')$ in $\Omega$ there belongs to every element
$\nu$ of the system $\Omega$ a determinate element $\psi(n')$ of
that system which as transform of $\nu$ we shall designate
by $\theta(\nu)$. Thus a transformation $\theta$ of $\Omega$ in itself is completely
determined,\footnote
  {Evidently $\theta$ is the transformation $\psi\,\phi\,\overline{\psi}$
  compounded by (25) out of $\overline{\psi}$, $\phi$, $\psi$.}
and in order to prove our theorem
we will show that by $\theta \Omega$ is set in order (71) as a
simply infinite system, i.~e., that the conditions $\alpha$, $\beta$,
$\gamma$, $\delta$ stated in the proof of (132) are all fulfilled. First
$\alpha$ is immediately obvious from the definition of $\theta$.
Since further to every number $n$ corresponds an element
$\nu = \phi(n)$, for which $\theta(\nu) = \psi(n')$, we have generally,
\[
\mathrm{III.\ }\psi(n') = \theta\psi(n),
\]
and thence in connection with I, II, $\alpha$ it results that
the transformations $\theta$, $\psi$ fulfill all the conditions of
theorem (126); therefore $\beta$ follows from (128) and I\@.
Further by (127) and I
\[
\psi(N') = \theta \psi(N) = \theta(\Omega),
\]
and thence in combination with II and the similarity
% [File: 100.png]
of the transformation $\psi$ follows $\gamma$, because otherwise
$\psi(1)$ must be contained in $\psi(N')$, hence by (27) the
number 1 in $N'$, which by (71, $\gamma$) is not the case. If
finally $\mu$, $\nu$ denote elements of $\Omega$ and $m$, $n$ the corresponding
numbers whose transforms are $\psi(m) = \mu$,
$\psi(n) = \nu$, then from the hypothesis $\theta(\mu) = \theta(\nu)$ it follows
by the foregoing that $\psi(m') = \psi(n')$, thence on
account of the similarity of $\psi$, $\phi$ that $m' = n'$, $m = n$,
therefore also $\mu = \nu$; hence also $\delta$ is true, which was
to be proved.

\mypara{134.} Remark. By the two preceding theorems
(132), (133) all simply infinite systems form a class in
the sense of (34). At the same time, with reference to
(71), (73) it is clear that every theorem regarding
numbers, i.~e., regarding the elements $n$ of the simply
infinite system $N$ set in order by the transformation $\phi$,
and indeed every theorem in which we leave entirely
out of consideration the special character of the elements
$n$ and discuss only such notions as arise from
the arrangement $\phi$, possesses perfectly general validity
for every other simply infinite system $\Omega$ set in order by
a transformation $\theta$ and its elements $\nu$, and that the
passage from $N$ to $\Omega$ (e.~g., also the translation of an
arithmetic theorem from one language into another)
is effected by the transformation $\psi$ considered in
(132), (133), which changes every element $n$ of $N$ into
an element $\nu$ of $\Omega$, i.~e., into $\psi(n)$. This element $\nu$
can be called the $n$th element of $\Omega$ and accordingly
the number $n$ is itself the $n$th number of the number-series
% [File: 101.png]
$N$. The same significance which the transformation
$\phi$ possesses for the laws in the domain $N$, in
so far as every element $n$ is followed by a determinate
element $\phi(n)=n'$, is found, after the change effected
by $\psi$, to belong to the transformation $\theta$ for the same
laws in the domain $\Omega$, in so far as the element $\nu=\psi(n)$
arising from the change of $n$ is followed by the element
$\theta(\nu)=\psi(n')$ arising from the change of $n'$; we
are therefore justified in saying that by $\psi \phi$ is changed
into $\theta$, which is symbolically expressed by $\theta=\psi\phi\overline{\psi}$,
$\phi=\overline{\psi}\theta\psi$. By these remarks, as I believe, the definition
of the notion of numbers given in (73) is fully
justified. We now proceed to further applications of
theorem (126).


%XI.

\mysect{VII}{XI}{Addition of Numbers}

\mypara{135.} Definition. It is natural to apply the definition
set forth in theorem (126) of a transformation $\psi$
of the number-series $N$, or of the \textit{function} $\psi(n)$ determined
by it to the case, where the system there denoted
by $\Omega$ in which the transform $\psi(N)$ is to be contained,
is the number-series $N$ itself, because for this
system $\Omega$ a transformation $\theta$ of $\Omega$ in itself already exists,
viz., that transformation $\phi$ by which $N$ is set in
order as a simply infinite system (71), (73). Then is
also $\Omega=N$, $\theta(n)=\phi(n)=n'$, hence
\[
\mathrm{I.\ }\psi(N) \partof N,
\]
and it remains in order to determine $\psi$ completely
% [File: 102.png]
only to select the element $\omega$ from $\Omega$, i.~e., from $N$, at
pleasure. If we take $\omega=1$, then evidently $\psi$ becomes
the identical transformation (21) of $N$, because the
conditions
\[
\psi(1)=1,\quad \psi(n')=(\psi(n))'
\]
are generally satisfied by $\psi(n)=n$. If then we are to
produce another transformation $\psi$ of $N$, then for $\omega$ we
must select a number $m'$ different from 1, by (78) contained
in $N$, where $m$ itself denotes any number; since
the transformation $\psi$ is obviously dependent upon the
choice of this number $m$, we denote the corresponding
transform $\psi(n)$ of an arbitrary number $n$ by the
symbol $m+n$, and call this number the \textit{sum} which
arises from the number $m$ by the \textit{addition} of the number
$n$, or in short the sum of the numbers $m$, $n$.
Therefore by (126) this sum is completely determined
by the conditions\footnote{The above definition of addition based immediately upon theorem (126)
seems to me to be the simplest. By the aid of the notion developed in (131)
we can, however, define the sum $m+n$ by $\phi^n(m)$ or also by $\phi^m(n)$, where $\phi$ has
again the foregoing meaning. In order to show the complete agreement of
these definitions with the foregoing, we need by (126) only to show that if
$\phi^n(m)$ or $\phi^m(n)$ is denoted by $\psi(n)$, the conditions $\psi(1)=m'$, $\psi(n')=\phi\psi(n)$ are
fulfilled which is easily done with the aid of complete induction (80) by the
help of (131).}

\begin{align*}
\text{II\@. }& m+1=m',\\
\text{III\@. }& m+n'=(m+n)'.
\end{align*}

\mypara{136.} Theorem. $m'+n=m+n'$.

Proof by complete induction (80). For

$\rho$. the theorem is true for $n=1$, since by (135, II)
\[
m'+1=(m')'=(m+1)',
\]
and by (135, III) $(m+1)'=m+1'$.

% [File: 103.png]

$\sigma$. If the theorem is true for a number $n$, and we
put the following number $n'=p$ then is $m'+n=
m+p$, hence also $(m'+n)'=(m+p)'$, whence by (135,
III) $m'+p=m+p'$; therefore the theorem is true
also for the following number $p$, which was to be
proved.

\mypara{137.} Theorem. $m'+n=(m+n)'$.

The proof follows from (136) and (135, III).

\mypara{138.} Theorem. $1+n=n'$.

Proof by complete induction (80). For

$\rho$. by (135, II) the theorem is true for $n=1$.

$\sigma$. If the theorem is true for a number $n$ and we
put $n'=p$, then $1+n=p$, therefore also $(1+n)'=p'$,
whence by (135, III) $1+p=p'$, i.~e., the theorem is
true also for the following number $p$, which was to be
proved.

\mypara{139.} Theorem. $1+n=n+1$.

The proof follows from (138) and (135, II).

\mypara{140.} Theorem. $m+n=n+m$.

Proof by complete induction (80). For

$\rho$. by (139) the theorem is true for $n=1$.

$\sigma$. If the theorem is true for a number $n$, then there
follows also $(m+n)'=(n+m)'$, i.~e., by (135, III)
$m+n'=n+m'$, hence by (136) $m+n'=n'+m$; therefore
the theorem is also true for the following number
$n'$, which was to be proved.

\mypara{141.} Theorem. $(l+m)+n=l+(m+n)$.

Proof by complete induction (80). For
% [File: 104.png]

$\rho$. the theorem is true for $n=1$, because by (135,
II, III, II) $(l+m)+1=(l+m)'=l+m'=l+(m+1)$.

$\sigma$. If the theorem is true for a number $n$, then there
follows also $\bigl((l+m)+n\bigr)'=\bigl(l+(m+n)\bigr)'$, i.~e., by
(135, III)
\[
(l+m)+n'=l+(m+n)'=l+(m+n'),
\]
therefore the theorem is also true for the following
number $n'$, which was to be proved.

\mypara{142.} Theorem.  $m+n>m$.

Proof by complete induction (80). For

$\rho$. by (135, II) and (91) the theorem is true for
$n = 1$.

$\sigma$. If the theorem is true for a number $n$, then by
(95) it is also true for the following number $n'$, because
by (135, III) and (91)
\[
m+n'=(m+n)'>m+n,
\]
which was to be proved.

\mypara{143.} Theorem. The conditions $m>a$ and $m+n>
a+n$ are equivalent.

Proof by complete induction (80). For

$\rho$. by (135, II) and (94) the theorem is true for
$n = 1$.

$\sigma$. If the theorem is true for a number $n$, then is it
also true for the following number $n'$, since by (94)
the condition $m+n>a+n$ is equivalent to $(m+n)'>
(a+n)'$, hence by (135, III) also equivalent to
\[
m+n'> a+n',
\]
which was to be proved.

% [File: 105.png]

\mypara{144.} Theorem. If $m>a$ and $n>b$, then is also
\[
m+n>a+b.
\]

Proof. For from our hypotheses we have by (143)
$m+n>a+n$ and $n+a>b+a$ or, what by (140) is
the same, $a+n>a+b$, whence the theorem follows
by (95).

\mypara{145.} Theorem. If $m+n=a+n$ then $m=a$.

Proof. For if $m$ does not $=a$, hence by (90) either
$m>a$ or $m<a$, then by (143) respectively $m+n>
a+n$ or $m+n<a+n$, therefore by (90) we surely
cannot have $m+n=a+n$ which was to be proved.

\mypara{146.} Theorem. If $l>n$, then there exists one and
by (157) only one number $m$ which satisfies the condition
$m+n=l$.

Proof by complete induction (80). For

$\rho$. the theorem is true for $n = 1$. In fact, if $l>1$,
i.~e., (89) if $l$ is contained in $N'$, and hence is the
transform $m'$ of a number $m$, then by (135, II) it follows
that $l=m+1$, which was to be proved.

$\sigma$. If the theorem is true for a number $n$, then we
show that it is also true for the following number $n'$.
In fact, if $l>n'$, then by (91), (95) also $l>n$, and hence
there exists a number $k$ which satisfies the condition
$l=k+n$; since by (138) this is different from 1 (otherwise
$l$ would be $=n'$) then by (78) is it the transform
$m'$ of a number $m$, consequently $l=m'+n$, therefore
also by (136) $l=m+n'$, which was to be proved.

% [File: 106.png]


%XII.

\mysect{VI}{XII}{Multiplication of Numbers}

\mypara{147.} Definition. After having found in XI an infinite
system of new transformations of the number-series
$N$ in itself, we can by (126) use each of these
in order to produce new transformations $\psi$ of $N$.
When we take $\Omega = N$, and $\theta(n) = m+n = n+m$,
where $m$ is a determinate number, we certainly again
have
\[
\text{I\@. } \psi(N) \partof N,
\]
and it remains, to determine $\psi$ completely only to select
the element $\omega$ from $N$ at pleasure. The simplest
case occurs when we bring this choice into a certain
agreement with the choice of $\theta$, by putting $\omega = m$.
Since the thus perfectly determinate $\psi$ depends upon
this number $m$, we designate the corresponding transform
$\psi(n)$ of any number $n$ by the symbol $m \times n$ or
$m\centerdot n$ or $mn$, and call this number the \textit{product} arising
from the number $m$ by \textit{multiplication} by the number $n$,
or in short the product of the numbers $m$, $n$. This
therefore by (126) is completely determined by the
conditions
\begin{align*}
\text{II\@. }& m\centerdot 1=m\\
\text{III\@. }& mn'=mn+m
\end{align*}

\mypara{148.} Theorem.  $m'n = mn+n$.

Proof by complete induction (80).  For

$\rho$. by (147, II) and (135, II) the theorem is true
for $n = 1$.

% [File: 107.png]

$\sigma$. If the theorem is true for a number $n$, we have
\[
m'n+m'=(mn+n)+m'
\]
and consequently by (147, III), (141), (140), (136),
(141), (147, III)
\[
m'n'=mn+(n+m')=mn+(m'+n)=mn+(m+n')
\]
\[
=(mn+m)+n'=mn'+n';
\]
therefore the theorem is true for the following number
$n'$, which was to be proved.

\mypara{149.} Theorem. $1\centerdot n = n$.

Proof by complete induction (80). For

$\rho$. by (147, II) the theorem is true for $n = 1$.

$\sigma$. If the theorem is true for a number $n$, then we
have $1\centerdot n+1=n+1$, i.~e., by (147, III), (135, II)
$1\centerdot n'=n'$, therefore the theorem also holds for the following
number $n'$, which was to be proved.

\mypara{150.} Theorem.  $mn=nm$.

Proof by complete induction (80). For

$\rho$. by (147, II), (149) the theorem is true for $n = 1$.

$\sigma$. If the theorem is true for a number $n$, then we
have
\[
mn+m = nm+m,
\]
i.~e., by (147, III), (148) $mn'=n'm$, therefore the theorem
is also true for the following number $n'$, which
was to be proved.

\mypara{151.} Theorem. $l(m+n)=lm+ln$.

Proof by complete induction (80). For

$\rho$. by (135, II), (147, III), (147, II) the theorem
is true for $n = 1$.

% [File: 108.png]

$\sigma$. If the theorem is true for a number $n$, we have
\[
l(m+n)+l=(lm+ln)+l;
\]
but by (147, III), (135, III) we have
\[
l(m+n)+l=l(m+n)'=l(m+n'),
\]
and by (141), (147, III)
\[
(lm+ln)+l=lm+(ln+l)=lm+ln',
\]
consequently $l(m+n')=lm+ln'$, i.~e., the theorem
is true also for the following number $n'$, which was to
be proved.

\mypara{152.} Theorem. $(m+n)l=ml+nl$.

The proof follows from (151), (150).

\mypara{153.} Theorem. $(lm)n=l(mn)$.

Proof by complete induction (80). For

$\rho$. by (147, II) the theorem is true for $n = 1$.

$\sigma$. If the theorem is true for a number $n$, then we
have
\[
(lm)n+lm=l(mn)+lm,
\]
i.~e., by (147, III), (151), (147, III)
\[
(lm)n'=l(mn+m)=l(mn'),
\]
hence the theorem is also true for the following number
$n'$, which was to be proved.

\mypara{154.} Remark. If in (147) we had assumed no relation
between $\omega$ and $\theta$, but had put $\omega=k$, $\theta(n)=
m+n$, then by (126) we should have had a less simple
transformation $\psi$ of the number-series $N$; for the number~1
would $\psi(1)=k$ and for every other number
(therefore contained in the form $n'$) would $\psi(n')=
mn+k$; since thus would be fulfilled, as one could
% [File: 109.png]
easily convince himself by the aid of the foregoing
theorems, the condition $\psi(n')=\theta\psi(n)$, i.~e., $\psi(n')=
m+\psi(n)$ for all numbers $n$.

%XIII.

\mysect{V}{XIII}{Involution of Numbers}

\mypara{155.} Definition. If in theorem (126) we again put
$\Omega = N$, and further $\omega=a$, $\theta(n)=an=na$, we get a
transformation $\psi$ of $N$ which still satisfies the condition
\[
\text{I\@. }\psi(N) \partof N;
\]
the corresponding transform $\psi(n)$ of any number $n$
we denote by the symbol $a^n$, and call this number a
\textit{power of the base} $a$, while $n$ is called the \textit{exponent} of
this power of $a$. Hence this notion is completely determined
by the conditions
\begin{align*}
\text{II\@. }& a^1=a\\
\text{III\@. }& a^{n'}=a\centerdot a^n =\rlap{$a^n\centerdot a$}.
\end{align*}

\mypara{156.} Theorem. $a^{m+n}=a^m\centerdot a^n$.

Proof by complete induction (80). For

$\rho$. by (135, II), (155, III), (155, II) the theorem
is true for $n = 1$.

$\sigma$. If the theorem is true for a number $n$, we have
\[
a^{m+n}\centerdot a=(a^m\centerdot a^n)a;
\]
but by (155, III), (135, III)
$a^{m+n}\centerdot a=a^{(m+n)'}=a^{m+n'}$,
and by (153), (155, III)
$(a^m\centerdot a^n)a=a^m(a^n\centerdot a)=a^m\centerdot a^{n'}$;
hence $a^{m+n'}=a^m\centerdot a^{n'}$, i.~e., the theorem is also true for
the following number $n'$, which was to be proved.

\mypara{157.} Theorem. $(a^m)^n=a^{mn}$.

% [File: 110.png]

Proof by complete induction (80). For

$\rho$. by (155, II), (147, II) the theorem is true for
$n = 1$.

$\sigma$. If the theorem is true for a number $n$, we have
\[
(a^m)^n\centerdot a^m=a^{mn}\centerdot a^m
\]
but by (155, III) $(a^m)^n\centerdot a^m=(a^m)^{n'}$, and by (156), (147,
III) $a^{mn}\centerdot a^m=a^{mn+m}=a^{mn'}$; hence $(a^m)^{n'}=a^{mn'}$, i.~e.,
the theorem is also true for the following number $n'$,
which was to be proved.

\mypara{158.} Theorem.  $(ab)^n=a^n\centerdot b^n$

Proof by complete induction (80). For

$\rho$. by (155, II) the theorem is true for $n = 1$.

$\sigma$. If the theorem is true for a number $n$, then by
(150), (153), (155, III) we have also $(ab)^n\centerdot a=
a(a^n\centerdot b^n)=(a\centerdot a^n)b^n=a^{n'}\centerdot b^n$, and thus
$\bigl( (ab)^n\centerdot a \bigr)b=
(a^{n'}\centerdot b^n)b$; but by (153), (155, III)
$\bigl( (ab)^n\centerdot a \bigr)b=
(ab)^n\centerdot (ab)=(ab)^{n'}$, and likewise
\[
 (a^{n'}\centerdot b^n)b = a^{n'}\centerdot (b^n\centerdot b)
= a^{n'}\centerdot b^{n'};
\]
therefore $(ab)^{n'}=a^{n'}\centerdot b^{n'}$ i.~e., the theorem is also true
for the following number $n'$, which was to be proved.


%XIV.

\mysect{II}{XIV}{Number of the Elements of a Finite System}

\mypara{159.} Theorem. If $\Sigma$ is an infinite system, then is
every one of the number-systems $Z_n$ defined in (98)
similarly transformable in $\Sigma$ (i.~e., similar to a part of
$\Sigma$), and conversely.

Proof. If $\Sigma$ is infinite, then by (72) there certainly
exists a part $T$ of $\Sigma$, which is simply infinite, therefore
% [File: 111.png]
by (132) similar to the number-series $N$, and consequently
by (35) every system $Z_n$ as part of $N$ is similar
to a part of $T$, therefore also to a part of $\Sigma$, which
was to be proved.

The proof of the converse---however obvious it
may appear---is more complicated. If every system $Z_n$
is similarly transformable in $\Sigma$, then to every number
$n$ corresponds such a similar transformation $\alpha_n$ of $Z_n$
that $\alpha_n(Z_n) \partof \Sigma$. From the existence of such a series
of transformations $\alpha_n$, regarded as given, but respecting
which nothing further is assumed, we derive first
by the aid of theorem (126) the existence of a new
series of such transformations $\psi_n$ possessing the special
property that whenever $m \leqq n$, hence by (100)
$Z_m \partof Z_n$, the transformation $\psi_m$ of the part $Z_m$ is contained
in the transformation $\psi_n$ of $Z_n$ (21), i.~e., the
transformations $\psi_m$ and $\psi_n$ completely coincide with
each other for all numbers contained in $Z_m$, hence always
\[
\psi_m(m) = \psi_n(m).
\]
In order to apply the theorem stated to gain this end
we understand by $\Omega$ that system whose elements are
all possible similar transformations of all systems $Z_n$
in $\Sigma$, and by aid of the given elements $\alpha_n$, likewise
contained in $\Omega$, we define in the following manner
a transformation $\theta$ of $\Omega$ in itself. If $\beta$ is any element
of $\Omega$, thus, e.~g., a similar transformation of the determinate
system $Z_n$ in $\Sigma$, then the system $\alpha_{n'}(Z_{n'})$
cannot be part of $\beta(Z_n)$, for otherwise $Z_{n'}$ would be
% [File: 112.png]
similar by (35) to a part of $Z_n$, hence by (107) to a
proper part of itself, and consequently infinite, which
would contradict theorem (119); therefore there certainly
exists in $Z_{n'}$ one number or several numbers $p$
such that $\alpha_{n'}(p)$ is not contained in $\beta(Z_n)$; from these
numbers $p$ we select---simply to lay down something
determinate---always the least $k$ (96) and, since $Z_{n'}$ by
(108) is compounded out of $Z_n$ and $n'$, define a transformation
$\gamma$ of $Z_{n'}$ such that for all numbers $m$ contained
in $Z_n$ the transform $\gamma(m)=\beta(m)$ and besides
$\gamma(n')=\alpha_{n'}(k)$; this obviously similar transformation $\gamma$
of $Z_{n'}$ in $\Sigma$ we consider then as a transform $\theta(\beta)$ of the
transformation $\beta$, %[**Greek: beta]
and thus a transformation $\theta$ of the
system $\Omega$ in itself is completely defined. After the
things named $\Omega$ and $\theta$ in (126) are determined we select
finally for the element of $\Omega$, denoted by $\omega$ the given
transformation $\alpha_1$; thus by (126) there is determined
a transformation $\psi$ of the number-series $N$ in $\Omega$, which,
if we denote the transform belonging to an arbitrary
number $n$, not by $\psi(n)$ but by $\psi_n$, satisfies the conditions
\begin{align*}
\text{II\@. }& \psi_1=\alpha_1\\
\text{III\@. }& \psi_{n'}=\theta(\psi_n)
\end{align*}
By complete induction (80) it results first that $\psi_n$ is a
similar transformation of $Z_n$ in $\Sigma$; for

$\rho$. by II this is true for $n =1$.

$\sigma$. if this statement is true for a number $n$, it follows
from III and from the character of the above described
transition $\theta$ from $\beta$ to $\gamma$, that the statement is
% [File: 113.png]
also true for the following number $n'$, which was to be
proved. Afterward we show likewise by complete induction
(80) that if $m$ is any number the above stated
property
\[
\psi_n(m)=\psi_m(m)
\]
actually belongs to all numbers $n$, which are $\geqq m$, and
therefore by (93), (74) belong to the chain $m_0$; in
fact,

$\rho$. this is immediately evident for $n = m$, and

$\sigma$. if this property belongs to a number $n$ it follows
again from III and the nature of $\theta$, that it also belongs
to the number $n'$, which was to be proved. After this
special property of our new series of transformations
$\psi_n$ has been established, we can easily prove our theorem.
We define a transformation $\chi$ of the number-series
$N$, in which to every number $n$ we let the transform
$\chi(n)=\psi_n(n)$ correspond; obviously by (21) all
transformations $\psi_n$ are contained in this one transformation
$\chi$. Since $\psi_n$ was a transformation of $Z_n$ in
$\Sigma$, it follows first that the number-series $N$ is likewise
transformed by $\chi$ in $\Sigma$, hence $\chi(N) \partof \Sigma$. If further $m$,
$n$ are different numbers we may by reason of symmetry
according to (90) suppose $m<n$; then by the
foregoing $\chi(m)=\psi_m(m)=\psi_n(m)$, and $\chi(n)=\psi_n(n)$;
but since $\psi_n$ was a similar transformation of $Z_n$ in $\Sigma$,
and $m$, $n$ are different elements of $Z_n$, then is $\psi_n(m)$
different from $\psi_n(n)$, hence also $\chi(m)$ different from
$\chi(n)$, i.~e., $\chi$ is a similar transformation of $N$. Since
further $N$ is an infinite system (71), the same thing
% [File: 114.png]
is true by (67) of the system $\chi(N)$ similar to it and
by (68), because $\chi(N)$ is part of $\Sigma$, also of $\Sigma$, which
was to be proved.

\mypara{160.} Theorem. A system $\Sigma$ is finite or infinite,
according as there does or does not exist a system
$Z_n$ similar to it.

Proof. If $\Sigma$ is finite, then by (159) there exist
systems $Z_n$ which are not similarly transformable in
$\Sigma$; since by (102) the system $Z_1$ consists of the single
number 1, and hence is similarly transformable in
every system, then must the least number $k$ (96) to
which a system $Z_k$ not similarly transformable in $\Sigma$ corresponds
be different from 1 and hence by (78) $=n'$,
and since $n<n'$ (91) there exists a similar transformation
$\psi$ of $Z_n$ in $\Sigma$; if then $\psi(z_n)$ were only a proper part
of $\Sigma$, i.~e., if there existed an element $\alpha$ in $\Sigma$ not contained
in $\psi(Z_n)$, then since $Z_{n'}=\mathfrak{M}(Z_n, n')$ (108)
we could extend this transformation $\psi$ to a similar
transformation $\psi$ of $Z_{n'}$ in $\Sigma$ by putting $\psi(n')=\alpha$ while
by our hypothesis $Z_{n'}$ is not similarly transformable
in $\Sigma$. Hence $\psi(Z_n) = \Sigma$, i.~e., $Z_n$ and $\Sigma$ are similar
systems. Conversely, if a system $\Sigma$ is similar to a
system $Z_n$, then by (119), (67) $\Sigma$ is finite, which was
to be proved.

\mypara{161.} Definition.  If $\Sigma$ is a finite system, then by
(160) there exists one and by (120), (33) only one
single number $n$ to which a system $Z_n$ similar to the
system $\Sigma$ corresponds; this number $n$ is called the
\textit{number} [\textit{Anzahl}] of the elements contained in $\Sigma$ (or
% [File: 115.png]
also the \textit{degree} of the system $\Sigma$) and we say $\Sigma$ consists
of or is a system of $n$ elements, or the number $n$ shows
\textit{how many} elements are contained in $\Sigma$.\footnote
  {For clearness and simplicity in what follows we restrict the notion of
  the number throughout to finite systems; if then we speak of a number of certain
  things, it is always understood that the system whose elements these
  things are is a finite system.}
If numbers
are used to express accurately this determinate property
of finite systems they are called \textit{cardinal numbers}.
As soon as a determinate similar transformation $\psi$ of
the system $Z_n$ is chosen by reason of which $\psi(Z_n)=Z$,
then to every number $m$ contained in $Z_n$ (i.~e., every
number $m$ which is $\leqq n$) there corresponds a determinate
element $\psi(m)$ of the system $\Sigma$, and conversely
by (26) to every element of $\Sigma$ by the inverse transformation
$\overline{\psi}$ there corresponds a determinate number
$m$ in $Z_n$. Very often we denote all elements of $\Sigma$ by a
single letter, e.~g., $\alpha$, to which we append the distinguishing
numbers $m$ as indices so that $\psi(m)$ is denoted
by $\alpha_m$. We say also that these elements are \textit{counted
and set in order} by $\psi$ in determinate manner, and call
$\alpha_m$ the $m$th element of $\Sigma$; if $m<n$ then $\alpha_{m'}$ is called
the element \textit{following} $\alpha_m$, and $\alpha_n$ is called the \textit{last} element.
In this counting of the elements therefore the
numbers $m$ appear again as ordinal numbers (73).

\mypara{162.} Theorem. All systems similar to a finite system
possess the same number of elements.

The proof follows immediately from (33), (161).

\mypara{163.} Theorem. The number of numbers contained
in $Z_n$, i.~e., of those numbers which are $\leqq n$, is $n$.

% [File: 116.png]

Proof. For by (32) $Z_n$ is similar to itself.

\mypara{164.} Theorem. If a system consists of a single
element, then is the number of its elements $=1$, and
conversely.

The proof follows immediately from (2), (26), (32),
(102), (161).

\mypara{165.} Theorem. If $T$ is proper part of a finite system
$\Sigma$, then is the number of the elements of $T$ less
than that of the elements of $\Sigma$.

Proof. By (68) $T$ is a finite system, therefore
similar to a system $Z_m$, where $m$ denotes the number
of the elements of $T$; if further $n$ is the number of
elements of $\Sigma$, therefore $\Sigma$ similar to $Z_n$, then by (35)
$T$ is similar to a proper part $E$ of $Z_n$ and by (33) also
$Z_m$ and $E$ are similar to each other; if then we were
to have $n \leqq m$, hence $Z_n \partof Z_m$, by (7) $E$ would also be
proper part of $Z_m$, and consequently $Z_m$ an infinite
system, which contradicts theorem (119); hence by
(90), $m<n$, which was to be proved.

\mypara{166.} Theorem. If $\Gamma=\mathfrak{M}(B,\gamma)$, where $B$ denotes
a system of $n$ elements, and $\gamma$ an element of $\Gamma$ not
contained in $B$, then $\Gamma$ consists of $n'$ elements.

Proof. For if $B=\psi(Z_n)$, where $\psi$ denotes a similar
transformation of $Z_n$, then by (105), (108) it may
be extended to a similar transformation $\psi$ of $Z_{n'}$, by
putting $\psi(n')=\gamma$, and we get $\psi(Z_{n'})=\Gamma$, which was to
be proved.

\mypara{167.} Theorem. If $\gamma$ is an element of a system $\Gamma$
% [File: 117.png]
consisting of $n'$ elements, then is $n$ the number of all
other elements of $\Gamma$.

Proof. For if $B$ denotes the aggregate of all elements
in $\Gamma$ different from $\gamma$, then is $\Gamma=\mathfrak{M}(B,\gamma)$; if
then $b$ is the number of elements of the finite system
$B$, by the foregoing theorem $b'$ is the number of elements
of $\Gamma$, therefore $=n'$, whence by (26) we get
$b=n$, which was to be proved.

\mypara{168.} Theorem. If $A$ consists of $m$ elements, and
$B$ of $n$ elements, and $A$ and $B$ have no common element,
then $\mathfrak{M}(A,B)$ consists of $m+n$ elements.

Proof by complete induction (80). For

$\rho$. by (166), (164), (135, II) the theorem is true
for $n = 1$.

$\sigma$. If the theorem is true for a number $n$, then is it
also true for the following number $n'$. In fact, if $\Gamma$ is
a system of $n'$ elements, then by (167) we can put
$\Gamma=\mathfrak{M}(B,\gamma)$ where $\gamma$ denotes an element and $B$ the
system of the $n$ other elements of $\Gamma$. If then $A$ is a
system of $m$ elements each of which is not contained
in $\Gamma$, therefore also not contained in $B$, and we put
$\mathfrak{M}(A,B)=\Sigma$, by our hypothesis $m+n$ the number
of elements of $\Sigma$, and since $\gamma$ is not contained in $\Sigma$,
then by (166) the number of elements contained in
$\mathfrak{M}(\Sigma,\gamma)=(m+n')$, therefore by (135, III) $=m+n'$;
but since by (15) obviously $\mathfrak{M}(\Sigma,\gamma)=\mathfrak{M}(A,B,\gamma)=
\mathfrak{M}(A,\Gamma)$, then is $m+n'$ the number of the elements
of $\mathfrak{M}(A,\Gamma)$, which was to be proved.

\mypara{169.} Theorem. If $A$, $B$ are finite systems of $m$, $n$
% [File: 118.png]
elements respectively, then is $\mathfrak{M}(A,B)$ a finite system
and the number of its elements is $\leqq m+n$.

Proof. If $B \partof A$, then $\mathfrak{M}(A,B)=A$ and the
number $m$ of the elements of this system is by (142)
$<m+n$, as was stated. But if $B$ is not part of $A$,
and $T$ is the system of all those elements of $B$ that
are not contained in $A$, then by (165) is their number
$p \leqq n$, and since obviously
\[
\mathfrak{M}(A,B)=\mathfrak{M}(A,T),
\]
then by (143) is the number $m+p$ of the elements of
this system $\leqq m+n$, which was to be proved.

\mypara{170.} Theorem. Every system compounded out of
a number $n$ of finite systems is finite.

Proof by complete induction (80). For

$\rho$. by (8) the theorem is self-evident for $n = 1$.

$\sigma$. If the theorem is true for a number $n$, and if $\Sigma$
is compounded out of $n'$ finite systems, then let $A$ be
one of these systems and $B$ the system compounded
out of all the rest; since their number by (167) $=n$,
then by our hypothesis $B$ is a finite system. Since
obviously $\Sigma=\mathfrak{M}(A,B)$, it follows from this and from
(169) that $\Sigma$ is also a finite system, which was to be
proved.

\mypara{171.} Theorem. If $\psi$ is a dissimilar transformation
of a finite system $\Sigma$ of $n$ elements, then is the number
of elements of the transform $\psi(\Sigma)$ less than $n$.

Proof. If we select from all those elements of $\Sigma$
that possess one and the same transform, always one
and only one at pleasure, then is the system $T$ of all
% [File: 119.png]
these selected elements obviously a proper part of
$\Sigma$, because $\psi$ is a dissimilar transformation of $\Sigma$ (26).
At the same time it is clear that the transformation
by (21) contained in $\psi$ of this part $T$ is a similar transformation,
and that $\psi(T)=\psi(\Sigma)$; hence the system
$\psi(\Sigma)$ is similar to the proper part $T$ of $\Sigma$, and consequently
our theorem follows by (162), (165).

\mypara{172.} Final remark. Although it has just been
shown that the number $m$ of the elements of $\psi(\Sigma)$ is
less than the number $n$ of the elements of $\Sigma$, yet in
many cases we like to say that the number of elements
of $\psi(\Sigma)=n$. The word number is then, of
course, used in a different sense from that used
hitherto (161); for if $\alpha$ is an element of $\Sigma$ and $a$ the
number of all those elements of $\Sigma$, that possess one
and the same transform $\psi(\alpha)$ then is the latter as element
of $\psi(\Sigma)$ frequently regarded still as representative
of $a$ elements, which at least from their derivation
may be considered as different from one another,
and accordingly counted as $a$-fold element of $\psi(\Sigma)$.
In this way we reach the notion, very useful in many
cases, of systems in which every element is endowed
with a certain frequency-number which indicates how
often it is to be reckoned as element of the system.
In the foregoing case, e.~g., we would say that $n$ is
the number of the elements of $\psi(\Sigma)$ counted in this
sense, while the number $m$ of the actually different
elements of this system coincides with the number of
the elements of $T$. Similar deviations from the original
% [File: 120.png]
meaning of a technical term which are simply extensions
of the original notion, occur very frequently
in mathematics; but it does not lie in the line of this
memoir to go further into their discussion.

% [File: 121.png]
%[Blank Page]
% [File: 122.png]

\newpage
\pagestyle{empty}
\begin{center}
{\Large CATALOGUE OF PUBLICATIONS}

{\scriptsize OF THE}

{\large OPEN COURT PUBLISHING CO.}

\rule[.5ex]{3cm}{.2pt}
\end{center}

{\parskip=0ex\raggedright

\parindent=0em \normalsize
COPE, E.~D.
\nopagebreak

\parindent=3ex\hangindent=9ex \small
THE PRIMARY FACTORS OF ORGANIC EVOLUTION.
\nopagebreak

\parindent=6ex\hangindent=9ex \footnotesize
121 cuts. Pp.~xvi, 547. Cloth, \$2.00 (10s.).
\smallskip

\parindent=0em \normalsize
M�LLER, F. MAX.
\nopagebreak

\parindent=3ex\hangindent=9ex\small
THREE INTRODUCTORY LECTURES ON THE SCIENCE OF THOUGHT.
\nopagebreak

\parindent=6ex\hangindent=9ex\footnotesize
128 pages. Cloth, 75c (3s.~6d.).
\smallskip

\parindent=3ex\hangindent=9ex\small
THREE LECTURES ON THE SCIENCE OF LANGUAGE.
\nopagebreak

\parindent=6ex\hangindent=9ex\footnotesize
112 pages, 2nd Edition. Cloth, 75c* (3s.~6d.).
\smallskip

\parindent=0em \normalsize
ROMANES, GEORGE JOHN.
\nopagebreak

\parindent=3ex\hangindent=9ex\small
DARWIN AND AFTER DARWIN.
\nopagebreak

\parindent=6ex\hangindent=9ex \footnotesize
Three Vols., \$4.00. Singly, as follows:
\\
1. \textsc{The Darwinian Theory}. 460 pages. 125 illustrations. Cloth, \$2.00
\\
2. \textsc{Post-Darwinian Questions}. Heredity and Utility. Pp.~338. \$1.50
\\
3. \textsc{Post-Darwinian Questions}. Isolation and Physiological Selection.
Pp.~181. \$1.00.
\smallskip

\parindent=3ex\hangindent=9ex \small
AN EXAMINATION OF WEISMANNISM.
\nopagebreak

\parindent=6ex\hangindent=9ex \footnotesize
236 pages.~Cloth, \$1.00.
\smallskip

\parindent=3ex\hangindent=9ex \small
THOUGHTS ON RELIGION.
\nopagebreak

\parindent=6ex\hangindent=9ex\footnotesize
Third Edition, Pages, 184. Cloth, gilt top, \$1.25.
\smallskip

\parindent=0em \normalsize
SHUTE, DR.\ D. KERFOOT.
\nopagebreak

\parindent=3ex\hangindent=9ex \small
FIRST BOOK IN ORGANIC EVOLUTION.
\nopagebreak

\parindent=6ex\hangindent=9ex\footnotesize
9 colored plates, 39 cuts. Pp.~xvi + 285. Price, \$2.00 (7s.~6d.).
\smallskip

\parindent=0em \normalsize
MACH, ERNST.
\nopagebreak

\parindent=3ex\hangindent=9ex \small
THE SCIENCE OF MECHANICS.
\nopagebreak

\parindent=6ex\hangindent=9ex\footnotesize
Translated by \textsc{T. J. McCormack}. 250 cuts. 534 pages. \$2.50 (12s.~6d.)
\smallskip

\parindent=3ex\hangindent=9ex \small
POPULAR SCIENTIFIC LECTURES.
\nopagebreak

\parindent=6ex\hangindent=9ex\footnotesize
Third Edition. 415 pages.~59 cuts.~Cloth, gilt top. \$1.50 (7s 6d.).
\smallskip

\parindent=3ex\hangindent=9ex \small
THE ANALYSIS OF THE SENSATIONS.
\nopagebreak

\parindent=6ex\hangindent=9ex\footnotesize
Pp.~208. 37 cuts. Cloth, \$1.25 (6s.~6d.).
\smallskip

\parindent=0em \normalsize
LAGRANGE, JOSEPH LOUIS.
\nopagebreak

\parindent=3ex\hangindent=9ex \small
LECTURES ON ELEMENTARY MATHEMATICS.
\nopagebreak

\parindent=6ex\hangindent=9ex\footnotesize
With portrait of the author. Pp.~172. Price, \$1.00 (5s.).
\smallskip

\parindent=0em \normalsize
DE MORGAN, AUGUSTUS.
\nopagebreak

\parindent=3ex\hangindent=9ex \small
ON THE STUDY AND DIFFICULTIES OF MATHEMATICS.
\nopagebreak

\parindent=6ex\hangindent=9ex\footnotesize
New Reprint edition with notes. Pp.~viii + 288. Cloth, \$1.25 (5s.).
\smallskip

\parindent=3ex\hangindent=9ex \small
ELEMENTARY ILLUSTRATIONS OF THE DIFFERENTIAL AND INTEGRAL CALCULUS.
\nopagebreak

\parindent=6ex\hangindent=9ex\footnotesize
New reprint edition. Price, \$1.00 (5s.).
\smallskip

\parindent=0em \normalsize
FINK, KARL.
\nopagebreak

\parindent=3ex\hangindent=9ex \small
A BRIEF HISTORY OF MATHEMATICS.
\nopagebreak

\parindent=6ex\hangindent=9ex\footnotesize
Trans.~by W. W. Beman and D. E. Smith. Pp.,~333. Cloth, \$1.50 (5s.~6d.)
\smallskip

\parindent=0em \normalsize
SCHUBERT, HERMANN.
\nopagebreak

\parindent=3ex\hangindent=9ex \small
MATHEMATICAL ESSAYS AND RECREATIONS.
\nopagebreak

\parindent=6ex\hangindent=9ex\footnotesize
Pp.~149. Cuts, 37. Cloth, 75c (3s.~6d.).
\smallskip

\parindent=0em \normalsize
HUC AND GABET, MM.
\nopagebreak

\parindent=3ex\hangindent=9ex \small
TRAVELS IN TARTARY, THIBET AND CHINA.
\nopagebreak

\parindent=6ex\hangindent=9ex\footnotesize
100 engravings. Pp.~28 + 660. 2 vols. \$2.00 (10s.), One vol., \$1.25 (5s.)
\smallskip

% [File: 123.png]

\parindent=0em \normalsize
CARUS, PAUL.
\nopagebreak

\parindent=3ex\hangindent=9ex \small
THE HISTORY OF THE DEVIL, AND THE IDEA OF EVIL.
\nopagebreak

\parindent=6ex\hangindent=9ex\footnotesize
311 Illustrations. Pages, 500. Price, \$6.00 (30s.).
\smallskip

\parindent=3ex\hangindent=9ex \small
EROS AND PSYCHE.
\nopagebreak

\parindent=6ex\hangindent=9ex\footnotesize
Retold after Apuleius. With Illustrations by Paul Thumann. Pp.~125.
Price, \$1.50 (6s.).
\smallskip

\parindent=3ex\hangindent=9ex \small
WHENCE AND WHITHER?
\nopagebreak

\parindent=6ex\hangindent=9ex\footnotesize
An Inquiry into the Nature of the Soul. 196 pages. Cloth, 75c (3s.~6d.)
\smallskip

\parindent=3ex\hangindent=9ex \small
THE ETHICAL PROBLEM.
\nopagebreak

\parindent=6ex\hangindent=9ex\footnotesize
Second edition, revised and enlarged. 351 pages. Cloth, \$1.25 (6s.~6d.)
\smallskip

\parindent=3ex\hangindent=9ex \small
FUNDAMENTAL PROBLEMS.
\nopagebreak

\parindent=6ex\hangindent=9ex\footnotesize
Second edition, revised and enlarged. 372 Pp.~Cl., \$1.50 (7s.~6d.).
\smallskip

\parindent=3ex\hangindent=9ex \small
HOMILIES OF SCIENCE.
\nopagebreak

\parindent=6ex\hangindent=9ex\footnotesize
317 pages. Cloth, Gilt Top, \$1.50 (7s.~6d.).
\smallskip

\parindent=3ex\hangindent=9ex \small
THE IDEA OF GOD.
\nopagebreak

\parindent=6ex\hangindent=9ex\footnotesize
Fourth edition. 32 pages. Paper, 15c (9d.).
\smallskip

\parindent=3ex\hangindent=9ex \small
THE SOUL OF MAN.
\nopagebreak

\parindent=6ex\hangindent=9ex\footnotesize
2nd ed. 182 cuts. 482 pages. Cloth, \$1.50 (6s.).
\smallskip

\parindent=3ex\hangindent=9ex \small
TRUTH IN FICTION\@. \textsc{Twelve Tales With a Moral.}
\nopagebreak

\parindent=6ex\hangindent=9ex\footnotesize
White and gold binding, gilt edges. Pp.~111. \$1.00 (5s.).
\smallskip

\parindent=3ex\hangindent=9ex \small
THE RELIGION OF SCIENCE.
\nopagebreak

\parindent=6ex\hangindent=9ex\footnotesize
Second, extra edition. Pp.~103. Price, 50c (2s.~6d.).
\smallskip

\parindent=3ex\hangindent=9ex \small
PRIMER OF PHILOSOPHY.
\nopagebreak

\parindent=6ex\hangindent=9ex\footnotesize
240 pages. Second Edition. Cloth, \$1.00 (5s.).
\smallskip

\parindent=3ex\hangindent=9ex \small
THE GOSPEL OF BUDDHA\@. According to Old Records.
\nopagebreak

\parindent=6ex\hangindent=9ex\footnotesize
Fifth Edition. Pp.~275. Cloth, \$1.00 (5s.). In German, \$1.25 (6s.~6d.)
\smallskip

\parindent=3ex\hangindent=9ex \small
BUDDHISM AND ITS CHRISTIAN CRITICS.
\nopagebreak

\parindent=6ex\hangindent=9ex\footnotesize
Pages, 311. Cloth, \$1.25 (6s.~6d.).
\smallskip

\parindent=3ex\hangindent=9ex \small
KARMA\@. \textsc{A Story of Early Buddhism.}
\nopagebreak

\parindent=6ex\hangindent=9ex\footnotesize
Illustrated by Japanese artists. Cr�pe paper, 75c (3s.~6d.).
\smallskip

\parindent=3ex\hangindent=9ex \small
NIRVANA: \textsc{A Story of Buddhist Psychology.}
\nopagebreak

\parindent=6ex\hangindent=9ex\footnotesize
Japanese edition, like \textit{Karma}, \$1.00 (4s.~6d.).
\smallskip

\parindent=3ex\hangindent=9ex \small
LAO-TZE'S TAO-TEH-KING.
\nopagebreak

\parindent=6ex\hangindent=9ex\footnotesize
Chinese-English. Pp.~360. Cloth, \$3.00 (15s.).
\medskip

\parindent=0em \normalsize
CORNILL, CARL HEINRICH.
\nopagebreak

\parindent=3ex\hangindent=9ex \small
THE PROPHETS OF ISRAEL.
\nopagebreak

\parindent=6ex\hangindent=9ex\footnotesize
Pp., 200. Cloth, \$1.00 (5s.).
\smallskip

\parindent=3ex\hangindent=9ex \small
HISTORY OF THE PEOPLE OF ISRAEL.
\nopagebreak

\parindent=6ex\hangindent=9ex\footnotesize
Pp.~vi + 325. Cloth, \$1.50 (7s.~6d.).
\medskip


\parindent=0em \normalsize
POWELL, J. W.

\parindent=3ex\hangindent=9ex \small
TRUTH AND ERROR; or, the Science of Intellection.
\nopagebreak

\parindent=6ex\hangindent=9ex\footnotesize
Pp.~423. Cloth, \$1.75 (7s.~6d.).
\medskip


\parindent=0em \normalsize
RIBOT, TH.
\nopagebreak

\parindent=3ex\hangindent=9ex \small
THE PSYCHOLOGY OF ATTENTION.
\smallskip

\parindent=3ex\hangindent=9ex \small
THE DISEASES OF PERSONALITY.
\smallskip

\parindent=3ex\hangindent=9ex \small
THE DISEASES OF THE WILL.
\nopagebreak

\parindent=6ex\hangindent=9ex\footnotesize
Cloth, 75 cents each (3s.~6d.). \textit{Full set, cloth, \$1.75} (9s.).
\smallskip

\parindent=3ex\hangindent=9ex \small
EVOLUTION OF GENERAL IDEAS.
\nopagebreak

\parindent=6ex\hangindent=9ex\footnotesize
Pp.~231. Cloth, \$1.25 (5s.).
\medskip


\parindent=0em \normalsize
WAGNER, RICHARD.
\nopagebreak

\parindent=3ex\hangindent=9ex \small
A PILGRIMAGE TO BEETHOVEN.
\nopagebreak

\parindent=6ex\hangindent=9ex\footnotesize
A Story. With portrait of Beethoven. Pp.~40. Boards, 50c (2s.~6d.).
\medskip


\parindent=0em \normalsize
HUTCHINSON, WOODS.

\parindent=3ex\hangindent=9ex \small
THE GOSPEL ACCORDING TO DARWIN.
\nopagebreak

\parindent=6ex\hangindent=9ex\footnotesize
Pp.~xii + 241. Price, \$1.50 (6s.).
\medskip


\parindent=0em \normalsize
FREYTAG, GUSTAV.
\nopagebreak

\parindent=3ex\hangindent=9ex \small
THE LOST MANUSCRIPT\@. A Novel.
\nopagebreak

\parindent=6ex\hangindent=9ex\footnotesize
2 vols. 953 pages. Extra cloth, \$4.00 (21s). One vol., cl., \$1.00 (5s.)
\smallskip

\parindent=3ex\hangindent=9ex \small
MARTIN LUTHER.
\nopagebreak

\parindent=6ex\hangindent=9ex\footnotesize
Illustrated. Pp.~130. Cloth, \$1.00 (5s.).
\smallskip

% [File: 124.png]

\parindent=0em \normalsize
A�VAGHOSHA.
\nopagebreak

\parindent=3ex\hangindent=9ex \small
DISCOURSE ON THE AWAKENING OF FAITH in the Mah�y�na.
\nopagebreak

\parindent=6ex\hangindent=6ex\footnotesize
Translated for the first time from the Chinese version by Tietaro
Suzuki. Pages, 176. Price, cloth, \$1.25 (5s.~6d.).
\medskip


\parindent=0em \normalsize
TRUMBULL, M. M.
\nopagebreak

\parindent=3ex\hangindent=9ex \small
THE FREE TRADE STRUGGLE IN ENGLAND.
\nopagebreak

\parindent=6ex\hangindent=6ex\footnotesize
Second Edition. 296 pages. Cloth, 75c (3s.~6d.).
\smallskip

\parindent=3ex\hangindent=9ex \small
WHEELBARROW: \textsc{Articles and Discussions on the Labor Question},
\nopagebreak

\parindent=6ex\hangindent=6ex\footnotesize
With portrait of the author. 303 pages. Cloth, \$1.00 (5s.).
\smallskip

\parindent=3ex\hangindent=9ex \small
GOETHE AND SCHILLER'S XENIONS.
\nopagebreak

\parindent=6ex\hangindent=6ex\footnotesize
Translated by Paul Carus. Album form. Pp.~162. Cl., \$1.00 (5s.).
\medskip


\parindent=0em \normalsize
OLDENBERG, H.
\nopagebreak

\parindent=3ex\hangindent=9ex \small
ANCIENT INDIA: ITS LANGUAGE AND RELIGIONS.
\nopagebreak

\parindent=6ex\hangindent=6ex\footnotesize
Pp.~l00. Cloth, 50c (2s.~6d.).
\medskip


\parindent=0em \normalsize
CONWAY, DR.\ MONCURE DANIEL.
\nopagebreak

\parindent=3ex\hangindent=9ex \small
SOLOMON, AND SOLOMONIC LITERATURE.
\nopagebreak

\parindent=6ex\hangindent=6ex\footnotesize
Pp.~243. Cloth, \$1.50 (6s.).
\medskip


\parindent=0em \normalsize
GARBE, RICHARD.
\nopagebreak

\parindent=3ex\hangindent=9ex \small
THE REDEMPTION OF THE BRAHMAN\@. \textsc{A Tale of Hindu Life}.
\nopagebreak

\parindent=6ex\hangindent=6ex\footnotesize
Laid paper. Gilt top. 96 pages. Price, 75c (3s.~6d.).
\smallskip

\parindent=3ex\hangindent=9ex \small
THE PHILOSOPHY OF ANCIENT INDIA.
\nopagebreak

\parindent=6ex\hangindent=6ex\footnotesize
Pp.~89. Cloth, 50c (2s.~6d.).
\medskip


\parindent=0em \normalsize
HUEPPE, FERDINAND.
\nopagebreak

\parindent=3ex\hangindent=9ex \small
THE PRINCIPLES OF BACTERIOLOGY.
\nopagebreak

\parindent=6ex\hangindent=6ex\footnotesize
28 Woodcuts. Pp.~x + 467. Price, \$1.75 (9s.).
\medskip


\parindent=0em \normalsize
L�VY-BRUHL, PROF.\ L.
\nopagebreak

\parindent=3ex\hangindent=9ex \small
HISTORY OF MODERN PHILOSOPHY IN FRANCE.
\nopagebreak

\parindent=6ex\hangindent=6ex\footnotesize
23 Portraits. Handsomely bound. Pp.~500. Price, \$3.00 (12s.).
\medskip


\parindent=0em \normalsize
TOPINARD, DR.\ PAUL.
\nopagebreak

\parindent=3ex\hangindent=9ex \small
SCIENCE AND FAITH, \textsc{or Man as an Animal and Man as a Member
of Society}.
\nopagebreak

\parindent=6ex\hangindent=6ex\footnotesize
Pp.~374. Cloth, \$1.50 (6s.~6d.).
\medskip


\parindent=0em \normalsize
BINET, ALFRED.
\nopagebreak

\parindent=3ex\hangindent=9ex \small
THE PSYCHOLOGY OF REASONING.
\nopagebreak

\parindent=6ex\hangindent=6ex\footnotesize
Pp.~193. Cloth, 75c (3s.~6d.).
\smallskip

\parindent=3ex\hangindent=9ex \small
THE PSYCHIC LIFE OF MICRO-ORGANISMS.
\nopagebreak

\parindent=6ex\hangindent=6ex\footnotesize
Pp.~135. Cloth, 75 cents.
\smallskip

\parindent=3ex\hangindent=9ex \small
ON DOUBLE CONSCIOUSNESS.
\nopagebreak

\parindent=6ex\hangindent=6ex\footnotesize
See No.~8, Religion of Science Library.
\medskip


\parindent=0em \normalsize
THE OPEN COURT.
\nopagebreak

\parindent=3ex\hangindent=3ex \small
A Monthly Magazine Devoted to the Science of Religion, the Religion of
Science, and the Extension of the Religious Parliament Idea.
\nopagebreak

\parindent=6ex\hangindent=6ex\footnotesize
Terms: \$1.00 a year; 5s.~6d.\ to foreign countries in the Postal Union.
Single Copies, 10 cents (6d.).
\medskip


\parindent=0em \normalsize
THE MONIST.
\nopagebreak

\parindent=3ex \small
A Quarterly Magazine of Philosophy and Science.
\nopagebreak

\parindent=6ex\hangindent=6ex\footnotesize
Per copy, 50 cents; Yearly, \$2.00. In England and all countries in
U.P.U. per copy, 2s.~6d.: Yearly, 9s.~6d.

} %end of \parskip=0ex\raggedright

\begin{center}
\rule[.5ex]{3cm}{.2pt}

CHICAGO:
\medskip

{\Large THE OPEN COURT PUBLISHING CO.,}

Monon Building, 324 Dearborn St.

{\small LONDON: Kegan Paul, Trench, Tr�bner \& Company, Ltd.}
\end{center}

% [File: 125.png]

\newpage

\begin{center}
\textit{\Large The Religion of Science Library.}

\rule[.5ex]{3cm}{.2pt}
\end{center}

\begin{small}
A collection of bi-monthly publications, most of which are reprints of
books published by The Open Court Publishing Company. Yearly, \$1.50.
Separate copies according to prices quoted. The books are printed upon
good paper, from large type.

The Religion of Science Library, by its extraordinarily reasonable price
will place a large number of valuable books within the reach of all readers.

The following have already appeared in the series:
\begin{itemize}
\item[]\hspace{-3ex}\llap{No.\ 1.}
\textit{The Religion of Science}. By \textsc{Paul Carus}. 25c (1s.~6d.).

\item[]\hspace{-3ex}\llap{2.}
\textit{Three Introductory Lectures on the Science of Thought}, By \textsc{F.~Max
M�ller}.  25c (1s.~6d.).

\item[]\hspace{-3ex}\llap{3.}
\textit{Three Lectures on the Science of Language}. \textsc{F.~Max M�ller}. 25c (1s.~6d.)

\item[]\hspace{-3ex}\llap{4.}
\textit{The Diseases of Personality}, By \textsc{Th.~Ribot}. 25c (1s.~6d.).

\item[]\hspace{-3ex}\llap{5.}
\textit{The Psychology of Attention}. By \textsc{Th.~Ribot}. 25c (1s.~6d.).

\item[]\hspace{-3ex}\llap{6.}
\textit{The Psychic Life of Micro-Organisms},  By \textsc{Alfred Binet}. 25c (1s.~6d.)

\item[]\hspace{-3ex}\llap{7.}
\textit{The Nature of the State}. By \textsc{Paul Carus}. 15c (9d.).

\item[]\hspace{-3ex}\llap{8.}
\textit{On Double Consciousness}. By \textsc{Alfred Binet.} 15c (9d.).

\item[]\hspace{-3ex}\llap{9.}
\textit{Fundamental Problems}. By \textsc{Paul Carus}. 50c (2s.~6d.).

\item[]\hspace{-3ex}\llap{10.} \textit{The Diseases of the Will}, By \textsc{Th.~Ribot}. 25c (1s.~6d.).

\item[]\hspace{-3ex}\llap{11.}
\textit{The Origin of Language}, By \textsc{Ludwig Noire}. 15c (9d.).

\item[]\hspace{-3ex}\llap{12.}
\textit{The Free Trade Struggle in England}. \textsc{M.~M. Trumbull}. 25c (1s.~6d.)

\item[]\hspace{-3ex}\llap{13.}
\textit{Wheelbarrow on the Labor Question}, By \textsc{M.~M. Trumbull}. 35c (2s.).

\item[]\hspace{-3ex}\llap{14.}
\textit{The Gospel of Buddha}. By \textsc{Paul Carus}. 35c (2s.).

\item[]\hspace{-3ex}\llap{15.}
\textit{The Primer of Philosophy}. By \textsc{Paul Carus}. 25c (1s.~6d.).

\item[]\hspace{-3ex}\llap{16.}
\textit{On Memory}, and \textit{The Specific Energies of the Nervous System}. By \textsc{Prof.\
Ewald Hering}. 15c (9d.).

\item[]\hspace{-3ex}\llap{17.}
\textit{The Redemption of the Brahman}. Tale of Hindu Life.  By \textsc{Richard
Garbe}. 25c (1s.~6d.).

\item[]\hspace{-3ex}\llap{18.}
\textit{An Examination of Weismannism}. By \textsc{G.~J. Romanes}. 35c (2s.).

\item[]\hspace{-3ex}\llap{19.}
\textit{On Germinal Selection}. By \textsc{August Weismann}. 25c (1s.~6d.).

\item[]\hspace{-3ex}\llap{20.}
\textit{Lovers Three Thousand Years Ago}, By \textsc{T.~A. Goodwin}. (Out of print.)

\item[]\hspace{-3ex}\llap{21.}
\textit{Popular Scientific Lectures}. By \textsc{Ernst Mach}, 50c (2s.~6d.).

\item[]\hspace{-3ex}\llap{22.}
\textit{Ancient India: Its Language and Religions}. By \textsc{H.~Oldenberg}. 25c
(1s.~6d.).

\item[]\hspace{-3ex}\llap{23.}
\textit{The Prophets of Israel}. By Prof. \textsc{C.~H. Cornill.} 25c (1s.~6d.).

\item[]\hspace{-3ex}\llap{24.}
\textit{Homilies of Science}, By \textsc{Paul Carus}. 35c (2s.).

\item[]\hspace{-3ex}\llap{25.}
\textit{Thoughts on Religion}. By \textsc{G.~J. Romanes}. 50c (2s.~6d.).

\item[]\hspace{-3ex}\llap{26.}
\textit{The Philosophy of Ancient India}. By \textsc{Prof.\ Richard Garbe}. 25c (1s.~6d.)

\item[]\hspace{-3ex}\llap{27.}
\textit{Martin Luther}. By \textsc{Gustav Freytag}. 25c (1s.~6d.).

\item[]\hspace{-3ex}\llap{28.}
\textit{English Secularism}. By \textsc{George Jacob Holyoake}. 25c (1s.~6d.).

\item[]\hspace{-3ex}\llap{29.}
\textit{On Orthogenesis}. By \textsc{Th.~Eimer}. 25c (1s.~6d.).

\item[]\hspace{-3ex}\llap{30.}
\textit{Chinese Philosophy}. By \textsc{Paul Carus}. 25c (1s.~6d.).

\item[]\hspace{-3ex}\llap{31.}
\textit{The Lost Manuscript}. By \textsc{Gustav Freytag}. 60c (3s.).

\item[]\hspace{-3ex}\llap{32.}
\textit{A Mechanico-Physiological Theory of Organic Evolution}. By \textsc{Carl von
Naegeli}. 15c (9d.).

\item[]\hspace{-3ex}\llap{33.}
\textit{Chinese Fiction}. By \textsc{Dr.\ George T.\ Candlin}. 15c (9d.).

\item[]\hspace{-3ex}\llap{34.}
\textit{Mathematical Essays and Recreations}. By \textsc{H.~Schubert}. 25c (1s.~6d.)

\item[]\hspace{-3ex}\llap{35.}
\textit{The Ethical Problem}. By \textsc{Paul Carus}. 50c (2s.~6d.).

\item[]\hspace{-3ex}\llap{36.}
\textit{Buddhism, and Its Christian Critics}. By \textsc{Paul Carus}. 50c (2s.~6d.).

\item[]\hspace{-3ex}\llap{37.}
\textit{Psychology for Beginners}. By \textsc{Hiram M.\ Stanley}. 20c (1s.).

\item[]\hspace{-3ex}\llap{38.}
\textit{Discourse on Method}. By \textsc{Descartes}. 25c (1s.~6d.).

\item[]\hspace{-3ex}\llap{39.}
\textit{The Dawn of a New Era}. By \textsc{Paul Carus}. 15c (9d.).

\item[]\hspace{-3ex}\llap{40.}
\textit{Kant and Spencer}. By \textsc{Paul Carus}. 20c (1s.).

\item[]\hspace{-3ex}\llap{41.}
\textit{The Soul of Man}. By \textsc{Paul Carus}. 75c (3s.~6d.).

\item[]\hspace{-3ex}\llap{42.}
\textit{World's Congress Addresses}. By \textsc{C.~C. Bonney}. 15c (9d.).

\item[]\hspace{-3ex}\llap{43.} \textit{The Gospel According to Darwin}. By \textsc{Woods Hutchinson}. 50c (2s.~6d.)

\item[]\hspace{-3ex}\llap{44.}
\textit{Whence and Whither}. By \textsc{Paul Carus}. 25c (1s.~6d.).

\item[]\hspace{-3ex}\llap{45.}
\textit{Enquiry Concerning Human Understanding}. By \textsc{David Hume.} 25c
(1s.~6d.).

\item[]\hspace{-3ex}\llap{46.}
\textit{Enquiry Concerning the Principles of Morals}. By \textsc{David Hume}.
25c (1s.~6d.)
\end{itemize}
\end{small}

\begin{center}
\smallskip
\rule[.5ex]{2cm}{.2pt}

\medskip
{\large THE OPEN COURT PUBLISHING CO.,}

\textsc{\small CHICAGO: 324 Dearborn Street.}

{\small LONDON: Kegan Paul, Trench, Tr�bner \& Company, Ltd.}
\end{center}

% [File: 126.png]
%[Blank Page]

\newpage

\begin{center}\textsc{Typographical Errors corrected in Project Gutenberg edition}\end{center}

p.~\pageref{chain} ``$A' \partof B$, $B' \partof B$, $C' \partof C$'' in original,
amended to ``$A' \partof A$'' etc\@.

p.~\pageref{psi1} ``the transformation $\psi$, is therefore completely defined'' in original,
amended to ``$\psi_1$''.

\newpage
\small
\pagenumbering{gobble}
\begin{verbatim}

End of Project Gutenberg's Essays on the Theory of Numbers,
by Richard Dedekind

*** END OF THIS PROJECT GUTENBERG EBOOK THEORY OF NUMBERS ***

*** This file should be named 21016-t.tex or 21016-t.zip ***
*** or                    21016-pdf.pdf or 21016-pdf.pdf ***
This and all associated files of various formats will be found in:
        http://www.gutenberg.org/2/1/0/1/21016/


Produced by Jonathan Ingram, Keith Edkins and the Online
Distributed Proofreading Team at http://www.pgdp.net


Updated editions will replace the previous one--the old editions
will be renamed.

Creating the works from public domain print editions means that no
one owns a United States copyright in these works, so the Foundation
(and you!) can copy and distribute it in the United States without
permission and without paying copyright royalties.  Special rules,
set forth in the General Terms of Use part of this license, apply to
copying and distributing Project Gutenberg-tm electronic works to
protect the PROJECT GUTENBERG-tm concept and trademark.  Project
Gutenberg is a registered trademark, and may not be used if you
charge for the eBooks, unless you receive specific permission.  If
you do not charge anything for copies of this eBook, complying with
the rules is very easy.  You may use this eBook for nearly any
purpose such as creation of derivative works, reports, performances
and research.  They may be modified and printed and given away--you
may do practically ANYTHING with public domain eBooks.
Redistribution is subject to the trademark license, especially
commercial redistribution.



*** START: FULL LICENSE ***

THE FULL PROJECT GUTENBERG LICENSE PLEASE READ THIS BEFORE YOU
DISTRIBUTE OR USE THIS WORK

To protect the Project Gutenberg-tm mission of promoting the free
distribution of electronic works, by using or distributing this work
(or any other work associated in any way with the phrase "Project
Gutenberg"), you agree to comply with all the terms of the Full
Project Gutenberg-tm License (available with this file or online at
http://gutenberg.net/license).


Section 1.  General Terms of Use and Redistributing Project
Gutenberg-tm electronic works

1.A.  By reading or using any part of this Project Gutenberg-tm
electronic work, you indicate that you have read, understand, agree
to and accept all the terms of this license and intellectual
property (trademark/copyright) agreement.  If you do not agree to
abide by all the terms of this agreement, you must cease using and
return or destroy all copies of Project Gutenberg-tm electronic
works in your possession. If you paid a fee for obtaining a copy of
or access to a Project Gutenberg-tm electronic work and you do not
agree to be bound by the terms of this agreement, you may obtain a
refund from the person or entity to whom you paid the fee as set
forth in paragraph 1.E.8.

1.B.  "Project Gutenberg" is a registered trademark.  It may only be
used on or associated in any way with an electronic work by people
who agree to be bound by the terms of this agreement.  There are a
few things that you can do with most Project Gutenberg-tm electronic
works even without complying with the full terms of this agreement.
See paragraph 1.C below.  There are a lot of things you can do with
Project Gutenberg-tm electronic works if you follow the terms of
this agreement and help preserve free future access to Project
Gutenberg-tm electronic works.  See paragraph 1.E below.

1.C.  The Project Gutenberg Literary Archive Foundation ("the
Foundation" or PGLAF), owns a compilation copyright in the
collection of Project Gutenberg-tm electronic works.  Nearly all the
individual works in the collection are in the public domain in the
United States.  If an individual work is in the public domain in the
United States and you are located in the United States, we do not
claim a right to prevent you from copying, distributing, performing,
displaying or creating derivative works based on the work as long as
all references to Project Gutenberg are removed.  Of course, we hope
that you will support the Project Gutenberg-tm mission of promoting
free access to electronic works by freely sharing Project
Gutenberg-tm works in compliance with the terms of this agreement
for keeping the Project Gutenberg-tm name associated with the work.
You can easily comply with the terms of this agreement by keeping
this work in the same format with its attached full Project
Gutenberg-tm License when you share it without charge with others.

1.D.  The copyright laws of the place where you are located also
govern what you can do with this work.  Copyright laws in most
countries are in a constant state of change.  If you are outside the
United States, check the laws of your country in addition to the
terms of this agreement before downloading, copying, displaying,
performing, distributing or creating derivative works based on this
work or any other Project Gutenberg-tm work.  The Foundation makes
no representations concerning the copyright status of any work in
any country outside the United States.

1.E.  Unless you have removed all references to Project Gutenberg:

1.E.1.  The following sentence, with active links to, or other
immediate access to, the full Project Gutenberg-tm License must
appear prominently whenever any copy of a Project Gutenberg-tm work
(any work on which the phrase "Project Gutenberg" appears, or with
which the phrase "Project Gutenberg" is associated) is accessed,
displayed, performed, viewed, copied or distributed:

This eBook is for the use of anyone anywhere at no cost and with
almost no restrictions whatsoever.  You may copy it, give it away or
re-use it under the terms of the Project Gutenberg License included
with this eBook or online at www.gutenberg.net

1.E.2.  If an individual Project Gutenberg-tm electronic work is
derived from the public domain (does not contain a notice indicating
that it is posted with permission of the copyright holder), the work
can be copied and distributed to anyone in the United States without
paying any fees or charges.  If you are redistributing or providing
access to a work with the phrase "Project Gutenberg" associated with
or appearing on the work, you must comply either with the
requirements of paragraphs 1.E.1 through 1.E.7 or obtain permission
for the use of the work and the Project Gutenberg-tm trademark as
set forth in paragraphs 1.E.8 or 1.E.9.

1.E.3.  If an individual Project Gutenberg-tm electronic work is
posted with the permission of the copyright holder, your use and
distribution must comply with both paragraphs 1.E.1 through 1.E.7
and any additional terms imposed by the copyright holder.
Additional terms will be linked to the Project Gutenberg-tm License
for all works posted with the permission of the copyright holder
found at the beginning of this work.

1.E.4.  Do not unlink or detach or remove the full Project
Gutenberg-tm License terms from this work, or any files containing a
part of this work or any other work associated with Project
Gutenberg-tm.

1.E.5.  Do not copy, display, perform, distribute or redistribute
this electronic work, or any part of this electronic work, without
prominently displaying the sentence set forth in paragraph 1.E.1
with active links or immediate access to the full terms of the
Project Gutenberg-tm License.

1.E.6.  You may convert to and distribute this work in any binary,
compressed, marked up, nonproprietary or proprietary form, including
any word processing or hypertext form.  However, if you provide
access to or distribute copies of a Project Gutenberg-tm work in a
format other than "Plain Vanilla ASCII" or other format used in the
official version posted on the official Project Gutenberg-tm web
site (www.gutenberg.net), you must, at no additional cost, fee or
expense to the user, provide a copy, a means of exporting a copy, or
a means of obtaining a copy upon request, of the work in its
original "Plain Vanilla ASCII" or other form.  Any alternate format
must include the full Project Gutenberg-tm License as specified in
paragraph 1.E.1.

1.E.7.  Do not charge a fee for access to, viewing, displaying,
performing, copying or distributing any Project Gutenberg-tm works
unless you comply with paragraph 1.E.8 or 1.E.9.

1.E.8.  You may charge a reasonable fee for copies of or providing
access to or distributing Project Gutenberg-tm electronic works
provided that

- You pay a royalty fee of 20% of the gross profits you derive from
   the use of Project Gutenberg-tm works calculated using the method
   you already use to calculate your applicable taxes.  The fee is
   owed to the owner of the Project Gutenberg-tm trademark, but he
   has agreed to donate royalties under this paragraph to the
   Project Gutenberg Literary Archive Foundation.  Royalty payments
   must be paid within 60 days following each date on which you
   prepare (or are legally required to prepare) your periodic tax
   returns.  Royalty payments should be clearly marked as such and
   sent to the Project Gutenberg Literary Archive Foundation at the
   address specified in Section 4, "Information about donations to
   the Project Gutenberg Literary Archive Foundation."

- You provide a full refund of any money paid by a user who notifies
   you in writing (or by e-mail) within 30 days of receipt that s/he
   does not agree to the terms of the full Project Gutenberg-tm
   License.  You must require such a user to return or
   destroy all copies of the works possessed in a physical medium
   and discontinue all use of and all access to other copies of
   Project Gutenberg-tm works.

- You provide, in accordance with paragraph 1.F.3, a full refund of
   any money paid for a work or a replacement copy, if a defect in
   the electronic work is discovered and reported to you within 90
   days of receipt of the work.

- You comply with all other terms of this agreement for free
   distribution of Project Gutenberg-tm works.

1.E.9.  If you wish to charge a fee or distribute a Project
Gutenberg-tm electronic work or group of works on different terms
than are set forth in this agreement, you must obtain permission in
writing from both the Project Gutenberg Literary Archive Foundation
and Michael Hart, the owner of the Project Gutenberg-tm trademark.
Contact the Foundation as set forth in Section 3 below.

1.F.

1.F.1.  Project Gutenberg volunteers and employees expend
considerable effort to identify, do copyright research on,
transcribe and proofread public domain works in creating the Project
Gutenberg-tm collection.  Despite these efforts, Project
Gutenberg-tm electronic works, and the medium on which they may be
stored, may contain "Defects," such as, but not limited to,
incomplete, inaccurate or corrupt data, transcription errors, a
copyright or other intellectual property infringement, a defective
or damaged disk or other medium, a computer virus, or computer codes
that damage or cannot be read by your equipment.

1.F.2.  LIMITED WARRANTY, DISCLAIMER OF DAMAGES - Except for the
"Right of Replacement or Refund" described in paragraph 1.F.3, the
Project Gutenberg Literary Archive Foundation, the owner of the
Project Gutenberg-tm trademark, and any other party distributing a
Project Gutenberg-tm electronic work under this agreement, disclaim
all liability to you for damages, costs and expenses, including
legal fees.  YOU AGREE THAT YOU HAVE NO REMEDIES FOR NEGLIGENCE,
STRICT LIABILITY, BREACH OF WARRANTY OR BREACH OF CONTRACT EXCEPT
THOSE PROVIDED IN PARAGRAPH F3.  YOU AGREE THAT THE FOUNDATION, THE
TRADEMARK OWNER, AND ANY DISTRIBUTOR UNDER THIS AGREEMENT WILL NOT
BE LIABLE TO YOU FOR ACTUAL, DIRECT, INDIRECT, CONSEQUENTIAL,
PUNITIVE OR INCIDENTAL DAMAGES EVEN IF YOU GIVE NOTICE OF THE
POSSIBILITY OF SUCH DAMAGE.

1.F.3.  LIMITED RIGHT OF REPLACEMENT OR REFUND - If you discover a
defect in this electronic work within 90 days of receiving it, you
can receive a refund of the money (if any) you paid for it by
sending a written explanation to the person you received the work
from.  If you received the work on a physical medium, you must
return the medium with your written explanation.  The person or
entity that provided you with the defective work may elect to
provide a replacement copy in lieu of a refund.  If you received the
work electronically, the person or entity providing it to you may
choose to give you a second opportunity to receive the work
electronically in lieu of a refund.  If the second copy is also
defective, you may demand a refund in writing without further
opportunities to fix the problem.

1.F.4.  Except for the limited right of replacement or refund set
forth in paragraph 1.F.3, this work is provided to you 'AS-IS', WITH
NO OTHER WARRANTIES OF ANY KIND, EXPRESS OR IMPLIED, INCLUDING BUT
NOT LIMITED TO WARRANTIES OF MERCHANTIBILITY OR FITNESS FOR ANY
PURPOSE.

1.F.5.  Some states do not allow disclaimers of certain implied
warranties or the exclusion or limitation of certain types of
damages. If any disclaimer or limitation set forth in this agreement
violates the law of the state applicable to this agreement, the
agreement shall be interpreted to make the maximum disclaimer or
limitation permitted by the applicable state law.  The invalidity or
unenforceability of any provision of this agreement shall not void
the remaining provisions.

1.F.6.  INDEMNITY - You agree to indemnify and hold the Foundation,
the trademark owner, any agent or employee of the Foundation, anyone
providing copies of Project Gutenberg-tm electronic works in
accordance with this agreement, and any volunteers associated with
the production, promotion and distribution of Project Gutenberg-tm
electronic works, harmless from all liability, costs and expenses,
including legal fees, that arise directly or indirectly from any of
the following which you do or cause to occur: (a) distribution of
this or any Project Gutenberg-tm work, (b) alteration, modification,
or additions or deletions to any Project Gutenberg-tm work, and (c)
any Defect you cause.


Section  2.  Information about the Mission of Project Gutenberg-tm

Project Gutenberg-tm is synonymous with the free distribution of
electronic works in formats readable by the widest variety of
computers including obsolete, old, middle-aged and new computers.
It exists because of the efforts of hundreds of volunteers and
donations from people in all walks of life.

Volunteers and financial support to provide volunteers with the
assistance they need, is critical to reaching Project Gutenberg-tm's
goals and ensuring that the Project Gutenberg-tm collection will
remain freely available for generations to come.  In 2001, the
Project Gutenberg Literary Archive Foundation was created to provide
a secure and permanent future for Project Gutenberg-tm and future
generations. To learn more about the Project Gutenberg Literary
Archive Foundation and how your efforts and donations can help, see
Sections 3 and 4 and the Foundation web page at
http://www.pglaf.org.


Section 3.  Information about the Project Gutenberg Literary Archive
Foundation

The Project Gutenberg Literary Archive Foundation is a non profit
501(c)(3) educational corporation organized under the laws of the
state of Mississippi and granted tax exempt status by the Internal
Revenue Service.  The Foundation's EIN or federal tax identification
number is 64-6221541.  Its 501(c)(3) letter is posted at
http://pglaf.org/fundraising.  Contributions to the Project
Gutenberg Literary Archive Foundation are tax deductible to the full
extent permitted by U.S. federal laws and your state's laws.

The Foundation's principal office is located at 4557 Melan Dr. S.
Fairbanks, AK, 99712., but its volunteers and employees are
scattered throughout numerous locations.  Its business office is
located at 809 North 1500 West, Salt Lake City, UT 84116, (801)
596-1887, email business@pglaf.org.  Email contact links and up to
date contact information can be found at the Foundation's web site
and official page at http://pglaf.org

For additional contact information:
     Dr. Gregory B. Newby
     Chief Executive and Director
     gbnewby@pglaf.org

Section 4.  Information about Donations to the Project Gutenberg
Literary Archive Foundation

Project Gutenberg-tm depends upon and cannot survive without wide
spread public support and donations to carry out its mission of
increasing the number of public domain and licensed works that can
be freely distributed in machine readable form accessible by the
widest array of equipment including outdated equipment.  Many small
donations ($1 to $5,000) are particularly important to maintaining
tax exempt status with the IRS.

The Foundation is committed to complying with the laws regulating
charities and charitable donations in all 50 states of the United
States.  Compliance requirements are not uniform and it takes a
considerable effort, much paperwork and many fees to meet and keep
up with these requirements.  We do not solicit donations in
locations where we have not received written confirmation of
compliance.  To SEND DONATIONS or determine the status of compliance
for any particular state visit http://pglaf.org

While we cannot and do not solicit contributions from states where
we have not met the solicitation requirements, we know of no
prohibition against accepting unsolicited donations from donors in
such states who approach us with offers to donate.

International donations are gratefully accepted, but we cannot make
any statements concerning tax treatment of donations received from
outside the United States.  U.S. laws alone swamp our small staff.

Please check the Project Gutenberg Web pages for current donation
methods and addresses.  Donations are accepted in a number of other
ways including including checks, online payments and credit card
donations.  To donate, please visit: http://pglaf.org/donate


Section 5.  General Information About Project Gutenberg-tm
electronic works.

Professor Michael S. Hart is the originator of the Project
Gutenberg-tm concept of a library of electronic works that could be
freely shared with anyone.  For thirty years, he produced and
distributed Project Gutenberg-tm eBooks with only a loose network of
volunteer support.

Project Gutenberg-tm eBooks are often created from several printed
editions, all of which are confirmed as Public Domain in the U.S.
unless a copyright notice is included.  Thus, we do not necessarily
keep eBooks in compliance with any particular paper edition.

Most people start at our Web site which has the main PG search
facility:

     http://www.gutenberg.net

This Web site includes information about Project Gutenberg-tm,
including how to make donations to the Project Gutenberg Literary
Archive Foundation, how to help produce our new eBooks, and how to
subscribe to our email newsletter to hear about new eBooks.

*** END: FULL LICENSE ***
\end{verbatim}
\end{document}
%% ---------------------------------------------------------
%% Below is appended the log from the most recent compile.
%% You may use it to compare against a log from a new
%% compile to help spot differences.
%%   (Comment marks so it doesn't pass through lprep!)
%%This is pdfeTeX, Version 3.141592-1.30.6-2.2 (MiKTeX 2.5) (preloaded format=pdflatex 2006.11.12)  9 APR 2007 08:52
%%entering extended mode
%%**21016-t
%%(21016-t.tex
%%LaTeX2e <2005/12/01>
%%Babel <v3.8g> and hyphenation patterns for english, dumylang, nohyphenation, ge
%%rman, ngerman, french, loaded.
%%(C:\MiKTeX\tex\latex\base\book.cls
%%Document Class: book 2005/09/16 v1.4f Standard LaTeX document class
%%(C:\MiKTeX\tex\latex\base\bk10.clo
%%File: bk10.clo 2005/09/16 v1.4f Standard LaTeX file (size option)
%%)
%%\c@part=\count79
%%\c@chapter=\count80
%%\c@section=\count81
%%\c@subsection=\count82
%%\c@subsubsection=\count83
%%\c@paragraph=\count84
%%\c@subparagraph=\count85
%%\c@figure=\count86
%%\c@table=\count87
%%\abovecaptionskip=\skip41
%%\belowcaptionskip=\skip42
%%\bibindent=\dimen102
%%) (C:\MiKTeX\tex\latex\amsmath\amsmath.sty
%%Package: amsmath 2000/07/18 v2.13 AMS math features
%%\@mathmargin=\skip43
%%
%%For additional information on amsmath, use the `?' option.
%%(C:\MiKTeX\tex\latex\amsmath\amstext.sty
%%Package: amstext 2000/06/29 v2.01
%%
%%(C:\MiKTeX\tex\latex\amsmath\amsgen.sty
%%File: amsgen.sty 1999/11/30 v2.0
%%\@emptytoks=\toks14
%%\ex@=\dimen103
%%))
%%(C:\MiKTeX\tex\latex\amsmath\amsbsy.sty
%%Package: amsbsy 1999/11/29 v1.2d
%%\pmbraise@=\dimen104
%%)
%%(C:\MiKTeX\tex\latex\amsmath\amsopn.sty
%%Package: amsopn 1999/12/14 v2.01 operator names
%%)
%%\inf@bad=\count88
%%LaTeX Info: Redefining \frac on input line 211.
%%\uproot@=\count89
%%\leftroot@=\count90
%%LaTeX Info: Redefining \overline on input line 307.
%%\classnum@=\count91
%%\DOTSCASE@=\count92
%%LaTeX Info: Redefining \ldots on input line 379.
%%LaTeX Info: Redefining \dots on input line 382.
%%LaTeX Info: Redefining \cdots on input line 467.
%%\Mathstrutbox@=\box26
%%\strutbox@=\box27
%%\big@size=\dimen105
%%LaTeX Font Info:    Redeclaring font encoding OML on input line 567.
%%LaTeX Font Info:    Redeclaring font encoding OMS on input line 568.
%%\macc@depth=\count93
%%\c@MaxMatrixCols=\count94
%%\dotsspace@=\muskip10
%%\c@parentequation=\count95
%%\dspbrk@lvl=\count96
%%\tag@help=\toks15
%%\row@=\count97
%%\column@=\count98
%%\maxfields@=\count99
%%\andhelp@=\toks16
%%\eqnshift@=\dimen106
%%\alignsep@=\dimen107
%%\tagshift@=\dimen108
%%\tagwidth@=\dimen109
%%\totwidth@=\dimen110
%%\lineht@=\dimen111
%%\@envbody=\toks17
%%\multlinegap=\skip44
%%\multlinetaggap=\skip45
%%\mathdisplay@stack=\toks18
%%LaTeX Info: Redefining \[ on input line 2666.
%%LaTeX Info: Redefining \] on input line 2667.
%%)
%%(C:\MiKTeX\tex\latex\amsfonts\amssymb.sty
%%Package: amssymb 2002/01/22 v2.2d
%%
%%(C:\MiKTeX\tex\latex\amsfonts\amsfonts.sty
%%Package: amsfonts 2001/10/25 v2.2f
%%\symAMSa=\mathgroup4
%%\symAMSb=\mathgroup5
%%LaTeX Font Info:    Overwriting math alphabet `\mathfrak' in version `bold'
%%(Font)                  U/euf/m/n --> U/euf/b/n on input line 132.
%%))
%%(C:\MiKTeX\tex\generic\babel\babel.sty
%%Package: babel 2005/05/21 v3.8g The Babel package
%%
%%*************************************
%%* Local config file bblopts.cfg used
%%*
%%(C:\MiKTeX\tex\latex\00miktex\bblopts.cfg
%%File: bblopts.cfg 2006/07/31 v1.0 MiKTeX 'babel' configuration
%%)
%%(C:\MiKTeX\tex\generic\babel\greek.ldf
%%Language: greek 2005/03/30 v1.3l Greek support from the babel system
%% (C:\MiKTeX\tex\generic\babel\babel.def
%%File: babel.def 2005/05/21 v3.8g Babel common definitions
%%\babel@savecnt=\count100
%%\U@D=\dimen112
%%)
%%
%%Package babel Warning: No hyphenation patterns were loaded for
%%(babel)                the language `greek'
%%(babel)                I will use the patterns loaded for \language=0 instead.
%%
%%\l@greek = a dialect from \language0
%%Loading the definitions for the Greek font encoding
%%(C:\MiKTeX\tex\generic\babel\lgrenc.def
%%File: lgrenc.def 2001/01/30 v2.2e Greek Encoding
%%))
%%(C:\MiKTeX\tex\generic\babel\english.ldf
%%Language: english 2005/03/30 v3.3o English support from the babel system
%%\l@british = a dialect from \language\l@english 
%%\l@UKenglish = a dialect from \language\l@english 
%%\l@canadian = a dialect from \language\l@american 
%%\l@australian = a dialect from \language\l@british 
%%\l@newzealand = a dialect from \language\l@british 
%%))
%%(C:\MiKTeX\tex\latex\base\inputenc.sty
%%Package: inputenc 2006/05/05 v1.1b Input encoding file
%%\inpenc@prehook=\toks19
%%\inpenc@posthook=\toks20
%% (C:\MiKTeX\tex\latex\base\latin1.def
%%File: latin1.def 2006/05/05 v1.1b Input encoding file
%%))
%%(C:\MiKTeX\tex\latex\teubner\teubner.sty
%%Package: teubner 2006/01/14 v.2.2c extensions for Greek philology
%% (C:\MiKTeX\tex\generic\babel\lgrcmr.fd
%%File: lgrcmr.fd 2001/01/30 v2.2e Greek Computer Modern
%%) (C:\MiKTeX\tex\latex\base\exscale.sty
%%Package: exscale 1997/06/16 v2.1g Standard LaTeX package exscale
%%LaTeX Font Info:    Redeclaring symbol font `largesymbols' on input line 52.
%%LaTeX Font Info:    Overwriting symbol font `largesymbols' in version `normal'
%%(Font)                  OMX/cmex/m/n --> OMX/cmex/m/n on input line 52.
%%LaTeX Font Info:    Overwriting symbol font `largesymbols' in version `bold'
%%(Font)                  OMX/cmex/m/n --> OMX/cmex/m/n on input line 52.
%%\big@size=\dimen113
%%)
%%LaTeX Info: Redefining \textlatin on input line 104.
%%LaTeX Info: Redefining \: on input line 591.
%%LaTeX Info: Redefining \; on input line 592.
%%LaTeX Info: Redefining \| on input line 601.
%%\Uunit=\dimen114
%%LaTeX Info: Redefining \star on input line 774.
%%\c@verso=\count101
%%\c@subverso=\count102
%%\versoskip=\skip46
%%LaTeX Info: Redefining \breve on input line 975.
%%\br@cedmetrics=\skip47
%%LaTeX Info: Redefining \Greeknumeral on input line 1054.
%%LaTeX Info: Redefining \greeknumeral on input line 1057.
%%)
%%(C:\MiKTeX\tex\latex\graphics\graphicx.sty
%%Package: graphicx 1999/02/16 v1.0f Enhanced LaTeX Graphics (DPC,SPQR)
%%
%%(C:\MiKTeX\tex\latex\graphics\keyval.sty
%%Package: keyval 1999/03/16 v1.13 key=value parser (DPC)
%%\KV@toks@=\toks21
%%)
%%(C:\MiKTeX\tex\latex\graphics\graphics.sty
%%Package: graphics 2006/02/20 v1.0o Standard LaTeX Graphics (DPC,SPQR)
%%
%%(C:\MiKTeX\tex\latex\graphics\trig.sty
%%Package: trig 1999/03/16 v1.09 sin cos tan (DPC)
%%)
%%(C:\MiKTeX\tex\latex\00miktex\graphics.cfg
%%File: graphics.cfg 2005/12/29 v1.2 MiKTeX 'graphics' configuration
%%)
%%Package graphics Info: Driver file: pdftex.def on input line 90.
%%
%%(C:\MiKTeX\tex\latex\graphics\pdftex.def
%%File: pdftex.def 2006/08/14 v0.03t Graphics/color for pdfTeX
%%\Gread@gobject=\count103
%%))
%%\Gin@req@height=\dimen115
%%\Gin@req@width=\dimen116
%%)
%%(C:\MiKTeX\tex\latex\mh\mathtools.sty
%%Package: mathtools 2004/10/10 v1.01a mathematical typesetting tools (MH)
%% (C:\MiKTeX\tex\latex\tools\calc.sty
%%Package: calc 2005/08/06 v4.2 Infix arithmetic (KKT,FJ)
%%\calc@Acount=\count104
%%\calc@Bcount=\count105
%%\calc@Adimen=\dimen117
%%\calc@Bdimen=\dimen118
%%\calc@Askip=\skip48
%%\calc@Bskip=\skip49
%%LaTeX Info: Redefining \setlength on input line 75.
%%LaTeX Info: Redefining \addtolength on input line 76.
%%\calc@Ccount=\count106
%%\calc@Cskip=\skip50
%%)
%%(C:\MiKTeX\tex\latex\mh\mhsetup.sty
%%Package: mhsetup 2004/10/10 v1.0b programming setup (MH)
%%)
%%\g_MT_multlinerow_int=\count107
%%\l_MT_multwidth_dim=\dimen119
%%) (21016-t.aux)
%%LaTeX Font Info:    Checking defaults for OML/cmm/m/it on input line 83.
%%LaTeX Font Info:    ... okay on input line 83.
%%LaTeX Font Info:    Checking defaults for T1/cmr/m/n on input line 83.
%%LaTeX Font Info:    ... okay on input line 83.
%%LaTeX Font Info:    Checking defaults for OT1/cmr/m/n on input line 83.
%%LaTeX Font Info:    ... okay on input line 83.
%%LaTeX Font Info:    Checking defaults for OMS/cmsy/m/n on input line 83.
%%LaTeX Font Info:    ... okay on input line 83.
%%LaTeX Font Info:    Checking defaults for OMX/cmex/m/n on input line 83.
%%LaTeX Font Info:    ... okay on input line 83.
%%LaTeX Font Info:    Checking defaults for U/cmr/m/n on input line 83.
%%LaTeX Font Info:    ... okay on input line 83.
%%LaTeX Font Info:    Checking defaults for LGR/cmr/m/n on input line 83.
%%LaTeX Font Info:    ... okay on input line 83.
%%
%%(C:\MiKTeX\tex\context\base\supp-pdf.tex
%%(C:\MiKTeX\tex\context\base\supp-mis.tex
%%loading : Context Support Macros / Miscellaneous (2004.10.26)
%%\protectiondepth=\count108
%%\scratchcounter=\count109
%%\scratchtoks=\toks22
%%\scratchdimen=\dimen120
%%\scratchskip=\skip51
%%\scratchmuskip=\muskip11
%%\scratchbox=\box28
%%\scratchread=\read1
%%\scratchwrite=\write3
%%\zeropoint=\dimen121
%%\onepoint=\dimen122
%%\onebasepoint=\dimen123
%%\minusone=\count110
%%\thousandpoint=\dimen124
%%\onerealpoint=\dimen125
%%\emptytoks=\toks23
%%\nextbox=\box29
%%\nextdepth=\dimen126
%%\everyline=\toks24
%%\!!counta=\count111
%%\!!countb=\count112
%%\recursecounter=\count113
%%)
%%loading : Context Support Macros / PDF (2004.03.26)
%%\nofMPsegments=\count114
%%\nofMParguments=\count115
%%\MPscratchCnt=\count116
%%\MPscratchDim=\dimen127
%%\MPnumerator=\count117
%%\everyMPtoPDFconversion=\toks25
%%) [1
%%
%%{psfonts.map}]
%%LaTeX Font Info:    External font `cmex7' loaded for size
%%(Font)              <7> on input line 129.
%%LaTeX Font Info:    External font `cmex7' loaded for size
%%(Font)              <5> on input line 129.
%%LaTeX Font Info:    Try loading font information for U+msa on input line 129.
%% (C:\MiKTeX\tex\latex\amsfonts\umsa.fd
%%File: umsa.fd 2002/01/19 v2.2g AMS font definitions
%%)
%%LaTeX Font Info:    Try loading font information for U+msb on input line 129.
%%
%%(C:\MiKTeX\tex\latex\amsfonts\umsb.fd
%%File: umsb.fd 2002/01/19 v2.2g AMS font definitions
%%)
%%LaTeX Font Info:    Try loading font information for U+euf on input line 129.
%% (C:\MiKTeX\tex\latex\amsfonts\ueuf.fd
%%File: ueuf.fd 2002/01/19 v2.2g AMS font definitions
%%)
%%[1
%%
%%] [2] [3] [4] [1
%%
%%]
%%LaTeX Font Info:    External font `cmex7' loaded for size
%%(Font)              <6> on input line 461.
%%LaTeX Font Info:    Try loading font information for OMS+cmr on input line 461.
%%
%% (C:\MiKTeX\tex\latex\base\omscmr.fd
%%File: omscmr.fd 1999/05/25 v2.5h Standard LaTeX font definitions
%%)
%%LaTeX Font Info:    Font shape `OMS/cmr/m/n' in size <8> not available
%%(Font)              Font shape `OMS/cmsy/m/n' tried instead on input line 461.
%% [2] [3] [4] [5]
%%[6] [7] [8] [9] [10] [11] [12] [13] [14] [15] [16] [17] [18]
%%LaTeX Font Info:    Font shape `OMS/cmr/m/n' in size <10> not available
%%(Font)              Font shape `OMS/cmsy/m/n' tried instead on input line 1550.
%%
%% [19] [20] [21]
%%[22] [23] [24] [25] [26] [27] [28] [29] [30] [31] [32] [33] [34] [35] [36]
%%[37] [38] [39] [40] [41] [42] [43] [44] [45] [46] [47] [48] [49] [50] [51]
%%[52] [53] [54] [55] [56] [57] [58] [59] [60] [61] [62] [63] [64] [65] [1]
%%[2] [3] [4] [5] [6] [7] [8] (21016-t.aux) ) 
%%Here is how much of TeX's memory you used:
%% 3718 strings out of 95419
%% 39891 string characters out of 1187819
%% 110668 words of memory out of 1106176
%% 6710 multiletter control sequences out of 60000
%% 15010 words of font info for 52 fonts, out of 1000000 for 2000
%% 14 hyphenation exceptions out of 8191
%% 27i,11n,25p,236b,250s stack positions out of 5000i,500n,10000p,200000b,32768s
%%PDF statistics:
%% 329 PDF objects out of 300000
%% 0 named destinations out of 300000
%% 1 words of extra memory for PDF output out of 10000
%%<C:/MiKTeX/fonts/type1/bluesky/cm/cm
%%ti12.pfb><C:/MiKTeX/fonts/type1/bluesky/cm/cmsl8.pfb><C:/MiKTeX/fonts/type1/blu
%%esky/cm/cmsy6.pfb><C:/MiKTeX/fonts/type1/bluesky/cm/cmmi6.pfb><C:/MiKTeX/fonts/
%%type1/bluesky/cm/cmsy5.pfb><C:/MiKTeX/fonts/type1/bluesky/cm/cmmi7.pfb><C:/MiKT
%%eX/fonts/type1/bluesky/cm/cmmi8.pfb><C:/MiKTeX/fonts/type1/bluesky/cm/cmex10.pf
%%b><C:/MiKTeX/fonts/type1/bluesky/cm/cmbx10.pfb> <C:\Documents and Settings\All 
%%Users\Application Data\MiKTeX\2.5\fonts\pk\ljfour\public\cb\dpi600\grml1000.pk>
%%<C:/MiKTeX/fonts/type1/bluesky/cm/cmsy7.pfb><C:/MiKTeX/fonts/type1/bluesky/ams/
%%msam10.pfb><C:/MiKTeX/fonts/type1/bluesky/cm/cmsy10.pfb><C:/MiKTeX/fonts/type1/
%%bluesky/cm/cmsy8.pfb><C:/MiKTeX/fonts/type1/bluesky/cm/cmti8.pfb><C:/MiKTeX/fon
%%ts/type1/bluesky/cm/cmr6.pfb><C:/MiKTeX/fonts/type1/bluesky/cm/cmmi10.pfb><C:/M
%%iKTeX/fonts/type1/bluesky/cm/cmr8.pfb><C:/MiKTeX/fonts/type1/bluesky/cm/cmr17.p
%%fb><C:/MiKTeX/fonts/type1/bluesky/cm/cmr7.pfb><C:/MiKTeX/fonts/type1/bluesky/cm
%%/cmti9.pfb><C:/MiKTeX/fonts/type1/bluesky/cm/cmr9.pfb><C:/MiKTeX/fonts/type1/bl
%%uesky/cm/cmr12.pfb><C:/MiKTeX/fonts/type1/bluesky/cm/cmr10.pfb><C:/MiKTeX/fonts
%%/type1/bluesky/ams/eufm10.pfb><C:/MiKTeX/fonts/type1/bluesky/cm/cmti10.pfb><C:/
%%MiKTeX/fonts/type1/bluesky/cm/cmcsc10.pfb><C:/MiKTeX/fonts/type1/bluesky/cm/cmt
%%t9.pfb>
%%Output written on 21016-t.pdf (78 pages, 468138 bytes).
